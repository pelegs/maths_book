\usepackage{booktabs}
\usepackage{blindtext}
\usepackage{xargs}

%-----------%
% XREF DICT %
%-----------%

\usepackage{pgfkeys}
\pgfkeyssetvalue{/xref dict/fig}{Figure}
\pgfkeyssetvalue{/xref dict/eq}{Equation}
\pgfkeyssetvalue{/xref dict/eqs}{Equations}
\pgfkeyssetvalue{/xref dict/tab}{Table}
\pgfkeyssetvalue{/xref dict/def}{Definition}
\pgfkeyssetvalue{/xref dict/note}{Note}
\pgfkeyssetvalue{/xref dict/challenge}{Challenge}
\pgfkeyssetvalue{/xref dict/example}{Example}
\pgfkeyssetvalue{/xref dict/0}{tapir_rainbow}
\pgfkeyssetvalue{/xref dict/1}{tapir_voronoi}
\pgfkeyssetvalue{/xref dict/2}{tapir_stars}
\pgfkeyssetvalue{/xref dict/3}{tapir_comics}
\pgfkeyssetvalue{/xref dict/4}{tapir_strawberry_two}
\pgfkeyssetvalue{/xref dict/5}{tapir_fourier}
\pgfkeyssetvalue{/xref dict/6}{tapir_warhol}
\pgfkeyssetvalue{/xref dict/7}{tapir_butterfly}

%----------%
% GRAPHICS %
%----------%

\usepackage{svg}
\usepackage{caption}
\usepackage{subcaption}

\usepackage{tikz}
\usepackage{tikz-3dplot}
\usepackage{tikzpagenodes}
\usetikzlibrary{backgrounds, positioning, calc, decorations.pathreplacing, decorations.text, angles, quotes, perspective, tikzmark, fpu}

\makeatletter
\tikzoption{canvas is plane}[]{\@setOxy#1}
\def\@setOxy O(#1,#2,#3)x(#4,#5,#6)y(#7,#8,#9)%
  {\def\tikz@plane@origin{\pgfpointxyz{#1}{#2}{#3}}%
   \def\tikz@plane@x{\pgfpointxyz{#4}{#5}{#6}}%
   \def\tikz@plane@y{\pgfpointxyz{#7}{#8}{#9}}%
   \tikz@canvas@is@plane
  }
\makeatother

\tikzset{
	arrow/.style={thick, ->, >=stealth},
	vector/.style={ultra thick, ->, >=stealth, cap=round},
	complex/.style={circle, minimum size=5pt, fill, inner sep=0pt},
	axisline/.style={thick, stealth-stealth}
}
\newcommand{\draworigin}{
	\fill (0,0) circle (0.075);
}

\usepackage{pgfplots}
\usepgfplotslibrary{fillbetween, colormaps, colorbrewer, patchplots}
\usepgfmodule{nonlineartransformations}
\pgfplotsset{
		compat=1.16,
		every axis/.append style={
			font=\small,
		},
		graph2d/.style={
		axis x line=middle,
		axis y line=middle,
		every axis x label/.style={
			at={(ticklabel* cs:1.01)},
			anchor=west,
		},
		every axis y label/.style={
			at={(ticklabel* cs:1.01)},
			anchor=south,
		},
		axis line style={stealth-stealth, thick},
		label style={font=\large},
		xlabel=$x$,
		ylabel=$y$,
		tick label style={font=\small},
		samples=200,
		grid=both,
		grid style={line width=.1pt, draw=gray!20},
		major grid style={line width=.2pt,draw=gray!50},
		minor tick num=4,
	},
	function/.style={line width=1.5pt},
}
\pgfkeys{
	/pgfplots/vector plane/.style={
		axis x line=middle,
		axis y line=middle,
		xlabel=$x$,
		ylabel=$y$,
		every axis x label/.style={
			at={(ticklabel* cs:1.02)},
			anchor=west,
		},
		every axis y label/.style={
			at={(ticklabel* cs:1.02)},
			anchor=south,
		},
		axis line style={stealth-stealth, thick},
		label style={font=\large},
		tick label style={font=\large},
		samples=100,
		xmin=-5, xmax=5,
		ymin=-5, ymax=5,
		domain=-5:5,
		grid=both,
		major grid style={black!5},
		minor grid style={black!5},
	},
	/pgfplots/tapir frame/.style={
		anchor=center,
		hide axis,
		axis lines=left,
		xtick=\empty,
		ytick=\empty,
		xmin=-0.6, xmax=0.6,
		ymin=-0.5, ymax=0.5,
	},
}

\newcommand{\tapirTransComp}[7]{
	\pgftransformcm{#1}{#2}{#3}{#4}{\pgfpoint{#5}{#6}}
	\begin{axis}[tapir frame]
		\node (#7) at (0,0) {\includesvg[scale=0.75]{figures/linear_algebra/tapir_transform2}};
	\end{axis}
	\pgftransformreset
}			

%-------%
% FONTS %
%-------%

\usepackage{fontawesome}
\usepackage{kpfonts}

%--------%
% TABLES %
%--------%

\usepackage{multirow}
\usepackage{array}

%--------%
% COLORS %
%--------%

\usepackage{xcolor}
\usepackage{colortbl}
\definecolor{xred}{HTML}{BD4242}
\definecolor{xblue}{HTML}{4268BD}
\definecolor{xgreen}{HTML}{52B256}
\definecolor{xpurple}{HTML}{7F52B2}
\definecolor{xorange}{HTML}{FD9337}
\definecolor{xdotted}{HTML}{999999}
\definecolor{xgray}{HTML}{777777}
\definecolor{xcyan}{HTML}{80F5DC}
\definecolor{xpink}{HTML}{F690EA}
\definecolor{xgrayblue}{HTML}{49B095}
\definecolor{xgraycyan}{HTML}{5AA1B9}

% Normal colors
\colorlet{xcol0}{xred}
\colorlet{xcol1}{xblue}
\colorlet{xcol2}{xgreen}
\colorlet{xcol3}{xpurple}
\colorlet{xcol4}{xorange}
\colorlet{xcol5}{xgray}
\colorlet{xcol6}{xcyan}

% Dark colors
\colorlet{xdarkred}{red!85!black}
\colorlet{xdarkblue}{xblue!85!black}
\colorlet{xdarkgreen}{xgreen!85!black}
\colorlet{xdarkpurple}{xpurple!85!black}
\colorlet{xdarkorange}{xorange!85!black}
\colorlet{xdarkcyan}{xcyan!85!black}

% Very dark colors
\colorlet{xverydarkblue}{xblue!50!black}

% Document-specific colors
\colorlet{normaltextcolor}{black}
\colorlet{figtextcolor}{xblue}

%------- %
% XHFILL %
%------- %

\usepackage{xhfill}

%---------------%
% CHAPTER TITLE %
%---------------%

% Used to set chapter title look.
% "explicit" is there so we can
% use the first parameter (#1).
\usepackage[explicit]{titlesec}
\newcommand{\chapternumsize}{\fontsize{70}{70}\selectfont}
\newcommand{\chaptertitlesize}{\fontsize{30}{30}\selectfont}

\titleformat{\chapter}{\sffamily\bfseries}{}{0pt}{
	\centering
	\begin{tikzpicture}[font=\sffamily\bfseries]
		\fill[black!10, rounded corners] (0,0) rectangle (5cm,5cm);
		\node (chapternum) at (2.5cm,4.5cm) {\fontsize{85}{85}\selectfont\thechapter};
		\node[below of=chapternum, yshift=-2.5cm] {\includesvg[scale=1.5]{figures/chapters/\pgfkeysvalueof{/xref dict/\thechapter}}};
		\node[align=center, below of=chapternum, yshift=-6cm] {\fontsize{30}{30}\selectfont\uppercase{#1}};
		\node[align=center, above of=chapternum, yshift=5mm] {\huge CHAPTER};
	\end{tikzpicture}
}

\titleformat{\section}
  {\normalfont\Large\bfseries}% format
  {}% label
  {0pt}% sep
	{\titlerule\newline\thetitle~~~\uppercase{#1}}% before code
  [{\titlerule[0.4pt]}]% after code

%--------%
% TABLES %
%--------%

\usepackage{array}
\newcolumntype{+}{>{\global\let\currentrowstyle\relax}}
\newcolumntype{^}{>{\currentrowstyle}}
\newcommand{\rowstyle}[1]{\gdef\currentrowstyle{#1}%
  #1\ignorespaces
}

%------------%
% REFERENCES %
%------------%

\usepackage{chngcntr}
\counterwithin{table}{chapter}
\counterwithin{figure}{chapter}
\def\chapterautorefname{Chapter}
\def\sectionautorefname{Section}
\makeatletter
\def\tcb@cnt@exampleautorefname{Example}
\def\tcb@cnt@noteautorefname{Note}
\makeatother

%---------%
% HEADERS %
%---------%

\usepackage{fancyhdr}
\pagestyle{fancy}
\fancyhf{}% Clear header/footer
\renewcommand{\chaptermark}[1]{\markboth{\chaptername\ \thechapter:\ #1}{}}
\fancyhead[RO,LE]{\leftmark}% Chapter details in book
\fancyfoot[RO,LE]{\thepage}
\makeatother
\def\MakeNoNewlines#1{\begingroup\def\\{ }#1\endgroup}
\renewcommand{\chaptermark}[1]{\markboth{Chapter \thechapter:\ \MakeNoNewlines{#1}}{}}

%-------%
% BOXES %
%-------%

\usepackage[most]{tcolorbox}

\tikzset{
	second box/.style={
		anchor=east,
		text=white,
		rounded corners,
		fill=#1,
		xshift=-4mm,
	},
}

\tcbset{
	common/.style n args={2}{
		colframe={#1},
		colback={#1!5},
		colbacktitle={#1},
		lower separated=false,
		coltitle=white,
		boxrule=1pt,
		fonttitle=\bfseries,
		enhanced,
		breakable,
		top=8pt,
		before skip=1em,
		after skip=2em,
		attach boxed title to top left={
			yshift=-0.25cm,
			xshift=0.38cm,
		},
		boxed title style={
			boxrule=0pt,
			colframe=white,
			arc=5pt,
			outer arc=4pt,
		},
		separator sign={~~},
		overlay unbroken and last={
			\node[text=white, align=right, rounded corners, fill=#1, xshift=-4mm, minimum height=6mm, anchor=east] at (frame.south east) {#2};
		}
	},
	defstyle/.style={
		common={xpurple}{$\bm{\pi}$},
	},
	theoremstyle/.style={
		common={xgraycyan}{$\multimapdotbothA$},
	},
	lemmastyle/.style={
		common={xgrayblue}{$\multimap$},
	},
	proofstyle/.style={
		common={xgreen}{\textbf{QED}},
	},
	examplestyle/.style={
		common={xblue}{\faStar},
	},
	notestyle/.style={
		common={xred}{\textbf{!}},
	},
	challengestyle/.style={
		common={xorange}{\textbf{?}},
	},
}

\newtcbtheorem[auto counter, number within=chapter]{definition}{Definition}{defstyle}{def}
\newtcbtheorem[auto counter, number within=chapter]{theorem}{Theorem}{theoremstyle}{theorem}
\newtcbtheorem[auto counter, number within=chapter]{lemma}{Lemma}{lemmastyle}{lemma}
\newtcbtheorem[auto counter, number within=chapter]{proof}{Proof}{proofstyle}{proof}
\newtcbtheorem[auto counter, number within=chapter]{example}{Example}{examplestyle}{example}
\newtcbtheorem[auto counter, number within=chapter]{note}{Note}{notestyle}{note}
\newtcbtheorem[auto counter, number within=chapter]{challenge}{Challenge}{challengestyle}{challenge}

%---------%
% FIGURES %
%---------%

\usepackage{float}
\usepackage{caption}
\usepackage[colorlinks=true, linkcolor=xblue]{hyperref}
\usepackage{subcaption}
\usepackage{cleveref}
\captionsetup[subfigure]{subrefformat=simple,labelformat=simple}
\renewcommand\thesubfigure{(\alph{subfigure})}

% Caption label
\DeclareCaptionLabelSeparator{doublespace}{\ \ }
\captionsetup{labelfont={color=xblue,bf}, figurename=Figure, labelsep=doublespace}

% Refs format
\newcommand{\xref}[2][fig]{
	\color{figtextcolor}\textbf{\pgfkeysvalueof{/xref dict/#1}}~\ref{#1:#2} \color{normaltextcolor}
}

%-------------------%
% HIGHLIGHTED WORDS %
%-------------------%

\usepackage{imakeidx}
\usepackage{marginnote}
\makeindex
\renewcommand\emph[1]{\color{xpurple}{\textbf{#1}}\color{normaltextcolor}\index{#1}}%\marginnote[#1]{}

%-------%
% MATHS %
%-------%

\usepackage{amsmath, bm}
\allowdisplaybreaks
\usepackage{nicematrix}
\numberwithin{equation}{section}
\usepackage{siunitx}
\usepackage[thicklines]{cancel}
\renewcommand\CancelColor{\color{xred}}
\newcommand{\Rs}[1]{\mathbb{R}^{#1}}
\newcommand{\stcomp}[1]{{#1}^\complement}
%\newcommand{\true}{\colorbox{xblue!25}{true}}
%\newcommand{\false}{\colorbox{xred!25}{false}}
\newcommand{\true}{\textcolor{xblue}{\textbf{true}}}
\newcommand{\false}{\textcolor{xred}{\textbf{false}}}
\newcommand{\AND}{\textbf{AND}}
\newcommand{\OR}{\textbf{OR}}
\newcommand{\opand}{\wedge}
\newcommand{\opor}{\vee}
\newcommand{\falseprop}[1]{
	\begingroup
	\color{xred}
	\underset{\false}{#1}
	\endgroup
}
\newcommand{\trueprop}[1]{
	\begingroup
	\color{xblue}
	\underset{\true}{#1}
	\endgroup
}
\newcommand{\defeq}{:=}
\newcommand{\eqdef}{=:}
\newcommand{\conj}[1]{\overline{#1}}
\newcommand{\nroot}[2]{
	\sqrt[\leftroot{3}\uproot{3}#1]{#2}
}
\newcommand{\iu}{\mathrm{i}\mkern1mu}
\newcommand{\eu}{\mathrm{e}}
\newcommand{\Z}[1]{\mathbb{Z}_{#1}}
\renewcommand{\gcd}[2]{\text{gcd}\left(#1,#2\right)}
\newcommand{\Ctrig}{\text{c}}
\newcommand{\Strig}{\text{s}}
\newcommand{\Refl}{\text{Ref}}

% Column vectors
\makeatletter
\newcommand{\colvec}[2][c]{%
	\gdef\@VORNE{1}
	\left[\hskip-\arraycolsep%
		\begin{array}{#1}\vekSp@lten{#2}\end{array}%
\hskip-\arraycolsep\right]}

\def\vekSp@lten#1{\xvekSp@lten#1;vekL@stLine;}
\def\vekL@stLine{vekL@stLine}
\def\xvekSp@lten#1;{\def\temp{#1}%
	\ifx\temp\vekL@stLine
	\else
		\ifnum\@VORNE=1\gdef\@VORNE{0}
		\else\@arraycr\fi%
			#1%
			\expandafter\xvekSp@lten
	\fi}
\makeatother

% Colored (common) vectors and angles I use
\newcommand{\vu}{\textcolor{xred}{\vec{u}}}
\newcommand{\vv}{\textcolor{xblue}{\vec{v}}}
\newcommand{\vw}{\textcolor{xpurple}{\vec{w}}}
\newcommand{\ath}{\textcolor{xpurple}{\theta}}

% Norm of a whatever
\newcommand{\gnorm}[1]{\left\|\,  #1 \, \right\|}

% Norm of a vector
\newcommand{\norm}[1]{\left\|\,  \vec{#1} \, \right\|}

% Projection
\newcommandx*{\projection}[4][1=u, 2=v, 3=xred, 4=xblue]{
  \text{proj}_{\textcolor{#4}{\vec{#2}}}\textcolor{#3}{\vec{#1}}
}
\newcommand{\proj}[2]{
	\text{proj}_{#1}{#2}
}

% Standard basis vectors
\newcommand{\eb}[1]{\hat{e}_{#1}}

% Normal vector
\newcommand{\normalVec}[2]{
	\hat{#1}_{\mathbf{#2}}
}

% Color-coded matrix elements
\newcommand{\Matel}[5]{
	#1_{\textcolor{#2}{#4}\textcolor{#3}{#5}}
}
\newcommand{\Ma}[2]{
	\Matel{a}{xred}{xblue}{#1}{#2}
}

% Non-linear transformations
\makeatletter
\def\nltransA{%
	\pgfmathsetmacro{\myX}{\pgf@x+0.0025*\pgf@y^2+0.3*\pgf@x+3}
	\pgfmathsetmacro{\myY}{\pgf@y+0.0050*\pgf@x^2-5}
	\setlength{\pgf@x}{\myX pt}
	\setlength{\pgf@y}{\myY pt}
}
\def\nltransB{%
	\pgfmathsetmacro{\myX}{\pgf@x}
	\pgfmathsetmacro{\myY}{\pgf@y+0.1*abs(\pgf@x)^1.01}
	\setlength{\pgf@x}{\myX pt}
	\setlength{\pgf@y}{\myY pt}
}
\makeatother

% 2D figure for linear transformations
\newcommand{\tapirTrans}[5]{
	\begin{tikzpicture}
		\pgftransformcm{#1}{#2}{#3}{#4}{\pgfpoint{0}{0}}
		\begin{axis}[
			vector plane,
			anchor=center,
			width=#5, height=#5,
			xmin=-1, xmax=1,
			ymin=-1, ymax=1,
			xlabel={},
			ylabel={},
			xtick={-1,-0.9,...,1},
			ytick={-1,-0.9,...,1},
			xticklabels={,,},
			yticklabels={,,},
		]
		\pgfmathsetmacro{\sc}{#5/8cm}
		\node at (0,0) {\includesvg[scale=\sc]{figures/linear_algebra/tapir_transform2}};
	\end{axis}
	\pgftransformreset
	\end{tikzpicture}
}
\newcommand{\tapirTransCM}[7]{
	\pgftransformcm{#1}{#2}{#3}{#4}{\pgfpoint{#5}{#6}}
	\pgfmathsetmacro{\sc}{#7/8cm}
	\node at (0,0) {\includesvg[scale=\sc]{figures/linear_algebra/tapir_transform2}};
	\pgftransformreset
}

%-------------%
% OTHER STUFF %
%-------------%

\usepackage[version=4]{mhchem} % for chemistry
\newcommand{\shrug}[1][]{% shrug emoji
	\begin{tikzpicture}[baseline,x=0.8\ht\strutbox,y=0.8\ht\strutbox,line width=0.125ex,#1]
		\def\arm{(-2.5,0.95) to (-2,0.95) (-1.9,1) to (-1.5,0) (-1.35,0) to (-0.8,0)};
		\draw \arm;
		\draw[xscale=-1] \arm;
		\def\headpart{(0.6,0) arc[start angle=-40, end angle=40,x radius=0.6,y radius=0.8]};
		\draw \headpart;
		\draw[xscale=-1] \headpart;
		\def\eye{(-0.075,0.15) .. controls (0.02,0) .. (0.075,-0.15)};
		\draw[shift={(-0.3,0.8)}] \eye;
		\draw[shift={(0,0.85)}] \eye;
% draw mouth
		\draw (-0.1,0.2) to [out=15,in=-100] (0.4,0.95);
\end{tikzpicture}
}

% List enumeration with chapter number
\usepackage{enumitem}
\setenumerate[1]{label=\thechapter.\arabic*.}
\setenumerate[2]{label*=\arabic*.}

% Lists that have bold/underline labels

\newcommand{\descitemlabel}[1]{%
  \textbullet\ \textbf{#1}:%
}

\newcommand{\listitemlabel}[1]{%
	% Instead of bullet there should be a counter here,
	% but it doesn't work :(
	\textbullet\ \underline{#1}:%
}

\newenvironment{descitemize}
{\begin{description}[itemsep=1.5em, before=\let\makelabel\descitemlabel]}
{\end{description}}

\newenvironment{listitemize}
{\begin{description}[leftmargin=0pt, itemsep=1.25em, before=\let\makelabel\listitemlabel]}
{\end{description}}

% Ornaments
\usepackage{pgfornament}
