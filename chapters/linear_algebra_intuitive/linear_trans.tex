\section{Linear transformations}
\subsection{Definition}
In the previous section we introduced real vectors and their most important properties. In this section we explore a special set of operations that can act on vectors, namely \emph{linear transformations}. As mentioned in \autoref{chapter:intro}, a ``transformations`` is simply another name for a function. Thus in this context, linear transformations are some functions that act on vectors: a linear transformation $T$ takes a vector as an input, and outputs another vector, possibly of a different dimension, i.e.
\begin{equation}
	T:\Rs{n}\to\Rs{m}.
	\label{eq:linear transformation signature}
\end{equation}

What makes linear transformations more ``special'' than other functions is their property of \emph{linearity}, which entails the following two properties:
\begin{listitemize}
	\item[Scalability] for any scalar $\alpha$ and vector $\vec{v}$,
		\[
			T\left( \alpha\vec{v} \right) = \alpha T\left( \vec{v} \right).
		\]
	\item[Additivity] for any two vectors $\vec{u},\vec{v}$
		\[
			T\left( \vec{u}+\vec{v} \right) = T\left( \vec{u} \right) + T\left( \vec{v} \right).
		\]
\end{listitemize}

\begin{example}{A linear transformation}{linear transformation}
	\textbf{Claim}: the following $\Rs{3}\to\mathbb{R}$ transformation is linear:
	\[
		T\left(\colvec{x;y;z}\right) = 2x+3y-z.
	\]

	\textbf{Proof}: We can show this using the properties of linear transformations.
	\begin{listitemize}
	\item[Scalability] given a scalar $\alpha\in\mathbb{R}$,
		\[
			T\left(\colvec{\alpha x;\alpha y; \alpha z}\right)
			= 2(\alpha x)+3(\alpha y)-(\alpha z) = \alpha\left( 2x+3y-z \right)
			= \alpha T\left(\colvec{x;y;z}\right).
		\]
	\item[Additivity] given two vectors $\vec{u}=\colvec{u_{x};u_{y};u_{z}}$ and $\vec{v}=\colvec{v_{x};v_{y};v_{z}}$,
		\begin{align*}
			T\left( \colvec{u_{x};u_{y};u_{z}} + \colvec{v_{x};v_{y};v_{z}} \right) &= T\left( \colvec{u_{x}+v_{x};u_{y}+v_{y};u_{z}+v_{z}} \right)\\
			&= 2\left( u_{x}+v_{x} \right) +3\left( u_{y}+v_{y} \right) - \left( u_{z}+v_{z} \right)\\
			&= T\left( \colvec{u_{x};u_{y};u_{z}} \right) + T\left( \colvec{v_{x};v_{y};v_{z}} \right).
		\end{align*}
	\end{listitemize}
\end{example}

\begin{example}{A non-linear transformation}{non-linear transformation}
	\textbf{Claim}: the following $\Rs{3}\to\mathbb{R}$ transformation is \textbf{not} a linear transformation:
	\[
		T\left(\colvec{x;y;z}\right) = 2x^{2}+3y-z.
	\]

	\textbf{Proof}: this time we only need to show a single case where linearity breaks - let's choose \textit{scalability}. Given the vector $\vec{v}=\colvec{v_{x};v_{y};v_{z}}$, on one hand
	\[
		T\left( \alpha\colvec{u_{x};u_{y};u_{z}} \right) = 2\left(\alpha u_{x}\right)^{2}+3\alpha u_{y}-\alpha u_{z} = 2\alpha^{2}u_{x}^{2}+3\alpha u_{y}-\alpha u_{z}.
	\]
	On the other hand
	\[
		\alpha T\left( \colvec{u_{x};u_{y};u_{z}} \right) = \alpha\left( 2u_{x}^{2}+3u_{y}-u_{z} \right) = 2\alpha u_{x}^{2}+3\alpha u_{y}-\alpha u_{z}.
	\]
	For any $a\notin\left\{ 0,1 \right\}$ we get that $T\left( \alpha\vec{v} \right)\neq\alpha T\left( \vec{v} \right)$. Therefore, $T$ is not linear.
\end{example}

\subsection{Developing intuition}
Before moving on to explore the algebraic properties of linear transformations, we first shift our focus to gain some intuition about them. Much like in the last section, we do this using graphical representations of linear transformations in $\Rs{2}$ and $\Rs{3}$. We start with a single vector under transformation: let $\vec{u}=\colvec{2;4}$ and $T:\Rs{2}\to\Rs{2}$ defined by
\begin{equation}
	T\left( \colvec{x;y} \right) = \colvec{2x;-y}.
\end{equation}
(to the reader: verify that this transformation is indeed linear)

Applying $T$ to $\vec{u}$ yields the vector $\vec{v}=\colvec{4;-4}$ (see \autoref{fig:single vector LT}), i.e. it scales the $x$-component of $\vec{u}$ by $2$ and flippes over its $y$-component.

\begin{figure}
	\centering
	\begin{subfigure}[t]{0.45\textwidth}
		\begin{tikzpicture}[]
			\begin{axis}[
				vector plane,
				width=7cm, height=7cm,
				xmin=-7, xmax=7,
				ymin=-7, ymax=7,
				xtick={-6,-4,...,6},
				ytick={-6,-4,...,6},
			]
			\tikzstyle{every node}=[font=\Large]
			\draw[vector, xred] (0,0) -- (2,4) node[pos=1.15] (u) {$\vec{u}$};
			\draw[vector, xblue] (0,0) -- (4,-4) node[pos=1.1] (v) {$\vec{v}$};
			\draw[vector, dashed, xpurple] (u) to [out=-20, in=70] node[pos=0.35, xshift=9pt] {$T$} (v);
			\end{axis}
		\end{tikzpicture}
		\caption{The vector $\vec{u}=\colvec{2;4}$ is transformed by $T$ yielding the vector $\vec{v}=\colvec{4;-4}$.}
		\label{fig:single vector LT}
	\end{subfigure}
	\hfill
	\begin{subfigure}[t]{0.45\textwidth}
		\centering
		\begin{tikzpicture}[]
			\begin{axis}[
				vector plane,
				width=7cm, height=7cm,
				xmin=-7, xmax=7,
				ymin=-7, ymax=7,
				xtick={-6,-4,...,6},
				ytick={-6,-4,...,6},
			]
			\tikzstyle{every node}=[font=\Large]
			\draw[vector, xred] (0,0) -- (-2,2) node[pos=1.3] (a) {$\vec{a}$};
			\draw[vector, xblue] (0,0) -- (1,-6) node[pos=0.95, right] (b) {$\vec{b}$};
			\draw[vector, xred, dashed] (0,0) -- (-4,-4) node[pos=1.3] (Ta) {$T\left(\vec{a}\right)$};
			\draw[vector, xblue, dashed] (0,0) -- (2,6) node[pos=0.95, right] (Tb) {$T\left(\vec{b}\right)$};
			\end{axis}
		\end{tikzpicture}
		\caption{The vectors $\vec{a}=\colvec{-2;2}$ and $\vec{b}$ are transformed by the same $T$.}
		\label{fig:two vectors LT}
	\end{subfigure}
\end{figure}

If we take other vectors, e.g. $\vec{a}=\colvec{-2;-2}$ and $\vec{b}=\colvec{1;-6}$ we see that $T$ transforms them in the exact same manner: it scales their $x$-components by $2$ and flipps over their $y$-components (\autoref{fig:two vectors LT}). This is a fundamental aspect of linear transformations: they always transform all vectors in the exact same manner. We can use this fact to help visualize transformations, by looking at how they transform the entire space. For example, we can draw all grid lines and observe how they are transformed.

In \autoref{fig:linear transformation grid} a schematic of $\Rs{2}$ is shown before and after the application of a linear transformation $T$, by placing a transformed grid (blue) ontop of an untouched grid (gray). In this view, one can see how each point in space is transformed: assuming for example that each two adjacent grid points are 1 unit apart, the gray point at $(-2,2)$ is transformed to where the blue point is, i.e. $(-1,1)$ when measured using the original axes. 

\colorlet{xbefore}{black!40}
\colorlet{xafter}{xblue}

\begin{figure}
	\centering
	\begin{tikzpicture}
		\begin{axis}[
			vector plane,
			width=8cm, height=8cm,
			anchor=center,
			at={(current page.center)},
			xtick={-4,-2,...,4},
			ytick={-4,-2,...,4},
			xticklabels={,,},
			yticklabels={,,},
			grid style={line width=.1pt, draw=xbefore!35},
			major grid style={line width=.2pt, draw=xbefore!50},
			every tick/.style={xbefore},
			axis line style={xbefore, very thick},
		]
		\addplot[mark=*, mark size=2.5pt] coordinates {(-4,4)};
		% \fill[gray!50] (2,-2) rectangle (4,-4);
		\coordinate (p) at (-2,2);
		\coordinate (xlbl) at (5.5,2.05);
		\coordinate (ylbl) at (2.6,4.95);
		% \addplot[mark=*, xblue] coordinates {(-2,2)};
		\end{axis}
        \pgftransformcm{1}{0.4}{0.5}{0.9}{\pgfpoint{0cm}{0cm}}
		\begin{axis}[
			vector plane,
			width=8cm, height=8cm,
			anchor=center,
			at={(current page.center)},
			xtick={-4,-2,...,4},
			ytick={-4,-2,...,4},
			xticklabels={,,},
			yticklabels={,,},
			xlabel={},
			ylabel={},
			grid style={line width=.1pt, draw=xafter!35},
			major grid style={line width=.2pt, draw=xafter!50},
			every tick/.style={xafter},
			axis line style={xafter, ultra thick},
		]
		% \fill[xblue!50] (2,-2) rectangle (4,-4);
		\end{axis}
        \pgftransformreset
		\fill[xblue] (p) circle (3pt);
		\large
		\node[xblue] at (xlbl) {$x$};
		\node[xblue] at (ylbl) {$y$};
	\end{tikzpicture}
	\caption{$\Rs{2}$ after application of a linear transformation (blue), placed ontop of $\Rs{2}$ before the transformation (gray). Note the black point at the top left at $(-2,2)$ transforming into the blue point at $(-1,1)$.}
	\label{fig:linear transformation grid}
\end{figure}

For comparison, \autoref{fig:nonlinear transformation grid} shows a non linear transformation applied to $\Rs{2}$.

\begin{figure}
	\centering
	\begin{tikzpicture}
		\large
		\draw[black!25, step=0.5] (-3,-3) grid (3,3);
		\draw[stealth-stealth, very thick] (-3,0) -- (3,0) node[pos=1.1] {$x$};
		\draw[stealth-stealth, very thick] (0,-3) -- (0,3) node[pos=1.1] {$y$};
		\fill[black!50] (-3,-3) rectangle (-2.5,-2.5);
		\fill[black!50] (3,3) rectangle (2.5,2.5);
		\begin{scope}
			\pgftransformnonlinear{\nltransA}
			\draw[xblue!25, step=0.5] (-3,-3) grid (3,3);
			\draw[xblue, stealth-stealth, very thick] (-3,0) -- (3,0) node[pos=1.1] {$x$};
			\draw[xblue, stealth-stealth, very thick] (0,-3) -- (0,3) node[pos=1.1] {$y$};
			\fill[xblue!50] (-3,-3) rectangle (-2.5,-2.5);
			\fill[xblue!50] (3,3) rectangle (2.5,2.5);
		\end{scope}
	\end{tikzpicture}
	\caption{A non linear transformation applied to $\Rs{2}$ for comparison.}
	\label{fig:nonlinear transformation grid}
\end{figure}

\autoref{fig:linear transformation grid} shows some important properties of linear transformation (cf. \autoref{fig:nonlinear transformation grid}):
\begin{enumerate}
	\item The origin stays at the same place after application of the transformation, i.e. $T\left(\vec{0}\right) = \vec{0}$.
	\item Parallel lines remain parallel after application of the transformation.
	\item All areas are scaled by the same amount.
\end{enumerate}

It is rather easy to prove the first two properties.
\begin{proof}{Two properties of linear transformations}{}
	\begin{enumerate}
		\item Let $T$ be a transformation that does not perserve the origin, i.e.
	\[
		T\left(\vec{0}\right) = \vec{v} \neq \vec{0}.
	\]
	We can scale $\vec{0}$ by a scalar $\alpha\neq0$, which would yield
	\[
		T\left(\alpha\vec{0}\right) = T\left(\vec{0}\right) = \vec{v}.
	\]
	However, for $T$ to be linear we expect (due to scalability)
	\[
		T\left(\alpha\vec{0}\right) = \alpha\vec{v},
	\]
	but since $\alpha\neq0$ and $\vec{v}\neq\vec{0}$ this does not happen. Therefore, $T$ can not be linear - and in turn linear transformations must preserve the origin.

	\item A line is defined using a point $\vec{a}$, and a direction $\hat{v}$ as the set of all the points $\left\{x=\vec{a}+s\hat{v}, s\in\mathbb{R}\right\}$:

	\begin{center}
		\begin{tikzpicture}
			\begin{axis}[
				vector plane,
				width=8cm, height=8cm,
				xticklabels={,,},
				yticklabels={,,},
			]
			\coordinate (a) at (axis cs:2,2);
			\coordinate (v) at (axis direction cs:1,-0.5);
			\draw[black!50, dashed] ($(a)-6*(v)$) -- ($(a)+3*(v)$);
			\draw[vector, xred] (0,0) -- (a) node[midway, above left] {$\vec{a}$};
			\draw[vector, xblue] (a) -- ($(a)+(v)$) node[midway, below, xshift=-5pt, yshift=1pt] {$\hat{v}$};
			\end{axis}
			\foreach \s in {-6,-3.5,-1,0,1,2}{
				\fill ($(a)+\s*(v)$) circle (0.05) node[above, anchor=west, yshift=2pt] {\tiny$s=\s$};
			}
		\end{tikzpicture}
	\end{center}
	Parallel lines have the same direction $\hat{s}$, i.e. $x_{1}=\vec{a}_{1}+s_{1}\hat{v}$ and $x_{2}=\vec{a}_{2}+s_{2}\hat{v}$ are parallel lines. Applying a linear transformation $T$ to these lines yields (using the two defining properties of linear transformations)
	\begin{align*}
		T\left(x_{1}\right) &= T\left(\vec{a}_{1}\right) + s_{1}T\left(\hat{v}\right),\\
		T\left(x_{2}\right) &= T\left(\vec{a}_{2}\right) + s_{2}T\left(\hat{v}\right).
	\end{align*}
	We can see that the two right-hand side equations represent two new lines with the same direction, i.e. $T\left(\hat{v}\right)$. Therefore parallel lines remain parallel under a linear transformation.
	\end{enumerate}
\end{proof}

We will prove the the third property (all areas are scaled by the same amount) later in the chapter.

All linear transformations in $\Rs{2}$ can be created by composing transformations from a set of linear transformation which we will refer to as the \emph{fundamental linear transformations}\footnote{not an official name.}. To visualize these fundamental transformations we apply them on a figure of a tapir\footnote{trans(formation) tapir, if you will.}:

\begin{figure}[H]
	\centering
	\tapirTrans{1}{0}{0}{1}{8cm}
\end{figure}

\begin{figure}
	\centering
	\begin{subfigure}[c]{0.29\textwidth}
		\centering
		\tapirTrans{1}{0}{0}{1}{5cm}
		\caption{Identity - no change.}
		\label{fig:}
	\end{subfigure}
	\begin{subfigure}[c]{0.37\textwidth}
		\centering
		\tapirTrans{1.4}{0}{0}{1}{5cm}
		\caption{Scaling in the $x$-axis.}
		\label{fig:}
	\end{subfigure}
	\begin{subfigure}[c]{0.29\textwidth}
		\centering
		\tapirTrans{1}{0}{0}{1.3}{5cm}
		\caption{Scaling in the $y$-axis.}
		\label{fig:}
	\end{subfigure}
	\hfill
	\begin{subfigure}[c]{0.3\textwidth}
		\centering
		\tapirTrans{1}{0}{0.25}{1}{5cm}
		\caption{Skew in the $x$-axis.}
		\label{fig:}
	\end{subfigure}
	\begin{subfigure}[c]{0.3\textwidth}
		\centering
		\tapirTrans{1}{0.5}{0}{1}{5cm}
		\caption{Skew in the $y$-axis.}
		\label{fig:}
	\end{subfigure}
	\begin{subfigure}[c]{0.35\textwidth}
		\centering
		\tapirTrans{0.707}{-0.707}{0.707}{0.707}{5cm}
		\caption{Rotation about origin.}
		\label{fig:}
	\end{subfigure}
	\hfill
	\begin{subfigure}[c]{0.3\textwidth}
		\centering
		\tapirTrans{-1}{0}{0}{1}{5cm}
		\caption{Reflection about line.}
		\label{fig:}
	\end{subfigure}
	\begin{subfigure}[c]{0.3\textwidth}
		\centering
		\tapirTrans{-1}{0}{0}{-1}{5cm}
		\caption{Reflection through origin.}
		\label{fig:}
	\end{subfigure}
	\caption{The ``fundamental'' linear transformations, exemplified using a very happy tapir.}
	\label{fig:fundLinearTrans}
\end{figure}

\Blindtext
