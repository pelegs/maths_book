\section{Eigenvectors and eigenvalues}
Some linear transformations have special directions which only scale by the application of the transformation and are not mapped to different directions. Take for example the transformation $T:\Rs{2}\to\Rs{2}$, which scales space by $2$ in the $y$-direction. All vectors pointing in the $y$-direction get scaled by $T$ (namely by a factor of $2$) and still point in the $y$-direction after the application of $T$. All vectors pointing in the $x$-direction do not change at all (i.e. they are "scaled" by a factor of $1$), and of course still point in the $x$-direction after the application of $T$. Any other vector - i.e. those that have both components different than zero - change their direction after the application of $T$ (see \autoref{fig:y_scale_2}).

\begin{figure}
	\centering
	\begin{subfigure}[c]{0.45\textwidth}
		\begin{center}
			\begin{tikzpicture}
				\begin{axis}[
					vector plane,
					width=7.5cm, height=7.5cm,
					xlabel={},
					xticklabels={,},
					yticklabels={,},
					]
					\pgfmathsetmacro{\R}{2}
					\foreach \k in {0,...,11}{
						\pgfmathsetmacro{\th}{30*\k}
						\edef\temp{\noexpand
							\draw[vector, color=xrainbow\k] (0,0) -- ({\R*cos(\th)},{\R*sin(\th)});
						;}
						\temp
					}
				\end{axis}
			\end{tikzpicture}
		\end{center}
		\caption{Some vectors.}
		\label{fig:}
	\end{subfigure}
	\begin{subfigure}[c]{0.45\textwidth}
		\begin{center}
			\begin{tikzpicture}
				\begin{axis}[
					vector plane,
					width=7.5cm, height=7.5cm,
					xticklabels={,},
					yticklabels={,},
					]
					\pgfmathsetmacro{\R}{2}
					\foreach \k in {0,...,11}{
						\pgfmathsetmacro{\th}{30*\k}
						\edef\temp{\noexpand
							\draw[vector, color=xrainbow\k] (0,0) -- ({\R*cos(\th)},{2*\R*sin(\th)});
						;}
						\temp
					}
				\end{axis}
			\end{tikzpicture}
		\end{center}
		\caption{Same vectors after the application of $T$.}
		\label{fig:}
	\end{subfigure}
	\begin{subfigure}[c]{\textwidth}
		\begin{center}
			\begin{tikzpicture}
				\begin{axis}[
					vector plane,
					width=7.5cm, height=7.5cm,
					xticklabels={,},
					yticklabels={,},
					]
					\pgfmathsetmacro{\R}{2}
					\foreach \k in {0,...,11}{
						\pgfmathsetmacro{\th}{30*\k}
						\edef\temp{
							\noexpand\draw[vector, color=xrainbow\k] (0,0) -- ({\R*cos(\th)},{\R*sin(\th)});
							\noexpand\draw[vector, color=xrainbow\k] (0,0) -- ({\R*cos(\th)},{2*\R*sin(\th)});
						;}
						\temp
					}
				\end{axis}
			\end{tikzpicture}
		\end{center}
		\caption{The vectors before and after the application of $T$ layered on top of eachother.}
		\label{fig:}
	\end{subfigure}
	\caption{Some vectors before and after application of the $y$-scaling transformation $T$. Note how only the vectors pointing in the direction of the $x$- and $y$-axes stay in the same direction, while all the other vectors change their directions.}
	\label{fig:y_scale_2}
\end{figure}

We call such vectors the \emph{eigenvectors} of the transformation. The amount by which the are scaled is then their respective \emph{eigenvalues}.

\begin{example}{Eigenvectors and eigenvalues}{}
	Text here
\end{example}

In matrix form, a vector $\vec{v}$ is an eigenvector of a transformation represented by the matrix $A$, if
\begin{equation}
	A\vec{v} = \lambda\vec{v},
	\label{eq:eigenvector_matrix_form}
\end{equation}
where $\lambda\in\mathbb{R},\ \lambda\neq0$. This kind of equation is typically called an \emph{eigenvector equation}. When there are several eigenvectors for a transformation, each with its distinct eigenvalue, we simply add indeces to all relevant parts:
\begin{equation}
	A\vec{v}_{i} = \lambda_{i}\vec{v}_{i},
	\label{eq:eigenvector_matrix_form_indeces}
\end{equation}
where again $\lambda_{i}\in\mathbb{R}$ and $\lambda_{i}\neq0$.

Before continuing to explore some more examples of eigenvectors, there are two properties\footnote{actually one property and one non-property} of eigenvectors that are important to mention. Given a linear transformation $T$,
\begin{itemize}
	\item A scale of any eigenvector $\vec{v}$ of $T$ is also an eigenvector of $T$, with the same eigenvalue.
		\begin{proof}{Eigenvector scale}{}
			Let $T:\Rs{n}\to\Rs{n}$ be a linear transformation represented by the square matrix $A$, with eigenvector $\vec{v}$ and its respective eigenvalue $\lambda$. Then
			\[
				A\vec{v} = \lambda\vec{v}.
			\]
			Replacing $\vec{v}$ with a scale of itself, i.e. $\vec{u}=\alpha\vec{v}$, then applying $A$ to $\vec{u}$ gives us
			\[
				A\vec{u} \overset{(1)}{=} A \left( \alpha\vec{v} \right) \overset{(2)}{=} \alpha A \vec{v} \overset{(3)}{=} \alpha\lambda\vec{v} \overset{(4)}{=} \lambda\alpha\vec{v} \overset{(5)}{=} \lambda\vec{u}.
			\]
			where
			\begin{enumerate}[label=(\arabic*)]
				\item Substitution of $\vec{u}$ by its definition $\vec{u}=\alpha\vec{v}$.
				\item Due to the linearity of $A$ we can bring $\alpha$ out of the product.
				\item Resulting due to $\vec{v}$ being an eigenvector of $A$.
				\item The product of real numbers is commutative.
				\item Substituting back $\alpha\vec{v}=\vec{u}$.
			\end{enumerate}

			Therefore, $\vec{u}$ is also an eigenvector of $A$ (and thus $T$) with the same eigenvalue $\lambda$ as $\vec{v}$.
		\end{proof}

	\item The linear combination of two eigenvectors of $T$ is \textbf{not neccessarily an eigenvector of} $\bm{T}$! For example, consider the above transformation which scales all vectors by $2$ in the $y$-direction: as we saw, any vector in the $x$-direction is an eigenvector of the transformation, and so does any vector in the $y$-direction. Specifically, the vectors $\vec{a}=\colvec{1;0}$ and $\vec{b}=\colvec{0;1}$ are two separate eigenvectors of the transformation (with eigenvalues $1$ and $2$, respectively), however the vector
		\[
			\vec{c} = \vec{a}+\vec{b} = \colvec{1;1}
		\]
		is \textbf{NOT} an eigenvector of the transformation, since
		\[
			\colvec{1;1} \overset{T}{\mapsto} \colvec{1;2} \neq 2\colvec{1;1}.
		\]
\end{itemize}
