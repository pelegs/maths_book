\section{Vectors}
\emph{Vectors} are the fundamental objects of linear algebra: the entire field revolves around manipulation of vectors. In this chapter we deal with the so-called \emph{real vectors}, which can be be defined in a geometric way:

\begin{definition}{Real vectors}{real vectors}
	A \textit{real vector} is an object with a \emph{magnitude} and a \emph{direction}.
\end{definition}

In this chapter we refer to real vectors simply as \textit{vectors}.

\begin{example}{Real vectors}{real vectors}
	The following are all vectors in 2-dimensional space depicted as arrows:
  
	\vspace{1em}
	\centering
	\begin{tikzpicture}
		\draw[vector, xred] (0,0) -- ++(2,3);
		\draw[vector, xblue] (-1,0) -- ++(-1,2);
		\draw[vector, xgreen] (0,-1) -- ++(-3,0);
		\draw[vector, xpurple] (2,0) -- ++(-1,-3);
		\draw[vector, xorange] (-4,2) -- ++(0,-4);
		\draw[vector, black] (-7,1) -- ++(1,-1);
	\end{tikzpicture}
\end{example}

Vectors are usually denoted in one of the following ways:

\begin{descitemize}
	\setlength\itemsep{1em}
	\addtolength{\itemindent}{5mm}
	\item[Arrow above letter] $\vec{u},\ \vec{v},\ \vec{x},\ \vec{a},\ \dots$
	\item[Bold letter] $\bm{u},\ \bm{v},\ \bm{x},\ \bm{a},\ \dots$
	\item[Bar below letter] $\underline{u},\ \underline{v},\ \underline{x},\ \underline{a},\ \dots$
\end{descitemize}

In this book we use the first notation style, i.e. an arrow above the letter. In addition vectors will almost always be denoted using lowercase Lating script.

When discussing vectors in a single context, we always consider them starting at the same point, called the \emph{origin}, and \emph{translating} (moving) vectors around in space does not change their properties: only their magnitudes and directions matter.

\begin{example}{Real vectors}{real vectors}
	The vectors from the previous translated (moved) such that their origins all lie on the same point:
  
	\vspace{1em}
	\centering
	\begin{tikzpicture}
		\draw[vector, xred] (0,0) -- ++(2,3);
		\draw[vector, xblue] (0,0) -- ++(-1,2);
		\draw[vector, xgreen] (0,0) -- ++(-3,0);
		\draw[vector, xpurple] (0,0) -- ++(-1,-3);
		\draw[vector, xorange] (0,0) -- ++(0,-4);
		\draw[vector, black] (0,0) -- ++(1,-1);
		\fill (0,0) circle (0.05);
	\end{tikzpicture}
\end{example}

A vector can be scaled by a real number $\alpha$: when this happens, its magnitude is multiplied by $\alpha$ while its direction stays the same. We call $\alpha$ a \emph{scalar}.
  
\begin{example}{Scaling vectors}{scaling vectors}
	The vector $\vec{v}$ scaled by different scalars $\alpha=2,2.5,-1,-2$:

	\centering
	\begin{tikzpicture}[every node/.style={midway, left, xshift=-2mm}]
		\Large
		\draw[vector, xred] (0,0) -- ++(1.5,1) node {$\vec{v}$};
		\draw[vector, xblue] (2,0) -- ++(3,2) node {$2\cdot \vec{v}$};
		\draw[vector, xpurple] (4.5,0) -- ++(3.75,2.5) node {$2.5\cdot \vec{v}$};
		\draw[stealth-, thick, xgreen!85!black] (7.5,0) -- ++(1.5,1) node {$-1\cdot \vec{v}$};
		\draw[stealth-, thick, black] (9.5,0) -- ++(3,2) node {$-2\cdot \vec{v}$};
	\end{tikzpicture}
\end{example}

\begin{note}{Negative scale}{negative scale}
	As can be seen in the example above, when scaling a vector by a negative amount its direction reverses. However, we consider two opposing direction (i.e. directions that are $\ang{180}$ apart) as being the same direction.
\end{note}
