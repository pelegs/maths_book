\section{Vectors}
\subsection{Basics}
\emph{Vectors} are the fundamental objects of linear algebra: the entire field revolves around manipulation of vectors. In this chapter we deal with the so-called \emph{real vectors}, which can be be defined in a geometric way:

\begin{definition}{Real vectors}{real vectors}
	A \textit{real vector} is an object with a \emph{magnitude} (also called \emph{norm}) and a \emph{direction}.
\end{definition}

In this chapter we refer to real vectors simply as \textit{vectors}.

\begin{example}{Real vectors}{real vectors}
	The following are all vectors in 2-dimensional space depicted as arrows:
  
	\vspace{1em}
	\centering
	\begin{tikzpicture}
		\draw[vector, xred] (0,0) -- ++(2,3);
		\draw[vector, xblue] (-1,0) -- ++(-1,2);
		\draw[vector, xgreen] (0,-1) -- ++(-3,0);
		\draw[vector, xpurple] (2,0) -- ++(-1,-3);
		\draw[vector, xorange] (-4,2) -- ++(0,-4);
		\draw[vector, black] (-7,1) -- ++(1,-1);
	\end{tikzpicture}
\end{example}

Vectors are usually denoted in one of the following ways:

\begin{descitemize}
	\setlength\itemsep{1em}
	\addtolength{\itemindent}{5mm}
	\item[Arrow above letter] $\vec{u},\ \vec{v},\ \vec{x},\ \vec{a},\ \dots$
	\item[Bold letter] $\bm{u},\ \bm{v},\ \bm{x},\ \bm{a},\ \dots$
	\item[Bar below letter] $\underline{u},\ \underline{v},\ \underline{x},\ \underline{a},\ \dots$
\end{descitemize}

In this book we use the first notation style, i.e. an arrow above the letter. In addition vectors will almost always be denoted using lowercase Lating script.

When discussing vectors in a single context, we always consider them starting at the same point, called the \emph{origin}, and \emph{translating} (moving) vectors around in space does not change their properties: only their norms and directions matter.

\begin{example}{Real vectors}{real vectors}
	The vectors from the previous translated (moved) such that their origins all lie on the same point:
  
	\vspace{1em}
	\centering
	\begin{tikzpicture}
		\draw[vector, xred] (0,0) -- ++(2,3);
		\draw[vector, xblue] (0,0) -- ++(-1,2);
		\draw[vector, xgreen] (0,0) -- ++(-3,0);
		\draw[vector, xpurple] (0,0) -- ++(-1,-3);
		\draw[vector, xorange] (0,0) -- ++(0,-4);
		\draw[vector, black] (0,0) -- ++(1,-1);
		\fill (0,0) circle (0.05);
	\end{tikzpicture}
\end{example}

A vector can be scaled by a real number $\alpha$: when this happens, its norm is multiplied by $\alpha$ while its direction stays the same. We call $\alpha$ a \emph{scalar}.

\begin{example}{Scaling vectors}{scaling vectors}
	The following vector $\vec{v}$ scaled by different scalars $\alpha=2,2.5,-1,-2$:

	\centering
	\begin{tikzpicture}[every node/.style={midway, left, xshift=-2mm}]
		\Large
		\draw[vector, xred] (0,0) -- ++(1.5,1) node {$\vec{v}$};
		\draw[vector, xblue] (2,0) -- ++(3,2) node {$2\cdot \vec{v}$};
		\draw[vector, xpurple] (4.5,0) -- ++(3.75,2.5) node {$2.5\cdot \vec{v}$};
		\draw[stealth-, thick, xgreen!85!black] (7.5,0) -- ++(1.5,1) node {$-1\cdot \vec{v}$};
		\draw[stealth-, thick, black] (9.5,0) -- ++(3,2) node {$-2\cdot \vec{v}$};
	\end{tikzpicture}
\end{example}

\begin{note}{Negative scale}{negative scale}
	As can be seen in the example above, when scaling a vector by a negative amount its direction reverses. However, we consider two opposing direction (i.e. directions that are $\ang{180}$ apart) as being the same direction.
\end{note}

In this book we use the following notation for the norm of a vector $\vec{v}$: $\norm{v}$.

A vector $\vec{v}$ with norm $\norm{v}=1$ is called a \emph{unit vector}, and is usually denoted by replacing the arrow symbol by a hat symbol: $\hat{v}$. Any vector (except $\vec{0}$) can be scaled into a unit vector by scaling  the vector by $1$ over its own norm, i.e.
\begin{equation}
	\hat{v} = \frac{1}{\norm{v}}\vec{v}.
	\label{eq:normalized vector}
\end{equation}
The result of normalization is a vector of unit norm which points in the same direction of the original vector.

Two vectors can be added together to yield a third vector: $\vu+\vv=\vw$. To find $\vw$ we use the following procedure (depicted in \autoref{fig:vector addition geometric}):
% The items need to be typeset without the chapter number
\begin{enumerate}
	\item Move (translate) $\vv$ such that its origin lies on the head of $\vu$.
	\item The vector $\vw$ is the vector drawn from the origin of $\vu$ to the head of $\vv$.
\end{enumerate}

\renewcommand\thesubfigure{\arabic{subfigure}}
\begin{figure}[h]
	\centering
	 \begin{subfigure}[t]{0.45\textwidth}
		\centering
		\begin{tikzpicture}
			\coordinate (O) at (0,0);
			\coordinate (u) at (-2,1);
			\coordinate (v) at (1.5,1);
			\coordinate (w) at ($(u)+(v)$);
			\draw[vector, xred] (O) -- (u) node[above left] {$\vec{u}$};
			\draw[vector, xblue] (O) -- (v) node[above right] {$\vec{v}$};
			\draworigin
		\end{tikzpicture}
		\caption{The vectors $\vu$ and $\vv$.}
	\end{subfigure}
	\hfill
	\begin{subfigure}[t]{0.45\textwidth}
		\centering
		\begin{tikzpicture}
			\draw[vector, xred] (O) -- (u) node[above left] {$\vec{u}$};
			\draw[vector, xblue] (u) -- ++(v) node[above right] {$\vec{v}$};
			\draworigin
		\end{tikzpicture}
		\caption{Translating $\vv$ such that its origin lies at the head of $\vu$.}
	\end{subfigure}

	\vspace{3em}
	\begin{subfigure}[t]{0.45\textwidth}
		\centering
		\begin{tikzpicture}
			\draw[vector, xred] (O) -- (u) node[above left] {$\vec{u}$};
			\draw[vector, xblue] (u) -- ++(v) node[above right] {$\vec{v}$};
			\draw[vector, xpurple] (O) -- (w) node[right, yshift=-2mm] {$\vec{w}$};
			\draworigin
		\end{tikzpicture}
		\caption{Drawing the vector $\vw$ from the origin to the head of $\vv$.}
	\end{subfigure}
	\hfill
	\begin{subfigure}[t]{0.45\textwidth}
		\centering
		\begin{tikzpicture}
			\draw[vector, xred] (O) -- (u) node[above left] {$\vec{u}$};
			\draw[vector, xblue] (O) -- (v) node[above right] {$\vec{v}$};
			\draw[vector, xpurple] (O) -- (w) node[above] {$\vec{w}$};
			\draworigin
		\end{tikzpicture}
		\caption{Showing all three vectors.}
	\end{subfigure}
	\caption{Vector addition.}
	\label{fig:vector addition geometric}
\end{figure}

The addition of vectors as depicted here is commutative, i.e. $\vu+\vv = \vv+\vu$. This can be seen by using the \emph{parallogram law of vector addition} as depicted in \autoref{fig:parallelogram}: drawing the two vectors $\vu, \vv$ and their translated copies (each such that its origin lies on the other vector's head) results in a parallelogram.

\begin{figure}[h]
	\centering
	\begin{tikzpicture}
		\draw[vector, xred] (O) -- (u) node[above left] {$\vec{u}$};
		\draw[vector, xblue] (O) -- (v) node[above right] {$\vec{v}$};
		\draw[vector, xred] (v) -- ++(u);
		\draw[vector, xblue] (u) -- ++(v);
		\draw[vector, xpurple] (O) -- (w) node[above] {$\vec{w}$};
		\draworigin
	\end{tikzpicture}
	\caption{The parallogram law of vector addition.}
	\label{fig:parallelogram}
\end{figure}

An important vector is the \emph{zero-vector}, denoted as $\vec{0}$. The zero-vector has a unique property: it is neutral in respect to vector addition, i.e. for any vector $\vec{v}$,
\begin{equation}
	\vec{v} + \vec{0} = \vec{v}.
	\label{eq:zero-vector}
\end{equation}
(we also say that $\vec{0}$ is the \emph{additive identity} in respect to vectors.)

Any vector $\vec{v}$ always has an \emph{opposite} vector, denoted $-\vec{v}$. The addition of a vector and its opposite always result in the zero-vector, i.e.
\begin{equation}
	\vec{v} + \left( -\vec{v} \right) = \vec{0}.
	\label{eq:opposite vector}
\end{equation}

\subsection{Components}
Vectors can be decomposed to their components, the number of which depends on the dimension of space we're using: 2-dimensional vectors can be decomposed into 2 components, 3-dimensional vectors can be decomposed into 3 components, etc. To decompose a vector, say $\vec{v}$, we first choose a coordinate system: the most commonly used system, and the one we will use for most of this chapter, is the Cartesian coordinate system. We place the vector in the coordinate system such that its origin lies at the origin of the system. We then draw a perpendicular line from its head to each of the axes in the system (see \autoref{fig:vector components}), the point of interception on each axis is the component of the vector in that axis (we label these points $v_{x},v_{y},v_{z}$ in the case of 2- or 3-dimensional spaces, and generally $v_{1},v_{2},v_{3},\dots$). The vector can then be written as a column using these components:
\begin{equation}
	\vec{v} = \colvec{v_{1};v_{2};\vdots;v_{n}}.
	\label{eq:column vector}
\end{equation}

\begin{figure}[h]
	\centering
	\begin{tikzpicture}[every node/.style={font=\large}]
		\pgfmathsetmacro{\ux}{2.5}
		\pgfmathsetmacro{\uy}{2}
		\begin{axis}[
			vector plane,
			width=8cm, height=8cm,
			xmin=-1, xmax=3,
			ymin=-1, ymax=3,
			xticklabels={,},
			yticklabels={,},
			extra x ticks={\ux},
			extra x tick labels={$u_{x}$},
			extra x tick style={color=xred},
			extra y ticks={\uy},
			extra y tick labels={$u_{y}$},
			extra y tick style={color=xred},
			]
			\draw[dashed, black!50] (0,\uy) -- (\ux,\uy) -- (\ux,0);
			\draw (\ux,0.1) -- ({\ux+0.1},0.1) -- ({\ux+0.1},0);
			\draw (0.1,\uy) -- (0.1,{\uy+0.1}) -- (0,{\uy+0.1});
			\draw[vector, xred] (0,0) -- (\ux,\uy) node[above] {$\vec{u}=\colvec{u_{x};u_{y}}$};
			\draw[fill] (0,0) circle[radius=2pt];
		\end{axis}
	\end{tikzpicture}
	\caption{Placing a 2-dimensional vector $\vu$ on the 2-dimensional Cartesian coordinate system, showing its $x$- and $y$-components.}
	\label{fig:vector components}
\end{figure}

\begin{note}{Order of components}{}
	The order of the components of a vector is important, and should always be consistent. In the case of $2$- and $3$-dimensional the order is always $v_{x},v_{y},v_{z}$.
\end{note}

\begin{example}{Vector components in two dimensions}{}
	The following five $2$-dimensional vectors are decomposed each into its $x$- and $y$-components:

	\centering
	\begin{tikzpicture}
		\begin{axis}[
			vector plane,
			width=9cm, height=9cm,
			xmin=-3, xmax=3,
			ymin=-3, ymax=3,
			minor tick num=1,
			]
			\veccomp{u}{-2}{1}{xred}
			\veccomp{v}{1.5}{1}{xblue}
			\veccomp{w}{-0.5}{2}{xpurple}
			\veccomp{a}{0.5}{-2}{xgreen}
			\veccomp{b}{-1}{-2}{xorange}
			\draw[fill] (0,0) circle[radius=2pt];
		\end{axis}
	\end{tikzpicture}
\end{example}

\begin{example}{Vector components in three dimensions}{}
	The following $3$-dimensional vector is decomposed into its $x$-, $y$- and $z$-components:
	(THIS NEEDS TO BE IMPROVED AND FINISHED)

	\centering
	\tdplotsetmaincoords{75}{120}
	\begin{tikzpicture}[
			scale=5,
			tdplot_main_coords,
			vector guide/.style={dashed, thick, gray}
		]
		%standard tikz coordinate definition using x, y, z coords
		\coordinate (O) at (0,0,0);

		%tikz-3dplot coordinate definition using x, y, z coords
		\pgfmathsetmacro{\ax}{0.8}
		\pgfmathsetmacro{\ay}{0.8}
		\pgfmathsetmacro{\az}{0.8}

		\coordinate (P) at (\ax,\ay,\az);

		%draw axes
		\draw[vector] (0,0,0) -- (1,0,0) node[anchor=north east]{$x$};
		\draw[vector] (0,0,0) -- (0,1,0) node[anchor=north west]{$y$};
		\draw[vector] (0,0,0) -- (0,0,1) node[anchor=south]{$z$};

		%draw a vector from O to P
		\draw[vector, xred] (O) -- (P);

		%draw guide lines to components
		\draw[vector guide]         (O) -- (\ax,\ay,0);
		\draw[vector guide] (\ax,\ay,0) -- (P);
		\draw[vector guide]         (P) -- (0,0,\az);
		\draw[vector guide] (\ax,\ay,0) -- (0,\ay,0);
		\draw[vector guide] (\ax,\ay,0) -- (0,\ay,0);
		\draw[vector guide] (\ax,\ay,0) -- (\ax,0,0);
		\node[tdplot_main_coords, anchor=east] at (\ax,-0.05,0) {$v_{x}$};
		\node[tdplot_main_coords, anchor=west] at (-0.05,\ay,0) {$v_{y}$};
		\node[tdplot_main_coords, anchor=south] at (0.075,0,\az){$v_{z}$};
	\end{tikzpicture}
\end{example}

The column form of a vector is essentially equivalent to an order list of $n$ real numbers, i.e. $(v_{1},v_{2},\dots,v_{n})$. Why then are we using the column form and not the list form (mostly known as \emph{row vectors})? In fact, we could use either form - and even using both interchangeably - and with only minor adjusments the entire chapter would stay the same as it is now. However, there are some advantages of using only a single form, and consider the other form as a different object altogether. This idea will become clear in future chapters, when discussing \emph{covariant vectors}, \emph{contravarient vectors}, and \emph{tensors}. For now, we stick with the column form of vectors to stay consistent with common notation.

However, the row form of vectors highlights the space in which they exist: $n$-dimensional vectors live in a space we call $\Rs{n}$. Recall from \autoref{chapter:intro} that the set $\Rs{n}$ is a Cartesian product made up of $n$ times the set of real numbers, i.e.
\begin{equation}
	\Rs{n} = \underbrace{\mathbb{R} \times \mathbb{R} \times \cdots \times \mathbb{R}}_{n}.
	\label{eq:Rn}
\end{equation}

Each member of this set is a list of $n$ real numbers, and their order inside the list matters - very similar to vectors, be they in row or column form. For this reason, we refer to $\Rs{n}$ as the space of $n$-dimensional real vectors. As mentioned, in this chapter we use $\Rs{2}$ (the 2-dimensional real space) and $\Rs{3}$ (the 3-dimensional real space) for most ideas and examples.

Looking at vectors in $\Rs{2}$, it is rather straight-forward to calculate their norm: since the origin, the head of the vector and the point $v_{x}$ form a right triangle (see \autoref{fig:norm 2D vector}), we can use the Pythagorean theorem to calculate the norm of the vector, which is equal to the hypotenous of said triangle:
\begin{equation}
	\norm{v} = \sqrt{v_{x}^{2} + v_{y}^{2}}.
	\label{eq:2D vector norm}
\end{equation}

\begin{figure}[h]
	\centering
	\begin{tikzpicture}[every node/.style={font=\large}]
		\pgfmathsetmacro{\vx}{2.5}
		\pgfmathsetmacro{\vy}{2}
		\pgfmathsetmacro{\an}{atan(\vy/\vx)}
		\begin{axis}[
			vector plane,
			width=9cm, height=9cm,
			xmin=-1, xmax=3,
			ymin=-1, ymax=3,
			xticklabels={,},
			yticklabels={,},
			]
			\fill[xgreen, fill opacity=0.07] (0,0) -- (\vx,\vy) -- (\vx,0);
			\draw[dashed, black!50] (\vx,\vy) -- (\vx,0);
			\draw (\vx,0.1) -- ({\vx-0.1},0.1) -- ({\vx-0.1},0);
			\draw[vector, black] (0,0) -- node[midway, above, rotate=\an] {$\norm{v}=\sqrt{v_{x}^{2}+v_{y}^{2}}$} (\vx,\vy) node[above] {$\vec{v}=\colvec{v_{x};v_{y}}$};
			\draw[fill] (0,0) circle[radius=2pt];
			\draw[xgreen, ultra thick, decorate, decoration={brace, amplitude=3pt, raise=3pt, mirror}]
			(0,0) -- (\vx,0) node[midway, below, yshift=-7pt]{$v_{x}$};
			\draw[xgreen, ultra thick, decorate, decoration={brace, amplitude=3pt, raise=3pt, mirror}]
			(\vx,0) -- (\vx,\vy) node[midway, right, xshift=7pt]{$v_{y}$};
		\end{axis}
	\end{tikzpicture}
	\caption{Calculating the norm of a 2-dimensional column vector.}
	\label{fig:norm 2D vector}
\end{figure}

In $\Rs{3}$ the norm of a vector $\vec{v}$ is similarily
\begin{equation}
	\norm{v} = \sqrt{v_{x}^{2} + v_{y}^{2} + v_{z}^{2}}.
	\label{eq:norm 3D vector}
\end{equation}

\begin{challange}{Norm of a 3D vector}{}
	Show why \autoref{eq:norm 3D vector} is valid, by calculating the length $AB$ in the following figure, depicting a box of sides $\textcolor{xblue}{\bm{a}},\textcolor{xgreen}{\bm{b}}$ and $\textcolor{xpurple}{\bm{c}}$:

	\centering
	\begin{tikzpicture}[every path/.style={very thick}, node distance=1mm]
		\pgfmathsetmacro{\xside}{4};
		\pgfmathsetmacro{\yside}{2};
		\pgfmathsetmacro{\zside}{3};

		\coordinate (1) at (0,0,0);
		\coordinate (2) at (\xside,0,0);
		\coordinate (3) at (0,\yside,0);
		\coordinate (4) at (\xside,\yside,0);
		\coordinate (5) at (0,0,\zside);
		\coordinate (6) at (\xside,0,\zside);
		\coordinate (7) at (0,\yside,\zside);
		\coordinate (8) at (\xside,\yside,\zside);

		\draw (1) -- (2);
		\draw (1) -- (3);
		\draw (1) -- (5);
		\draw[densely dotted, red] (5) -- (4);
		\draw (5) -- (7);
		\draw (6) -- (8);
		\draw (2) -- (4);
		\draw (2) -- (6);
		\draw (3) -- (4);
		\draw (3) -- (7);
		\draw (4) -- (8);
		\draw (5) -- (6);
		\draw (7) -- (8);

		\node[left=of 5] {$A$};
		\node[above=of 4] {$B$};

		\draw[xblue, thick, decorate, decoration={brace, amplitude=3pt, raise=3pt, mirror}]
		(5) -- (6) node[midway, below, yshift=-5pt]{$a$};
		\draw[xgreen, thick, decorate, decoration={brace, amplitude=3pt, raise=3pt, mirror}]
		(6) -- (2) node[midway, right, xshift=2pt, yshift=-8pt]{$b$};
		\draw[xpurple, thick, decorate, decoration={brace, amplitude=3pt, raise=3pt, mirror}]
		(2) -- (4) node[midway, right, xshift=5pt]{$c$};
	\end{tikzpicture}
\end{challange}

Generalizing the vector norms in $\Rs{2}$ and $\Rs{3}$ to $\Rs{n}$ yields the following form:
\begin{equation}
	\norm{v} = \sqrt{v_{1}^{2} + v_{2}^{2} + v_{3}^{2} + \dots + v_{n}^{2}} = \sqrt{\sum\limits_{i=1}^{n}v_{i}^{2}}.
	\label{eq:norm nD vector}
\end{equation}

\begin{note}{Other norms}{}
	The norm shown here is called the $2$-norm. There are other possible norm that can be defined, and are used in different situations, such as the $1$-norm (also the called \emph{taxicab norm}), general $p$-norm where $p\geq1$ is a real number, the zero-norm, the max-norm, and many others. However, for the purpose of this chapter we use only the standard $2$-norm, since it is the most useful for describing basic concepts of linear algebra and its uses.
\end{note}

Scaling a vector $\vec{v}=\colvec{v_{1};v_{2};\vdots;v_{n}}$ by a real number $\alpha$ is done by multiplying each of its components by $\alpha$, i.e.
\begin{equation}
	\alpha\vec{v} = \colvec{\alpha v_{1};\alpha v_{2};\vdots;\alpha v_{n}}.
	\label{eq:scaling vectors}
\end{equation}

We can prove \autoref{eq:scaling vectors} by directly calculating the norm of a scaled vector $\vec{w}=\alpha\vec{v}$:
\begin{proof}{Scaling a column vector}{}
	Let $\vec{v}=\colvec{v_{1};v_{2};\vdots;v_{n}}$ and $\vec{w}=\colvec{\alpha v_{1};\alpha v_{2};\vdots;\alpha v_{n}}$, where $\alpha\in\mathbb{R}$. Then $\vec{w}$ has the following norm:
	\begin{align*}
		\norm{w} &= \sqrt{\sum\limits_{i=1}^{n}(\alpha v_{i})^{2}}\\
		&= \sqrt{(\alpha v_{1})^{2} + (\alpha v_{2})^{2} + \dots + (\alpha v_{1})^{2}}\\
		&= \sqrt{\alpha^{2}v_{1}^{2} + \alpha^{2}v_{2}^{2} + \dots + \alpha^{2}v_{n}^{2}}\\
		&= \sqrt{\alpha^{2}\left( v_{1}^{2} + v_{2}^{2} + \dots + v_{n}^{2} \right)}\\
		&= \alpha\sqrt{v_{1}^{2} + v_{2}^{2} + \dots + v_{n}^{2}}\\
		&= \alpha\norm{v}.
	\end{align*}

	This shows that indeed $\vec{w}=\alpha\vec{v}$.
\end{proof}

Another idea we can prove in column form is vector normalization (\autoref{eq:normalized vector}), by showing that dividing each component of a vector by its norm gives a vector of unit norm:
\begin{proof}
	Let $\vec{v}=\colvec{v_{1};v_{2};\vdots;v_{n}}$. Its norm is then $\norm{v}=\sqrt{v_{1}^{2}+v_{2}^{2}+\dots+v_{n}^{2}}$. Scaling $\vec{v}$ by $\frac{1}{\norm{v}}$ yields
	\begin{equation*}
		\hat{v} = \frac{1}{\norm{v}}\colvec{v_{1};v_{2};\vdots;v_{n}} = \frac{1}{\sqrt{v_{1}^{2}+v_{2}^{2}+\dots+v_{n}^{2}}}\colvec{v_{1};v_{2};\vdots;v_{n}}
	\end{equation*}

	The norm of $\hat{v}$ is therefore
	\begin{align*}
		\left\| \hat{v} \right\| &= \sqrt{\frac{v_{1}^{2}}{v_{1}^{2}+v_{2}^{2}+\dots+v_{n}^{2}} + \frac{v_{2}^{2}}{v_{1}^{2}+v_{2}^{2}+\dots+v_{n}^{2}} + \dots + \frac{v_{n}^{2}}{v_{1}^{2}+v_{2}^{2}+\dots+v_{n}^{2}}}\\
		&= \sqrt{\frac{1}{v_{1}^{2}+v_{2}^{2}+\dots+v_{n}^{2}}\left(v_{1}^{2}+v_{2}^{2}+\dots+v_{n}^{2} \right)}\\
		&= \sqrt{1} = 1,
	\end{align*}

	i.e. $\hat{v}$ is indeed a unit vector.
\end{proof}

\begin{example}{Normalizing a vector}{normalizing a vector}
	Let's normalize the vector $\vec{v}=\colvec{0;4;-3}$. Its norm is
	\[
		\norm{v} = \sqrt{0^{2}+4^{2}+(-3)^{2}} = \sqrt{0+16+9} = \sqrt{25} = 5.
	\]
	Therefore $\hat{v}$ (the normalized $\vec{v}$) is
	\[
		\hat{v} = \colvec{0;\frac{4}{5};-\frac{3}{5}}.
	\]

	By calculating the norm of $\hat{v}$ directly, we can see that it is indeed a unit vector:
	\begin{align*}
		\left\|\hat{v}\right\| = \sqrt{0^{2} + \frac{4^{2}}{5^{2}} + \frac{3^{2}}{5^{2}}} = \sqrt{\frac{0^{2}+4^{2}+3^{2}}{5^{2}}} = \sqrt{\frac{16+9}{25}} = \sqrt{\frac{25}{25}} = \sqrt{1} = 1.
	\end{align*}
\end{example}

The addition of two column vectors $\vec{u}=\colvec{u_{1};u_{2};\vdots;u_{n}}$ and $\vec{v}=\colvec{v_{1};v_{2};\vdots;v_{n}}$ is done by adding their respective components together, i.e.
\begin{equation}
	\vec{u} + \vec{v} = \colvec{u_{1}+v_{1};u_{2}+v_{2};\vdots;u_{n}+v_{n}}.
	\label{eq:adding vectors}
\end{equation}

TBW: how this addition is the same as the one shown in \autoref{fig:vector addition geometric}.

\begin{note}{No addition of vectors of different number of components!}{}
	Two vectors can only be added together if they have the same number of components. The addition of vectors with different number of components is undefined.
\end{note}

\subsection{Linear combinations}
As seen above, scaling a vector by a scalar results in a vector that has the same number of dimensions as the original vector. The same is true for adding two vectors: both of them must be of the same dimension, and the result is also a vector of the same dimension. Therefore, any combination of scaling and addition of vectors results in a vector of the same dimension as the original vector(s). This kind of combination is called a \emph{linear combination}.

Let's define linear combinations a little more formaly:

\begin{definition}{Linear combinations}{}
	A linear combination of $n$ vectors $\vec{v}_{1}, \vec{v}_{2}, \dots, \vec{v}_{n}$ of the same dimension, using $n$ scalars $\alpha_{1},\alpha_{2},\dots,\alpha_{n}$, is an expression of the form
	\begin{equation}
		\vec{w} = \alpha_{1}\vec{v}_{1} + \alpha_{2}\vec{v}_{2} + \dots + \alpha_{n}\vec{v}_{n} = \sum\limits_{i=1}^{n}\alpha_{i}\vec{v}_{i}.
		\label{eq:linear combination}
	\end{equation}
\end{definition}

Linear combinations of real vectors have geometric meanings. To understand this, recall that in in two dimensions any two points that are not the same represent a single line (see \autoref{fig:line in 2D}). The same is true in three dimensions, and in addition any three points that are not on the same line represent a single plane (see \autoref{fig:plane in 3D}).

\begin{figure}[h]
	\centering
	\begin{tikzpicture}
		\begin{axis}[
				vector plane,
				width=7cm, height=7cm,
				xticklabels={,},
				yticklabels={,},
				declare function={
					ax=2; ay=1;
					bx=-2; by=-2;
					f(\x)=(by-ay)/(bx-ax)*(\x-ax)+ay;
				},
			]
			\tikzset{
				point/.style={circle, fill=black, inner sep=0pt, minimum size=3pt},
				line/.style={ultra thick, xgreen},
				dline/.style={line, dashed},
			}
			\node[point, label=above:{$A$}] (A) at ({ax}, {ay}) {};
			\node[point, label=below:{$B$}] (B) at ({bx}, {by}) {};
			\draw[line] (A) -- (B);
			\draw[dline] (A) -- (6,{f(6)});
			\draw[dline] (B) -- (-6,{f(-6)});
		\end{axis}
	\end{tikzpicture}
	\caption{The two point $A,B$ define a single line in the 2D plane: $y=\frac{B_{y}-A_{y}}{B_{x}-A_{x}}(x-A_{x})+A_{y}$.}
	\label{fig:line in 2D}
\end{figure}

\begin{figure}[h]
	\centering
	\begin{tikzpicture}
		\begin{axis}[
				domain=-1:1,
				y domain=-1:1,
				grid=major,
				z buffer=sort,
		]
			\addplot3[surf, xblue, opacity=0.5, shader=flat] {y};
			\coordinate (A) at (-0.7,0.4,0.4);
			\coordinate (B) at (0.3,0.4,0.4);
			\coordinate (C) at (0,-0.5,-0.5);
			\path[fill=xgreen, fill opacity=0.75] (A) -- (B) -- (C) -- cycle;
			\draw plot[mark=*, mark size=2] coordinates {(A) (B) (C) (A)};
		\end{axis}
	\end{tikzpicture}
	\caption{The three points $A,B,C$ define a plane in 3D space (blue color). The triangle formed by the points is filled in green to make their poisition more visible.}
	\label{fig:plane in 3D}
\end{figure}

\Blindtext
