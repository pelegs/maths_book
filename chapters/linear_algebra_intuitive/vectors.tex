\section{Vectors}
\subsection{Basics}
\emph{Vectors} are the fundamental objects of linear algebra: the entire field revolves around manipulation of vectors. In this chapter we deal with the so-called \emph{real vectors}, which can be be defined in a geometric way:

\begin{definition}{Real vectors}{real vectors}
	A \textit{real vector} is an object with a \emph{magnitude} (also called \emph{norm}) and a \emph{direction}.
\end{definition}

In this chapter we refer to real vectors simply as \textit{vectors}. Vectors can have $1,2,3,\dots$ number of dimensions. $2$-dimensional vectors can be drawn as arrows on a plane:
\begin{center}
  \begin{tikzpicture}
    \draw[vector, xred] (0,0) -- ++(2,3);
    \draw[vector, xblue] (-1,0) -- ++(-1,2);
    \draw[vector, xgreen] (0,-1) -- ++(-3,0);
    \draw[vector, xpurple] (2,0) -- ++(-1,-3);
    \draw[vector, xorange] (-4,2) -- ++(0,-4);
    \draw[vector, black] (-7,1) -- ++(1,-1);
  \end{tikzpicture}
\end{center}

Similarly, $3$-dimensional vectors can be drawn as arrows in space:
% Work needs to be done on this illustration.
\begin{center}
	\centering
	\tdplotsetmaincoords{70}{45}
	\begin{tikzpicture}[tdplot_main_coords]
    \draw[vector, xorange] (0,0,0) -- (2,-3,-3);
		\draw[black!40, very thick, opacity=1, fill=gray, fill opacity=0.2] (4,4,0) -- (4,-4,0) -- (-4,-4,0) -- (-4,4,0) -- cycle;
    \draw[vector, xred] (0,0,0) -- (3,2,2);
    \draw[vector, xblue] (-2,-1,0) -- (-3,-1,3);
    \draw[vector, xpurple] (-3,3,0) -- ++(3,-4,0.5);
	\end{tikzpicture}
\end{center}

Unfortunately, it is difficult (if not impossible) to draw higher-dimensional vectors. In this chapter we will use mostly $2$-dimensional vectors and sometimes also $3$-dimensional vectors to illustrate the basic concepts of vectors and linear algenra, while most of these concepts apply in higher dimensions as well.

Vectors are usually denoted in one of the following ways:

\begin{descitemize}
	\setlength\itemsep{1em}
	\addtolength{\itemindent}{5mm}
	\item[Arrow above letter] $\vec{u},\ \vec{v},\ \vec{x},\ \vec{a},\ \dots$
	\item[Bold letter] $\bm{u},\ \bm{v},\ \bm{x},\ \bm{a},\ \dots$
	\item[Bar below letter] $\underline{u},\ \underline{v},\ \underline{x},\ \underline{a},\ \dots$
\end{descitemize}

In this book we use the first notation style, i.e. an arrow above the letter. In addition vectors will almost always be denoted using lowercase \textit{Latin} script.

When discussing vectors in a single context, we always consider them starting at the same point, called the \emph{origin}, and \emph{translating} (moving) vectors around in space does not change their properties: only their norms and directions matter. For example, we can draw the $2$-dimensional vectors from before such that their origins all lie on the same point:

\begin{center}
	\begin{tikzpicture}
		\draw[vector, xred] (0,0) -- ++(2,3);
		\draw[vector, xblue] (0,0) -- ++(-1,2);
		\draw[vector, xgreen] (0,0) -- ++(-3,0);
		\draw[vector, xpurple] (0,0) -- ++(-1,-3);
		\draw[vector, xorange] (0,0) -- ++(0,-4);
		\draw[vector, black] (0,0) -- ++(1,-1);
		\fill (0,0) circle (0.05);
	\end{tikzpicture}
\end{center}

Any vector can be scaled by a real number $\alpha$: when this happens, its norm is multiplied by $\alpha$ while its direction stays the same. We call $\alpha$ a \emph{scalar}. For example, the vector $\color{xred}{\vec{v}}$ on the left is scaled here by different scalars $\alpha=2,2.5,-1$ and $-2$:

\begin{center}
	\begin{tikzpicture}[every node/.style={midway, left, xshift=-2mm}]
		\Large
		\draw[vector, xred] (0,0) -- ++(1.5,1) node {$\vec{v}$};
		\draw[vector, xblue] (2,0) -- ++(3,2) node {$2\cdot \vec{v}$};
		\draw[vector, xpurple] (4.5,0) -- ++(3.75,2.5) node {$2.5\cdot \vec{v}$};
		\draw[stealth-, thick, xgreen!85!black] (7.5,0) -- ++(1.5,1) node {$-1\cdot \vec{v}$};
		\draw[stealth-, thick, black] (9.5,0) -- ++(3,2) node {$-2\cdot \vec{v}$};
	\end{tikzpicture}
\end{center}

\begin{note}{Negative scale}{negative scale}
	As can be seen in the example above, when scaling a vector by a negative amount its direction reverses. However, we consider two opposing direction (i.e. directions that are $\ang{180}$ apart) as being the same direction.
\end{note}

In this chapter we use the following notation for the norm of a vector $\vec{v}$: $\norm{v}$.

A vector $\vec{v}$ with norm $\norm{v}=1$ is called a \emph{unit vector}, and is usually denoted by replacing the arrow symbol by a hat symbol: $\hat{v}$. Any vector (except $\vec{0}$) can be scaled into a unit vector by scaling  the vector by $1$ over its own norm, i.e.
\begin{equation}
	\hat{v} = \frac{1}{\norm{v}}\vec{v}.
	\label{eq:normalized vector}
\end{equation}
The result of normalization is a vector of unit norm which points in the same direction of the original vector.

Two vectors can be added together to yield a third vector: $\vu+\vv=\vw$. To find $\vw$ we use the following procedure (depicted in \autoref{fig:vector addition geometric}):
% The items need to be typeset without the chapter number
\begin{enumerate}
	\item Move (translate) $\vv$ such that its origin lies on the head of $\vu$.
	\item The vector $\vw$ is the vector drawn from the origin of $\vu$ to the head of $\vv$.
\end{enumerate}

\renewcommand\thesubfigure{\arabic{subfigure}}
\begin{figure}[h]
	\centering
	 \begin{subfigure}[t]{0.45\textwidth}
		\centering
		\begin{tikzpicture}
			\coordinate (O) at (0,0);
			\coordinate (u) at (-2,1);
			\coordinate (v) at (1.5,1);
			\coordinate (w) at ($(u)+(v)$);
			\draw[vector, xred] (O) -- (u) node[above left] {$\vu$};
			\draw[vector, xblue] (O) -- (v) node[above right] {$\vv$};
			\draworigin
		\end{tikzpicture}
		\caption{The vectors $\vu$ and $\vv$.}
	\end{subfigure}
	\hfill
	\begin{subfigure}[t]{0.45\textwidth}
		\centering
		\begin{tikzpicture}
			\draw[vector, xred] (O) -- (u) node[above left] {$\vec{u}$};
			\draw[vector, xblue] (u) -- ++(v) node[above right] {$\vec{v}$};
			\draworigin
		\end{tikzpicture}
		\caption{Translating $\vv$ such that its origin lies at the head of $\vu$.}
	\end{subfigure}

	\vspace{3em}
	\begin{subfigure}[t]{0.45\textwidth}
		\centering
		\begin{tikzpicture}
			\draw[vector, xred] (O) -- (u) node[above left] {$\vec{u}$};
			\draw[vector, xblue] (u) -- ++(v) node[above right] {$\vec{v}$};
			\draw[vector, xpurple] (O) -- (w) node[right, yshift=-2mm] {$\vec{w}$};
			\draworigin
		\end{tikzpicture}
		\caption{Drawing the vector $\vw$ from the origin to the head of $\vv$.}
	\end{subfigure}
	\hfill
	\begin{subfigure}[t]{0.45\textwidth}
		\centering
		\begin{tikzpicture}
			\draw[vector, xred] (O) -- (u) node[above left] {$\vec{u}$};
			\draw[vector, xblue] (O) -- (v) node[above right] {$\vec{v}$};
			\draw[vector, xpurple] (O) -- (w) node[above] {$\vec{w}$};
			\draworigin
		\end{tikzpicture}
		\caption{All three vectors with the same origin.}
	\end{subfigure}
	\caption{Vector addition.}
	\label{fig:vector addition geometric}
\end{figure}

The addition of vectors as depicted here is commutative, i.e. $\vu+\vv = \vv+\vu$. This can be seen by using the \emph{parallogram law of vector addition} as depicted in \autoref{fig:parallelogram}: drawing the two vectors $\vu, \vv$ and their translated copies (each such that its origin lies on the other vector's head) results in a parallelogram.

\begin{figure}[h]
	\centering
	\begin{tikzpicture}
		\draw[vector, xred] (O) -- (u) node[above left] {$\vec{u}$};
		\draw[vector, xblue] (O) -- (v) node[above right] {$\vec{v}$};
		\draw[vector, xred] (v) -- ++(u);
		\draw[vector, xblue] (u) -- ++(v);
		\draw[vector, xpurple] (O) -- (w) node[above] {$\vec{w}$};
		\draworigin
	\end{tikzpicture}
	\caption{The parallogram law of vector addition.}
	\label{fig:parallelogram}
\end{figure}

An important vector is the \emph{zero-vector}, denoted as $\vec{0}$. The zero-vector has a unique property: it is neutral in respect to vector addition, i.e. for any vector $\vec{v}$,
\begin{equation}
	\vec{v} + \vec{0} = \vec{v}.
	\label{eq:zero-vector}
\end{equation}
(we also say that $\vec{0}$ is the \emph{additive identity} in respect to vectors.)

Any vector $\vec{v}$ always has an \emph{opposite} vector, denoted $-\vec{v}$. The addition of a vector and its opposite always result in the zero-vector, i.e.
\begin{equation}
	\vec{v} + \left( -\vec{v} \right) = \vec{0}.
	\label{eq:opposite vector}
\end{equation}

\subsection{Column representation}
In order to be able to use vectors for actual calculations we must somehow quantify their properties. When quantifying geometric shapes we often first define some unit of measurement, for example a centimeter (\si{cm}). We then use this unit to measure the length of different objects.

While this simple approach works well for describing lengths, in the case of vectors we also want to quantify directions - which becomes a bit complicated in higher dimensions if we use angles. Instead, we use more than one unit of measurement; in fact, we use a vector as a unit of measurement for each dimension (and call these vectors \emph{basis vectors}). For example, we can choose the following two $2$-dimensional vectors $\xr{\vec{e}_{1}}$ and $\xb{\vec{e}_{2}}$ to be used as basis vectors:

\begin{center}
  \begin{tikzpicture}
    \coordinate (O) at (0,0);
    \coordinate (e1) at (1,0.5);
    \coordinate (e2) at (-0.5,0.5);
    \draw[vector, xred] (O) -- ++(e1) node[midway, above] {$\vec{e}_{1}$};
    \draw[vector, xblue] (3,0) -- ++(e2) node[midway, above right] {$\vec{e}_{2}$};
  \end{tikzpicture}
\end{center}

We can then use these basis vectors to measure other $2$-dimensional vectors, for example the following vector $\xp{\vec{v}}$:

\begin{center}
  \begin{tikzpicture}
    \coordinate (v) at ($3*(e1)+2*(e2)$);
    \draw[vector, xpurple] (O) -- ++(v) node[midway, left] {$\vec{v}$};
    \draw[vector, xred, dashed] (O) -- ++(e1) node[midway, below] {$\vec{e}_{1}$};
    \draw[vector, xred, dashed] (e1) -- ++(e1) node[midway, below] {$\vec{e}_{1}$};
    \draw[vector, xred, dashed] ($2*(e1)$) -- ++(e1) node[midway, below] {$\vec{e}_{1}$};
    \draw[vector, xblue, dashed] ($3*(e1)$) -- ++(e2) node[midway, right] {$\vec{e}_{2}$};
    \draw[vector, xblue, dashed] ($3*(e1)+(e2)$) -- ++(e2) node[midway, right] {$\vec{e}_{2}$};
    \fill (O) circle (0.05);
  \end{tikzpicture}
\end{center}

We see that using these basis vectors, $\xp{\vec{v}}=3\xr{\vec{e}_{1}}+2\xb{\vec{e}_{2}}$. This means that we need to add three times $\xr{\vec{e}_{1}}$ and two times $\xb{\vec{e}_{2}}$ to construct $\xp{\vec{v}}$. We denote this fact by writing $\xp{\vec{v}}$ as a column of two numbers:

\[
  \xp{\vec{v}}=\colvec{\tikzmark{v1}{\xr{3}};\tikzmark{v2}{\xb{2}}}
\]

\begin{tikzpicture}[overlay, remember picture]
  \coordinate(v1b) at (pic cs:v1);
  \coordinate(v2b) at (pic cs:v2);
  \node[right of=v1b, anchor=west, yshift=5pt] (v1txt) {How much of $\xr{\vec{e}_{1}}$ is in $\xp{\vec{v}}$};
  \node[right of=v2b, anchor=west, yshift=-5pt] (v2txt) {How much of $\xb{\vec{e}_{2}}$ is in $\xp{\vec{v}}$};
  \draw[->, thick, xred] (v1txt) -- ($(v1b)+(10pt,0)$);
  \draw[->, thick, xblue] (v2txt) -- ($(v2b)+(10pt,0)$);
\end{tikzpicture}

\begin{example}{Vectors from basis vectors}{}
  Two more vectors represented as sums of the basis vectors $\xr{\vec{e}_{1}}$ and $\xb{\vec{e}_{2}}$:

  \begin{center}
    \begin{tikzpicture}
      \coordinate (u) at ($2*(e1)-2*(e2)$);
      \draw[vector, xorange] (O) -- (u) node[midway, below] {$\xo{\vec{u}}$};
      \draw[vector, xred, dashed] (O) -- ++(e1) node[midway, above] {$\vec{e}_{1}$};
      \draw[vector, xred, dashed] (e1) -- ++(e1) node[midway, above] {$\vec{e}_{1}$};
      \draw[vector, xblue, dashed] ($2*(e1)$) -- ++($-1*(e2)$) node[midway, right] {$-\vec{e}_{2}$};
      \draw[vector, xblue, dashed] ($2*(e1)-(e2)$) -- ++($-1*(e2)$) node[midway, right] {$-\vec{e}_{2}$};
      \fill (O) circle (0.05);
      \node[right of=O, xshift=3cm, anchor=west] (u_math) {$\xo{\vec{u}}=2\xr{\vec{e}_{1}}-2\xb{\vec{e}_{2}}\quad\longrightarrow\quad\xo{\vec{u}}=\colvec{\xr{1};\xb{-2}}$};
    \end{tikzpicture}

    \vspace{1.5em}
    \begin{tikzpicture}
      \coordinate (w) at ($-3*(e1)+1.5*(e2)$);
      \draw[vector, xgreen] (O) -- (w) node[midway, above] {$\xg{\vec{w}}$};
      \draw[vector, xred, dashed] (O) -- ++($-1*(e1)$) node[midway, below] {$-\vec{e}_{1}$};
      \draw[vector, xred, dashed] ($-1*(e1)$) -- ++($-1*(e1)$) node[midway, below] {$-\vec{e}_{1}$};
      \draw[vector, xred, dashed] ($-2*(e1)$) -- ++($-1*(e1)$) node[midway, below] {$-\vec{e}_{1}$};
      \draw[vector, xblue, dashed] ($-3*(e1)$) -- ++(e2) node[midway, left] {$\vec{e}_{2}$};
      \draw[vector, xblue, dashed] ($-3*(e1)+(e2)$) -- ++($0.5*(e2)$) node[midway, left] {$\frac{1}{2}\vec{e}_{2}$};
      \fill (O) circle (0.05);
      \node[right of=O, anchor=west] (u_math2) {$\xg{\vec{w}}=-3\xr{\vec{e}_{1}}+1.5\xb{\vec{e}_{2}}\quad\longrightarrow\quad\xg{\vec{w}}=\colvec{\xr{-3};\xb{1.5}}$};
    \end{tikzpicture}
  \end{center}
\end{example}

This notation is frequently refered to as a \emph{column vector}, and the numbers are called \emph{coordinates} or \emph{components}. The components themselves are usually denoted using the same symbol used for the vector (without the arrow sign) and an index: for example, the two components of the vector $\xp{\vec{v}}=\colvec{\xr{3};\xb{2}}$ are $\xr{v^{1}}=\xr{3}$ and $\xb{v^{2}}=\xb{2}$.

\begin{note}{Index notation}{}
  In this book we use upper-index notation to denote components of column vectors. Do not mistake these for powers! The reason for this choice (as opposed to lowe-index notation, i.e. $v_{1},\ v_{2},$ etc.) is to stay consistent with later parts of the book, where covectors and in general tensors are presented.
\end{note}

Vectors have different components when we use different basis vectors to describe them. For example, we can use the following two vectors $\xdg{\vec{d}_{1}}$ and $\xo{\vec{d}_{2}}$ as basis vectors:

\begin{center}
  \begin{tikzpicture}
    \coordinate (O) at (0,0);
    \coordinate (e1b) at (1,0);
    \coordinate (e2b) at (0,1);
    \draw[vector, xdarkgreen] (O) -- ++(e1b) node[midway, above] {$\vec{d}_{1}$};
    \draw[vector, xorange] (3,0) -- ++(e2b) node[midway, right] {$\vec{d}_{2}$};
  \end{tikzpicture}
\end{center}

Using these new basis vectors, the vector $\xp{\vec{v}}$ from earlier has the following column representation:

\vspace{-1.2em}
\begin{center}
  \begin{tikzpicture}
    \coordinate (O) at (0,0);
    \coordinate (v) at ($2*(e1b)+2.5*(e2b)$);
    \draw[vector, xdarkgreen] (O) -- ++(e1b) node[midway, below] {$\vec{d}_{1}$};
    \draw[vector, xdarkgreen] (e1b) -- ++(e1b) node[midway, below] {$\vec{d}_{1}$};
    \draw[vector, xorange] ($2*(e1b)$) -- ++(e2b) node[midway, right] {$\vec{d}_{2}$};
    \draw[vector, xorange] ($2*(e1b)+(e2b)$) -- ++(e2b) node[midway, right] {$\vec{d}_{2}$};
    \draw[vector, xorange] ($2*(e1b)+2*(e2b)$) -- ++($0.5*(e2b)$) node[midway, right] {$\frac{1}{2}\vec{d}_{2}$};
    \draw[vector, xpurple] (O) -- (v) node[above] {$\vec{v}$};
    \draw[->, thick] (3,1.3) -- ++(1,0);
    \node at (5,1.3) {$\xp{\vec{v}}=\colvec{\xdg{2};\xo{2.5}}$};
  \end{tikzpicture}
\end{center}
and its components are $\xdg{v^{1}}=\xdg{2}$ and $\xo{v^{2}}=\xo{2.5}$.

\begin{note}{Coordinates of the basis vectors}{}
  Since the column representation of vectors is different when different basis vectors are used, one must always be sure with which basis vectors they are working - otherwise mistakes are bound to happen and calculations might make no sense.
\end{note}

\begin{note}{Coordinates of the basis vectors}{}
  No matter which two vectors $\vec{u}, \vec{v}$ are used as basis vectors for $\Rs[2]$, their column vector representation would always be $\vec{u}=\colvec{1;0}$ and $\vec{v}=\colvec{0;1}$. This is because
  \begin{align*}
    \vec{u} &= 1\cdot\vec{u}+ 0\cdot\vec{v},\\
    \vec{v} &= 0\cdot\vec{u} + 1\cdot\vec{v}.
  \end{align*}
\end{note}

Since the column representation of vectors in $2$-dimensions has two entries, it is common to refer to the set of all $2$-dimensional vectors as $\Rs\times\Rs$ or simply $\Rs[2]$.

Any two vectors $\xr{\vec{e}_{1}}$ and $\xb{\vec{e}_{2}}$ can be used as basis vectors in $\Rs[2]$, as long as they are not pointing in the same direction, i.e. as long as they are not scales of each other:
\begin{equation}
  \xb{\vec{e}_{2}} \neq \alpha\xr{\vec{e}_{1}},
  \label{eq:basis_vectors_not_scales}
\end{equation}
where $\alpha$ is a real number. This also means that the zero vector can't be used as a basis vector. We will elaborate on that point later on.

\subsection{Cartesian coordiantes and the standard basis vectors}

The space $\Rs[2]$ can be placed inside a Cartesian coordinate system, such that the origin of all vectors is at the point $\bm{O}=(0,0)$. It is then very common to use the following two vectors as basis vectors: $\xr{\hat{x}}$, a vector of norm $1$ pointing in the direction of the $x$-axis, and $\xb{\hat{y}}$, a vector of norm $1$ pointing in the direction of the $y$-axis:

\colorlet{xbcol}{black!30}
\begin{center}
  \begin{tikzpicture}[]
    \begin{axis}[
      vector plane,
      width=5cm, height=5cm,
      xmin=-2, xmax=2,
      ymin=-2, ymax=2,
      xtick={-2,-1,1,2},
      ytick={-2,-1,1,2},
      axis line style={xbcol},
      tick style={xbcol},
      ticklabel style={xbcol},
      xlabel style={xbcol},
      ylabel style={xbcol},
      major grid style={black!2},
    ]
    \draw[vector, xred] (0,0) -- (1,0) node[midway, below] {$\hat{x}$};
    \draw[vector, xblue] (0,0) -- (0,1) node[midway, left] {$\hat{y}$};
    \fill (0,0) circle (0.05);
    \end{axis}
  \end{tikzpicture}
\end{center}

Using these basis vectors, any vector $\vec{v}$ has components that are equal to the coordinates of the point at it head (see \autoref{fig:vectors_std_basis_coords}).

\begin{figure}
  \centering
  \begin{tikzpicture}[]
    \begin{axis}[
      vector plane,
      width=10cm, height=10cm,
      xmin=-3, xmax=3,
      ymin=-3, ymax=3,
      xticklabels={},
      yticklabels={},
      axis line style={black!30},
      tick style={black!30},
      grid=both,
      major grid style={black!10},
      minor grid style={black!5},
      minor tick num=4,
    ]
    \draw[vector, xred] (0,0) -- (1,0) node[midway, below] (xh) {$\hat{x}$} node[anchor=north west] {$(1,0)$};
    \draw[vector, xblue] (0,0) -- (0,1) node[midway, left] (yh) {$\hat{y}$} node[anchor=south east] {$(0,1)$};
    \draw[vector, xpurple] (0,0) -- (2,1) node[midway, above] {$\vec{v}$} node[anchor=west] {$(2,1)$};
    \draw[vector, xdarkgreen] (0,0) -- (-2.6,1.2) node[midway, above] {$\vec{u}$} node[anchor=south, xshift=5] {$(-2.6,1.2)$};
    \draw[vector, xorange] (0,0) -- (-2,-2) node[midway, above] {$\vec{w}$} node[anchor=north east] {$(-2,-2)$};
    \fill (0,0) circle (0.05);
    \end{axis}
  \end{tikzpicture}
  \caption{The standard basis vectors $\xr{\hat{x}}=\colvec{\xr{1};\xb{0}}$ and $\xb{\hat{y}}=\colvec{\xr{0};\xb{1}}$ on a $2$-dimensional Cartesian coordinate system, together with the vectors $\xp{\vec{v}}=\colvec{\xr{2};\xb{1}}$, $\xdg{\vec{u}}=\colvec{\xr{-2.6};\xb{1.2}}$ and $\xo{\vec{w}}=\colvec{\xr{-2};\xb{-2}}$.}
  \label{fig:vectors_std_basis_coords}
\end{figure}
