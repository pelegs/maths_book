\section{Systems of linear equations}
Everything we learned so far about vectors and matrices can be used to solve and characterise a family of equations known as \emph{linear equations}. You're probably already very familiar with linear equations: they are equations in which the \emph{variables} appear directly, without any power or other functions acting on them. For example, the simple equation
\begin{equation}
	y = ax+b,
	\label{eq:linear_equation_2_vars}
\end{equation}
where $x,y$ are both variables and $a,b$ are both constant real numbers is a linear equation. \autoref{eq:linear_equation_2_vars} can be re-written as
\begin{equation}
	ax - y + b = 0,
	\label{eq:linear_equation_2_vars_rewritten}
\end{equation}
where now $a$ is the \emph{coefficient} of the variable $x$, while the variable $y$ has the coefficient $-1$ and $b$ is a so-called \emph{free coefficient}. In general, a linear equation of two variables has the form
\begin{equation}
	a_{0} + a_{x}x + a_{y}y = 0,
	\label{eq:general_linear_equation_2_vars}
\end{equation}
i.e. we changed the name of $a$ to $a_{x}$ and $b$ to $a_{0}$, and gave $y$ the coefficient $a_{y}$. We can also rename $x$ and $y$ to $x_{1}$ and $x_{2}$, respectively, and name their coefficients accordingly:
\begin{equation}
	a_{0} + a_{1}x_{1} + a_{2}x_{2} = 0.
	\label{eq:general_linear_equation_2_vars_renamed}
\end{equation}

The form shown in \autoref{eq:general_linear_equation_2_vars_renamed} can be easily expanded into $n$ variables:
\begin{equation}
	a_{0} + a_{1}x_{1} + a_{2}x_{2} + a_{3}x_{3} + \cdots + a_{n-1}x_{n-1} + a_{n}x_{n} = 0,
	\label{eq:general_linear_equation_n_vars}
\end{equation}
where $x_{1},x_{2},\dots,x_{n}$ are the variables of the equation, and $a_{0},a_{1},\dots,a_{n}$ are its coefficients. We say that $n$ is the \emph{order} (also: \emph{degree}) of the equation.

\begin{note}{Number set used for linear equations}{}
	As with other topics, in the context of this section both the variables and coefficients are all \textbf{real numbers}, however almost anything we discuss here can genrally be applied to complex numbers or other structures.
\end{note}

\begin{example}{Linear equations}{}
	The following is a linear equation of order $3$, using the variables $x,y,z$:
	\[
		3x+2y-z+4 = 0.
	\]
	The coefficients of the equation are
	\begin{align*}
		a_{0}&=4,\\
		a_{x}&=a_{1}=3,\\
		a_{y}&=a_{2}=2,\\
		a_{z}&=a_{3}=-1.
	\end{align*}

	Another linear equation of the same three variables is
	\[
		5x-2y+1 = 0.
	\]
	In this case the coefficient $a_{z}=a_{3}=0$. Depending on the context, this equation can be considered as either an equation of order $3$ or an equation of order $2$.
\end{example}

A linear equation of order $2$ represents a line in $\Rs{2}$ which doesn't necesserally go through the origin (and thus isn't necesserally a subspace of $\Rs{2}$). For a line to go through the origin, the free coefficient $a_{0}$ must equal zero (see \autoref{fig:linear_equations_2_vars}).

\begin{figure}
	\centering
	\begin{tikzpicture}
		\begin{axis}[
			vector plane,
			width=11cm, height=11cm,
			xmin=-1, xmax=6,
			ymin=-1, ymax=6,
			domain={-1:6},
		]
		\addplot[function, xred] {2*\x} node[pos=0.27, below, font=\Large, rotate=63.435] {$-2x+y=0$};
		\addplot[function, xblue] {-0.5*\x+5} node[pos=0.7, above, font=\Large, rotate=-30] {$\frac{1}{2}x+y+5=0$};
		\end{axis}
	\end{tikzpicture}
	\caption{Two linear equations represented as lines in $\Rs{2}$. Note how in the red equation the free coefficient is zero, and so the line goes through the origin.}
	\label{fig:linear_equations_2_vars}
\end{figure}
