\section{Some real life uses of linear algebra}
\tbw{you guessed it - the entire section.}
\subsection{Formulating plant nutrient solutions}
In order to grow and thrive, plants require \emph{nutrients}. These nutrients are chemical elements which the plants use in different processes to create new roots, stems, leaves, fruits, etc. and to sustain those that already exist. Generaly speaking, there are $17$ such elements, and they can be roughly divided in to three categories:
\begin{descitemize}
	\item[Nutrients from air] carbon (\ce{C}), oxygen (\ce{O}) and hydrogen (\ce{H}). Plants get these from water (\ce{H2O}), carbon dioxide (\ce{CO2}) and atmospheric oxygen (\ce{O2}), and use them together with light to build carbohydrates (such as glucose, fructose and cellulose), in a process known as photosynthesis.
	\item[Macronutrients from ground] nitrogen (\ce{N}), phosphorus (\ce{P}), potassium (\ce{K}), calcium (\ce{Ca}), magnesium (\ce{Mg}) and sulfur (\ce{S}). These nutrients are absorbed mostly through the roots of the plant, and are needed in relatively large quantities.
	\item[Micronutrients from groud] chlorine (\ce{Cl}), iron (\ce{Fe}), manganese (\ce{Mn}), zinc (\ce{Zn}), copper (\ce{Cu}), molybdenum (\ce{Mo}), boron (\ce{B}) and nickel (\ce{Ni}). These are also absorbed mainly via the roots, and are needed in relatively smaller quantities than the macronutrients.
\end{descitemize}

\autoref{tab:plant_nutrients} describes what each of the above elements is used for by plants.

\tbw{add data and format the table}
\begin{table}[htpb]
	\centering
	\caption{The $17$ essential elements for plants and their uses.}
	\label{tab:plant_nutrients}
	\begin{longtable}{p{3mm}lcll}
		\toprule
		& Element & symbol & [\si{ppm}] & some uses \\
		\midrule
		\multirow{3}{=}{\rotatebox{-90}{Air}}
		& Oxygen & \ce{O} & & practically all organic mulecules \\
		& Carbon & \ce{C} & & carbohydrates and protein synthesis \\
		& Hydrogen & \ce{H} & & practically all organic mulecules \\
		\midrule
		\multirow{6}{=}{\rotatebox{-90}{Macronutrients}}
		& Nitrogen & \ce{N} & $15,000$ & proteins \\
		& Phosphorus & \ce{P} & $2,000$ & phospholipids, ATP \\
		& Potassium & \ce{K} & $10,000$ & ion balance \\
		& Calcium & \ce{Ca} & $5,000$ & ion balance \\
		& Magnesium & \ce{Mg} & $2,000$ & chlorophyl \\
		& Sulfur & \ce{S} & $1,000$& proteins, vitamins, chloroplast \\
		\midrule
		\multirow{8}{=}{\rotatebox{-90}{Micronutrients}}
		& Chlorine & \ce{Cl} & $100$ & ion balance \\
		& Iron & \ce{Fe} & $100$ & enzyme cofactors, photosynthesis \\
		& Manganese & \ce{Mn} & $50$ & chloroplast \\
		& Zinc & \ce{Zn} & $20$ & DNA transcriptase \\
		& Copper & \ce{Cu} & $6$ & photosynthesis \\
		& Molybdenum & \ce{Mo} & $0.1$ & enzyme cofactors \\
		& Boron & \ce{B} & $20$ & metabolism \\
		& Nickel & \ce{Ni} & $0.1$ & nitrogen metabolism \\
		\bottomrule
	\end{longtable}
\end{table}
% Epstein. 1965. "Mineral metabolism" pp. 438-466. in: Plant Biochemistry (J.Bonner and J.E. Varner, eds.) Academic Press, London.

\tbw{rest of subsection}

\subsection{Primary components analsis}

\subsection{Hunter-pray population growth}
