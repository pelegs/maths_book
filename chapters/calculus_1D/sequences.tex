\section{Sequences and series}
\subsection{Basics}
A \emph{sequence} is an indexed collection of \emph{elements}. By \textit{indexed} we mean that the order of the elements in a sequence matters (unlike with sets): changing the order of any element changes the sequence as a whole. The following are some examples of sequences composed of real numbers:
\begin{itemize}
	\item $1,-3,0,-7,2,1.5,4,0,1,-0.35,\sqrt{2}$.
	\item $0,1,2,1,1,-1,0$.
	\item $1, \frac{1}{2}, \frac{1}{3}, \frac{1}{4}, \frac{1}{5}, \frac{1}{6}, \dots$
	\item $0,1,0,1,0,1,0,1,0,1,0,1,0,1,\dots$
\end{itemize}
The examples above present two more properties of sequences:
\begin{itemize}
	\item Elements may repeat (unlike in the case of sets), and
	\item sequences can be either \emph{finite} (as in the first two examples), or \emph{infinite} (as in the latter two examples).
\end{itemize}

The number of elements in a sequence is called its \emph{length}. In the case of infinite sequences we say that their length equals $\infty$ (inifinity). The elements of a sequence $a$ are usually indexed using a subscript, such that $a_{1}$ is the first element in the sequence, $a_{2}$ is the second element in the sequence, etc. - and generally $a_{i}$ is the $i$-th element in the sequence, where $i\in\mathbb{N}$.

We can therefore define a sequence somewhat more formally as a function from a subset of the natural numbers to the real numbers:
\begin{equation}
	a:N\to\Rs,
\end{equation}
where $N\subseteq\mathbb{N}$.
\begin{example}{Sequences as functions}{}
	The following $9$-element sequence $a$
	\begin{center}
		\begin{tikzpicture}
			\matrix[matrix of math nodes, nodes={anchor=center, minimum width=13mm}, row 1/.style={nodes={font=\large}}] (seq)
			{3, & 4, & \frac{1}{2}, & 0, & 2, & 6, & -\frac{2}{3}, & 0, & -1.\\};
			\foreach \k in {1,...,9}{
				\node[xred, below of=seq-1-\k] (a-\k) {$a(\k)$};
				\draw[xred, -stealth] (a-\k) -- (seq-1-\k);
			}
		\end{tikzpicture}
	\end{center}
	
	can be viewed as a function
	\[
		a:\left\{1,2,\dots,9\right\}\to\Rs,
	\]
	or more precisely as a function
	\[
		a:\left\{1,2,\dots,9\right\}\to\left\{-1,-\frac{2}{3},0,\frac{1}{2},2,3,4,6\right\}.
	\]

	\vspace{2em}
	The follow infinite sequence $b$
	\begin{center}
		\begin{tikzpicture}
			\matrix[matrix of math nodes, nodes={anchor=center, minimum width=13mm}, row 1/.style={nodes={font=\large}}] (seq)
			{1, & \frac{1}{2}, & \frac{1}{3}, & \frac{1}{4}, &  \frac{1}{5}, & \frac{1}{6}, & \frac{1}{7}, & \dots \\};
			\foreach \k in {1,...,7}{
				\node[xpurple, below of=seq-1-\k] (a-\k) {$b(\k)$};
				\draw[xpurple, -stealth] (a-\k) -- (seq-1-\k);
			}
		\end{tikzpicture}
	\end{center}
	can be viewed as a function
	\[
		b:\mathbb{N}\to(0,1].
	\]
\end{example}

Since sequences can be viewed as functions, they can be defined using formulas: for example, the sequence
\[
	1, \frac{1}{2}, \frac{1}{3}, \frac{1}{4}, \dots
\]
can be defined using the simple formula
\[
	a_{n}=\frac{1}{n}.
\]

\begin{example}{Some sequences defined using formulas}{}
	\renewcommand*{\arraystretch}{3}
	\centering
	\begin{tabular}{rll}
		$(-1)^{n}$ & $\Rightarrow$ & $-1,1,-1,1,-1,1,-1,1,-1,\dots$\\
		$3n+4$ & $\Rightarrow$ & $7,10,13,16,19,22,\dots$\\
		$(n+1)^{2}$ & $\Rightarrow$ & $4,9,16,25,36,49,\dots$\\
		$\begin{cases}
			\colorbox{xred!20}{2n+1} & \text{if}\ n\text{ is odd},\\
			\colorbox{xblue!20}{n-1}  & \text{if}\ n\text{ is even}.
		\end{cases}$ & $\Rightarrow$ & $\colorbox{xred!20}{3},\colorbox{xblue!20}{1},\colorbox{xred!20}{7},\colorbox{xblue!20}{3},\colorbox{xred!20}{11},\colorbox{xblue!20}{5},\colorbox{xred!20}{15},\colorbox{xblue!20}{7}, \dots$\\
	\end{tabular}
\end{example}

Sequences can also be defined using \emph{recursion}, where the value of an element is defined using previous values and a \emph{starting value}. For example:
\[
	a_{n} = a_{n-1}^{2}-2,
\]
with the starting value $a_{1}=3$. We the get that
\[
	a_{2} = a_{1}^{2}-2 = 3^{2}-2 = 7,
\]
and thus
\[
	a_{3} = a_{2}^{2}-2 = 7^{2}-2 = 47,
\]
etc.

\begin{example}{The Fibonacci sequnce}{}
	The \emph{Fibonacci sequences} is a well-known sequence defined using the following recursive rule:
	\[
		F_{n} = F_{n-1} + F_{n-2},
	\]
	with $F_{1}=F_{2}=1$. The first few elements of the sequence are therefore
	\[
		1,1,2,3,5,8,13,21,34,55,89,144,233,377,610,\dots
	\]
	See \autoref{fig:Fib_graphical} for a graphical representation of the Fibonacci sequence.

\end{example}

\begin{figure}
	\pgfmathsetmacro{\scale}{0.5}
	\pgfmathsetmacro{\helixscale}{28.34677*\scale} % Ratio of cm to bp
	\centering
	\begin{tikzpicture}[scale=\scale]
		\def\horizontal{{-1,-1,1,1}}
		\def\vertical{{1,-1,-1,1}}
		\def\center{{1,0}}
		\pgfmathsetmacro{\phi}{1.618033988749} % Golden ratio
		\pgfmathsetmacro{\x}{0}
		\pgfmathsetmacro{\y}{0}
		\foreach \n in {1,...,8}{
			% Calcs
			\pgfmathsetmacro{\k}{\n+1}
			\pgfmathsetmacro{\Fib}{int(ceil((\phi^(\k-1)-(\phi)^(-(\k-1)))/sqrt(5)))} % Fibonacci sequence
			\pgfmathsetmacro{\dx}{\Fib*\horizontal[mod(\n,4)]}
			\pgfmathsetmacro{\dy}{\Fib*\vertical[mod(\n,4)]}
			\pgfmathsetmacro{\dcx}{\center[mod(\n,2)]}
			\pgfmathsetmacro{\dcy}{\center[mod(\n+1,2)]}

			\coordinate (s) at (\x,\y);
			\coordinate (e) at ({\x+\dx},{\y+\dy});
			\coordinate (o) at ({\x+\dcx*\dx},{\y+\dcy*\dy}); % dx, dy, dx, dy, dx, dy, ...

			% Draw
			\pgfmathsetmacro{\m}{int(mod(\n,8))}
			\draw[thick, fill=xcol\m!20] (s) rectangle (e) node[midway, font=\tiny] {$\Fib$};
			\pic[draw, thick, angle radius=\helixscale*\Fib] {angle=s--o--e};

			% Re-calc
			\pgfmathparse{\x+\dx}
			\xdef\x{\pgfmathresult}
			\pgfmathparse{\y+\dy}
			\xdef\y{\pgfmathresult}
		}
	\end{tikzpicture}
	\caption{A graphical representation of the Fibonacci sequence: two squares of side $1$ are placed adjacent to each other on the plane. In each subsequent step a new square is place such that its side is equal to the combined sides of the previous two squares. This way, the side of each square in the sequence follows the Fibonacci sequence. In each square we draw a quarter circle centered on one of the vertices, such that we get the famous \emph{golden ratio} helix.}
	\label{fig:Fib_graphical}
\end{figure}

\begin{note}{Focus of section}{}
	From now on in the section we will focus on infinite sequences only.
\end{note}

\subsection{Types of sequences}%
\label{sub:Types of sequences}
Consider the sequence $a_{n}=n^{2}$. Since $n\in\mathbb{N}$, for any $n,\ a_{n+1} > a_{n}$, since $(n+1)^{2} > n^{2}$ (see \autoref{fig:n^2_seq}). We say that such a sequence is \emph{increasing}. In fact, for a sequence to be increasing some sequential elements can be equal: for example, the sequence $c_{n}=1,1,1,2,2,2,3,3,3,4,4,4,5,5,5,\dots$ is also a increasing sequence. Thus, the definition of a increasing sequence is the following:

\begin{definition}{increasing sequence}
	A sequence $a_{n}$ is said to be \textit{increasing} if for any $n\in\mathbb{N},\ a_{n+1}\geq a_{n}$.
\end{definition}
If we change the condition to $a_{n+1}>a_{n}$ we say that such a sequence is \emph{strictly increasing}. In the above examples $a_{n}$ is a strictly increasing sequence, while $c_{n}$ is just increasing (since for some indeces $n,\ c_{n+1}=c_{n}$).

Similarly, a \emph{decreasing} sequence is a sequence $b_{n}$ for which for any $n\in\mathbb{N},\ b_{n+1}\leq b_{n}$. An example of such sequence is $b_{n}=\frac{1}{n}$ (see \autoref{fig:1/n_seq}). ANd of course, if we change the condition to $b_{n+1}<b_{n}$ then the sequence is \emph{strictly decreasing}.

Generally, a sequence that is either increasin or decreasing is said to be \emph{monotone}. If a sequence is monotone starting only from a certain $n$, we say that the sequence is \emph{eventually monotone} (i.e. \textit{eventually increasing} or \textit{eventually decreasing}). An example of such sequence is $d_{n}=(n-5)^{2}$ (\autoref{fig:(n-5)^2_seq}): for $N\in{1,2,3,4,5}$ it is decreasing, but starting from $n=5$ it is increasing for any $n$.

As an example of a sequence which isn't monote, consider the sequence $e_{n}=\sin(n)$: for some values of $n,\ e_{n+1}>e_{n}$ and for some other values $e_{n+1}<e_{n}$ (see \autoref{fig:sin(n)_seq}).

\begin{figure}[]
	\centering
	\begin{tikzpicture}[]
		\begin{axis}[
			vector plane,
			width=10cm, height=7cm,
			xmin=0, xmax=21,
			ymin=0, ymax=400,
			axis line style={-stealth, thick},
			xlabel={$n$},
			ylabel={$a_{n}=n^{2}$},
			xtick={1,...,20},
			ticklabel style={font=\small},
			domain={0:20},
			samples=21,
		]
		\addplot[xred, only marks, mark=*] {x^2};
		\end{axis}
	\end{tikzpicture}
	\caption{The sequence $a_{n}=n^{2}$ is increasing, and is in fact \textit{strictly} increasing.}
	\label{fig:n^2_seq}
\end{figure}

\begin{figure}[]
	\centering
	\begin{tikzpicture}[]
		\begin{axis}[
			vector plane,
			width=10cm, height=7cm,
			xmin=0, xmax=21,
			ymin=0, ymax=1.1,
			axis line style={-stealth, thick},
			xlabel={$n$},
			ylabel={$b_{n}=\frac{1}{n}$},
			xtick={1,...,20},
			ticklabel style={font=\small},
			domain={0:20},
			samples=21,
		]
		\addplot[xblue, only marks, mark=*] {1/x};
		\end{axis}
	\end{tikzpicture}
	\caption{The sequence $b_{n}=\frac{1}{n}$ is decreasing, and is in fact \textit{strictly} decreasing.}
	\label{fig:1/n_seq}
\end{figure}

\begin{figure}[]
	\centering
	\begin{tikzpicture}[]
		\begin{axis}[
			vector plane,
			width=10cm, height=7cm,
			xmin=0, xmax=21,
			ymin=0, ymax=250,
			axis line style={-stealth, thick},
			xlabel={$n$},
			ylabel={$d_{n}=(n-5)^{2}$},
			xtick={1,...,20},
			ticklabel style={font=\small},
			domain={0:20},
			samples=21,
		]
		\addplot[xpurple, only marks, mark=*] {(x-5)^2};
		\end{axis}
	\end{tikzpicture}
	\caption{The sequence $d_{n}=(n-5)^{2}$ starts as a decreasing sequence, but starting from $n=5$ it is increasing, making it an \textit{eventually increasing sequence}.}
	\label{fig:(n-5)^2_seq}
\end{figure}

\begin{figure}[]
	\centering
	\begin{tikzpicture}[]
		\begin{axis}[
			vector plane,
			width=10cm, height=7cm,
			xmin=0, xmax=21,
			ymin=-1.1, ymax=1.1,
			x axis line style={-stealth, thick},
			xlabel={$n$},
			ylabel={$e_{n}=\sin(n)$},
			xtick={1,...,20},
			ticklabel style={font=\small},
			domain={0:20},
			samples=21,
		]
		\addplot[black!20, dashed, samples=150] {sin(deg(x))};
		\addplot[xorange, only marks, mark=*] {sin(deg(x))};
		\end{axis}
	\end{tikzpicture}
	\caption{The sequence $e_{n}=\sin(n)$ is neither increasing nor decreasing. For reference, the function $\sin(x)$ is plotted as a dashed line behind $e_{n}$.}
	\label{fig:sin(n)_seq}
\end{figure}

The following are two ways to determine whether a sequence $a_{n}$ is monotone:
\begin{descitemize}
	\item[Difference test] if $a_{n+1}-a_{n}\geq0$ for all $n\in\mathbb{N}$, then the sequence is increasing. If $a_{n+1}-a_{n}\leq0$ for all $n\in\mathbb{N}$ then the sequence is decreasing.
	\item[Ratio test] if $\frac{a_{n}+1}{a_{n}}\geq1$ for all $n\in\mathbb{N}$ then the sequence is increasing, and if $\frac{a_{n+1}}{a_{n}}<1$ for all $n\in\mathbb{N}$ then the sequence is decreasing.
\end{descitemize}

\begin{example}{Difference test}{}
	Given the sequence $a_{n}=\frac{n}{n+1}$, we look at the difference $a_{n+1}-a_{n}$:
	\begin{align*}
		a_{n+1}-a_{n} &= \frac{n+1}{n+2}-\frac{n}{n+1} = \frac{(n+1)(n+1)-(n+2)n}{(n+1)(n+2)}\\
					  &= \frac{n^{2}+2n+1-n^{2}-2n}{(n+1)(n+2)} = \frac{1}{(n+1)(n+2)} < 1\quad \forall n\in\mathbb{N}.
	\end{align*}
	The last (in)equality stems from the fact that no matter what $n$ we subtitute into $(n+1)(n+2)$, the result will be greater than $1$, and thus $\frac{1}{(n+1)(n+2)}$ is always smaller than $1$. Therefore, $a_{n}$ is a decreasing sequence.
\end{example}

\begin{example}{Ratio test}{}
	Given the sequence $b_{n}=\frac{2^{n}}{n^{2}}$, the ratio of $a_{n+1}$ to $a_{n}$ is
	\[
		\frac{a_{n+1}}{a_{n}} = \frac{\frac{2^{n+1}}{(n+1)^{2}}}{\frac{2^{n}}{n^{2}}} = 2\frac{n^{2}}{(n+1)^{2}}.
	\]
	Let's look at the first few approximated values of the ratio $\frac{n^{2}}{(n+1)^{2}}$:
	\begin{center}
		\begin{NiceTabular}{p{2mm}l}[
			cell-space-limits=3pt, code-before=\rowcolors{1}{\tabcol!15}{\tabcol!10} \rowcolor{\tabcol!50}{1}
			]
			\toprule
			\RowStyle{\bfseries} $n$ & $\frac{n^{2}}{(n+1)^{2}}$ \\
			\midrule
			0 & 0\\
			1 & 0.25\\
			2 & 0.44444\ldots\\
			3 & 0.5625\\
			4 & 0.64\\
			5 & 0.69444\ldots\\
			6 & 0.7346938775510204\\
			7 & 0.765625\\
			8 & 0.7901234567901234\\
			9 & 0.81\\
			10 & 0.8264462809917356\\
			11 & 0.840277777\ldots\\
			12 & 0.8520710059171598\\
			13 & 0.8622448979591837\\
			\bottomrule
		\end{NiceTabular}
	\end{center}
	We see that for any $n\geq3,\ \frac{n^{2}}{(n+1)^{2}}>\frac{1}{2}$, and therefore $2\frac{n^{2}}{(n+1)^{2}}>1$. Thus, the sequence is eventually increasing.
\end{example}

Some sequences are \emph{bounded} from below: this means that their elements never get smaller than some constant $\underline{M}\in\Rs$. For example, consider the simple sequence $a_{n}=n$, where $n=\left\{1,2,3,4,\dots\right\}$: there is no element in the sequence that is smaller than $1$. Therefore, $a_{n}$ is bounded from below by $1$. Of course, one may argue that $b_{n}$ is also bounded from below by $0$, or $-6$, or in fact any negative number. This is true, however we are usually interested in the \textit{maximal} number $\underline{M}$ that bounds the sequence from below, which in this case is $\underline{M}=1$. We call that number the \emph{infimum} of the sequence, and denote it as $\inf a_{n}$.

Similarly, a sequence $a_{n}$ can be bounded from above by some number $\overline{M}\in\Rs$, i.e. there exist no $n$ for which $a_{n}>\overline{M}$. We call the \textit{minimal} such number the \emph{supremum} of the sequence $a_{n}$, denoted $\sup a_{n}$. For example, the sequence $b_{n}=\frac{1}{n}$ is bounded from above by any real number $x\geq1$, and therefore $\sup b_{n}=1$. In fact, $b_{n}$ is also bounded from below by $\underline{M}=0$, and therefore we say that it is \emph{bounded}. Another example of a sequence that is bounded is $e_{n}=\sin(n)$, which is bounded from below by $\underline{M}=-1$ and from above by $\overline{M}=1$.

\begin{example}{Bounded and unbounded sequences}{}
	The following table shows some examples of sequences that are bounded from below, from above, or neither:

	\begin{center}
		\begin{NiceTabular}{llcc}[
			cell-space-limits=3pt, code-before=\rowcolors{1}{\tabcol!15}{\tabcol!10} \rowcolor{\tabcol!50}{1}
			]
			\toprule
			\RowStyle{\bfseries} $a_{n}$ & First 5 elements & $\inf a_{n}$ & $\sup a_{n}$\\
			\midrule
			$n^{2}-n$ & $0,2,6,12,20,\dots$ & $0$ & - \\
			$\frac{n}{n+1}$ & $\frac{1}{2},\frac{2}{3},\frac{3}{4},\frac{4}{5},\frac{5}{6},\dots$ & $\frac{1}{2}$ & $1$ \\
			$\eu^{-n}$ & $\eu^{-1},\eu^{-2},\eu^{-3},\eu^{-4},\eu^{-5},\ldots$ & $0$ & $\eu^{-1}$\\
			$\log(n)$ & $0,\log(2),\log(3),\log(4),\log(5),\dots$ & $0$ & - \\
			$(-1)^{n}$ & $-1,1,-1,1,-1,\dots$ & $-1$ & $1$\\
			$(-1)^{n}n$ & $-1,2,-3,4,-5,\ldots$ & - & - \\
			$(-2)^{n}$ & $-2,4,-8,16,-32,\dots$ & - & - \\
			\bottomrule
		\end{NiceTabular}
	\end{center}
\end{example}

\subsection{Subsequences}
Given any sequence $a_{n}$, we can remove from it any number of its elements (including $0$ elements) and get a new sequence $b_{n}$ which is a \emph{subsequence} of $a_{n}$. For example, let $a_{n}=n^{2}-5n$. We can remove each 2nd element from $a_{n}$ (i.e. those with indices $2,4,6,8,\ldots$) and get the following sequence $b_{n}$:

\begin{tikzpicture}[node distance=5mm]
	\node (a0) at (0,0) {$a_{n}=$};
	\foreach \n in {1,...,14}{
		\pgfmathsetmacro{\ai}{int(\n^2-5*\n)}
		\pgfmathsetmacro{\m}{int(\n-1)}
		\pgfmathsetmacro{\c}{int(mod(\n,2))}
		\ifnum \c=1
			\def\col{xred}
		\else
			\def\col{xblue}
		\fi
		\node[draw=black, right of=a\m, anchor=west, rounded corners, fill=\col!20] (a\n) {$\ai$};
		\node[right of=a\n, anchor=north east, xshift=1mm] (comma) {$,$};
	}
	\node[right of=comma, xshift=-1mm] (dots1) {$\dots$};
	
	\node[below of=a0, yshift=-15mm] (b0) {$b_{n}=$};
	\foreach \n in {1,...,7}{
		\pgfmathsetmacro{\bi}{int((2*\n-1)^2-5*(2*\n-1))}
		\pgfmathsetmacro{\m}{int(\n-1)}
		\pgfmathsetmacro{\k}{2*\n-1}
		\node[draw=black, right of=b\m, anchor=west, rounded corners, fill=xred!20] (b\n) {$\bi$};
		\node[right of=b\n, anchor=north east, xshift=1mm] (comma) {$,$};
		\draw[vector, thin] (a\k) -- (b\n);
	}
	\node[right of=comma, xshift=-1mm] (dots2) {$\dots$};
\end{tikzpicture}

% \begin{align*}
% 	a_{n}&=
% 	\begin{NiceArray}{llllllllllllll}[name=an, corners=NE, margin, hvlines]
% 		\CodeBefore
% 			\chessboardcolors{xred!30}{xblue!30}
% 		\Body
% 		-4 & -1 & 4 & 11 & 20 & 31 & 44 & 59 & 76 & 95 & 116 & 139 & 164 & 191 \\
% 	\end{NiceArray}\dots\\[1cm]
% 	b_{n}&=
% 	\begin{NiceArray}{lllllll}[name=bn, corners=NE, margin, hvlines]
% 		\CodeBefore
% 			\rowcolor{xred!30}{1}
% 		\Body
% 		-4 & 4 & 20 & 44 & 76 & 116 & 164 \\
% 	\end{NiceArray}\dots
% \end{align*}
% \begin{tikzpicture}[overlay, remember picture]
% 	\foreach \k in {1,...,7}{
% 		\pgfmathsetmacro{\m}{2*\k-1}
% 		\draw[vector, thin] ($(an-1-\m)-(4pt,7pt)$) to [out=-90, in=90, looseness=0.4] ($(bn-1-\k)+(0,5pt)$);
% 	}
% \end{tikzpicture}

