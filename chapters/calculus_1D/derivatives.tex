\section{Derivatives}
One of the most important tool in analyzing a function $f:\Rs\to\Rs$ is the ability to quantitatively describe the way it behaves as we change its argument $x$. At any given point a function can either increase in its value, decrease in its value, or stay constant. We would like to develop a method that is able to tell us exactly how such function changes at any given point.

\begin{example}{Quantitative measure of change}{}
  Compare the following three functions on the domain $x\in[a,b]$:

  \centering
  \begin{tikzpicture}
    \begin{axis}[
        graph2d,
        axis line style={-stealth, thick},
        xmin=1, xmax=6,
        ymin=0, ymax=20,
        domain={1:6},
        xticklabels={,,,,,,$b$},
        extra x ticks={1},
        extra x tick labels={$a$},
        extra x tick style={grid=none},
        yticklabels={},
      ]
       \addplot[function, xred] {exp(x)};
       \addplot[function, xgreen] {x^2};
       \addplot[function, xblue] {x^(1.5)};
    \end{axis}
  \end{tikzpicture}

  \flushleft
  While all three functions are increasing on $[a,b]$ it is clear that the rate of increase between the functions is different: the red function increases faster than the green one, which in turn increases faster than the blue one. In fact, even within each function the increase is not uniform: the more $x$ increases so does the rate of increase of the function.
\end{example}

To start developing a method to quantitatively measure the rate of change in a real function, we can first notice a property of some functions: if we zoom in on any point of the function we would see that the more we zoom in the more the function ``straightens out'', i.e. the function look more and more like a straight line (\autoref{fig:zoom_in}).

\begin{figure}
  \centering
  \begin{tikzpicture}
    \begin{axis}[
        zoomin={1}{1},
      ]
      \addplot[function, xred] {f(x)};
      \draw[thick] (1.4,1.4) rectangle (1.6,1.6);
      \node (1rnw) at (1.4,1.6) {};
      \node (1rne) at (1.6,1.6) {};
      \node (1rse) at (1.6,1.4) {};
      \node (1rsw) at (1.4,1.4) {};
    \end{axis}
    \pgftransformcm{1}{0}{0}{1}{\pgfpoint{0}{-6.5cm}}
    \begin{axis}[
        zoomin={0.1}{0.1},
      ]
      \addplot[function, xred] {f(x)};
      \node (01anw) at (1.4,1.6) {};
      \node (01ane) at (1.6,1.6) {};
      \node (01ase) at (1.6,1.4) {};
      \node (01asw) at (1.4,1.4) {};
      \draw[thick] (1.49,1.49) rectangle (1.51,1.51);
      \node (01rnw) at (1.49,1.51) {};
      \node (01rne) at (1.51,1.51) {};
      \node (01rse) at (1.51,1.49) {};
      \node (01rsw) at (1.49,1.49) {};
    \end{axis}
    \pgftransformcm{1}{0}{0}{1}{\pgfpoint{0}{-6.5cm}}
    \begin{axis}[
        zoomin={0.01}{0.01},
        xticklabel style={/pgf/number format/fixed, /pgf/number format/precision=3},
        yticklabel style={/pgf/number format/fixed, /pgf/number format/precision=3},
      ]
      \node (001anw) at (1.49,1.51) {};
      \node (001ane) at (1.51,1.51) {};
      \node (001ase) at (1.51,1.49) {};
      \node (001asw) at (1.49,1.49) {};
      \addplot[function, xred] {f(x)};
    \end{axis}
    % Lines
    \draw[thick, dashed] (1rnw) -- (01anw);
    \draw[thick, dashed] (1rne) -- (01ane);
    \draw[thick, dashed] (1rse) -- (01ase);
    \draw[thick, dashed] (1rsw) -- (01asw);

    \draw[thick, dashed] (01rnw) -- (001anw);
    \draw[thick, dashed] (01rne) -- (001ane);
    \draw[thick, dashed] (01rse) -- (001ase);
    \draw[thick, dashed] (01rsw) -- (001asw);
  \end{tikzpicture}
  \caption{Zooming in on a real function at $x=1.5$. Note how the more we zoom in, the more the function becomes ``linear'' and we can approximate its slope at $x=1.5$ as being $m=-1$.}
  \label{fig:zoom_in}
\end{figure}

This works for any point of all of the real fundamental functions and their compositions, except for some special points which we will discuss later. To quantify this ``linearity'' given a point on some function we can measure the slope of the line when we zoom in enough. In \autoref{fig:zoom_in} for example, at $x=1.5$ the slope would be approximately $m=-1$.
