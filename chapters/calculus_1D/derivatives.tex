\section{Derivatives}
One of the most important tool in analyzing a function $f:\Rs\to\Rs$ is the ability to quantitatively describe the way it behaves as we change its argument $x$. At any given point a function can either increase in its value, decrease in its value, or stay constant. We would like to develop a method that is able to tell us exactly how such function changes at any given point.

\begin{example}{Quantitative measure of change}{}
  Compare the following three functions on the domain $x\in[a,b]$:

  \centering
  \begin{tikzpicture}
    \begin{axis}[
        graph2d,
        axis line style={-stealth, thick},
        xmin=1, xmax=6,
        ymin=0, ymax=20,
        domain={1:6},
        xticklabels={,,,,,,$b$},
        extra x ticks={1},
        extra x tick labels={$a$},
        extra x tick style={grid=none},
        yticklabels={},
      ]
       \addplot[function, xred] {exp(x)};
       \addplot[function, xgreen] {x^2};
       \addplot[function, xblue] {x^(1.5)};
    \end{axis}
  \end{tikzpicture}

  \flushleft
  While all three functions are increasing on $[a,b]$ it is clear that the rate of increase between the functions is different: the red function increases faster than the green one, which in turn increases faster than the blue one. In fact, even within each function the increase is not uniform: the more $x$ increases so does the rate of increase of the function.
\end{example}

To start developing a method to quantitatively measure the rate of change in a real function, we can first notice a property of some functions: if we zoom in on any point of the function we would see that the more we zoom in the more the function ``straightens out'', i.e. the function looks more and more like a straight line (\autoref{fig:zoom_in}).

\pgfmathsetmacro{\zoomcen}{3.8}
\pgfmathsetmacro{\zoomSlope}{DzoomG(\zoomcen)}
\tikzset{
  bslope/.style={very thick, xblue},
  gslope/.style={very thick, xgreen},
}
\begin{figure}
  \centering
  \begin{tikzpicture}
    \pgfmathsetmacro{\za}{1}
    \pgfmathsetmacro{\zb}{0.2}
    \pgfmathsetmacro{\zc}{0.04}
    \pgfmathsetmacro{\R}{0.03}
    \begin{axis}[
      zoomin={\zoomcen}{\za},
      name=ax1,
      ]
      \draw[bslope] (\zoomcen,\zoomcen) -- ({\zoomcen+\za},{\zoomcen+\za*\zoomSlope});
      \draw[bslope] (\zoomcen,\zoomcen) -- ({\zoomcen-\za},{\zoomcen-\za*\zoomSlope});
      \addplot[function, xred] {zoomF(x,\zoomcen)};
      \fill (\zoomcen,\zoomcen) circle ({\R*\za});
      \coordinate (c1) at ({\zoomcen-\zb},{\zoomcen-\zb});
      \coordinate (c2) at ({\zoomcen-\zb},{\zoomcen+\zb});
      \coordinate (c3) at ({\zoomcen+\zb},{\zoomcen+\zb});
      \coordinate (c4) at ({\zoomcen+\zb},{\zoomcen-\zb});
      \draw (c1) rectangle (c3);
    \end{axis}
    \begin{axis}[
      zoomin={\zoomcen}{\zb},
      name=ax2,
      at={($(ax1.south east)+(2cm,0)$)},
      ]
      \draw[bslope] (\zoomcen,\zoomcen) -- ({\zoomcen+\zb},{\zoomcen+\zb*\zoomSlope});
      \draw[bslope] (\zoomcen,\zoomcen) -- ({\zoomcen-\zb},{\zoomcen-\zb*\zoomSlope});
      \addplot[function, xred] {zoomF(x,\zoomcen)};
      \fill (\zoomcen,\zoomcen) circle ({\R*\zb});
      \coordinate (d1) at ({\zoomcen-\zc},{\zoomcen-\zc});
      \coordinate (d2) at ({\zoomcen-\zc},{\zoomcen+\zc});
      \coordinate (d3) at ({\zoomcen+\zc},{\zoomcen+\zc});
      \coordinate (d4) at ({\zoomcen+\zc},{\zoomcen-\zc});
      \draw (d1) rectangle (d3);
    \end{axis}
    \begin{axis}[
      zoomin={\zoomcen}{\zc},
      name=ax3,
      anchor=north west,
      at={($(ax2.south west)+(0,-2cm)$)},
      ]
      \draw[bslope] (\zoomcen,\zoomcen) -- ({\zoomcen+\zc},{\zoomcen+\zc*\zoomSlope});
      \draw[bslope] (\zoomcen,\zoomcen) -- ({\zoomcen-\zc},{\zoomcen-\zc*\zoomSlope});
      \addplot[function, xred] {zoomF(x,\zoomcen)};
      \fill (\zoomcen,\zoomcen) circle ({\R*\zc});
    \end{axis}
    \draw[dashed] (c3) -- (ax2.north west);
    \draw[dashed] (c4) -- (ax2.south west);
    \draw[dashed] (d1) -- (ax3.north west);
    \draw[dashed] (d4) -- (ax3.north east);
  \end{tikzpicture}
  \caption{Zooming in on a real function at $x=\zoomcen$. Note how the more we zoom in on the function at $x=\zoomcen$, the more the it becomes linear (compare to the blue line).}
  \label{fig:zoom_in}
\end{figure}

This works for almost all the points of any of the real fundamental functions and their compositions (except for some special points which we will discuss later). Of course, we would like to quantify this ``linearity''. We can chose some point $x_{0}$ on some function $f(x)$ and measure the slope of the line we get when we zoom in enough on the point $p_{0}=\left(x_{0},f\left(x_{0}\right)\right)$. In \autoref{fig:zoom_in}, for example, at $x=\zoomcen$ the slope would be approximately $m=-0.4$.

To calculate the slope we can start by marking another point on the function at some distance to the left of $x_{0}$: $x_{1}=x_{0}+\Delta x$. We then connect the two points $\left(x_{0},f\left(x_{0}\right)\right)$ and $\left(a,f(a)\right)$ with a line. The slope of the line would be the difference in the $y$-values of the points divided by the difference in their $x$-values (see \autoref{fig:line_two_points_function}).

\begin{figure}
  \centering
  \begin{tikzpicture}
    \pgfmathsetmacro{\za}{2}
    \pgfmathsetmacro{\dxa}{0.15}
    \begin{axis}[
      zoomin={\zoomcen}{\za},
      xticklabels={,},
      yticklabels={,},
      ]
      \addplot[function, xred] {zoomF(x,\zoomcen)};
      \addplot[only marks, mark=*] coordinates {
          (\zoomcen,{zoomF(\zoomcen,\zoomcen)})
          ({\zoomcen+\dxa},{zoomF({\zoomcen+\dxa},\zoomcen)})
        };
      \draw[bslope] (\zoomcen,\zoomcen) -- ({\zoomcen+\za},{\zoomcen+\za*\zoomSlope});
      \draw[bslope] (\zoomcen,\zoomcen) -- ({\zoomcen-\za},{\zoomcen-\za*\zoomSlope});
      \draw (\zoomcen,{zoomF(\zoomcen,\zoomcen)}) -- ({\zoomcen+\dxa},{zoomF({\zoomcen+\dxa},\zoomcen)});
      \pgfmathsetmacro{\m}{(zoomF(\zoomcen+\dxa,\zoomcen)-\zoomcen)/\dxa}
      \draw[gslope] (\zoomcen,\zoomcen) -- ({\zoomcen+\za},{\zoomcen+\za*\m});
      \draw[gslope] (\zoomcen,\zoomcen) -- ({\zoomcen-\za},{\zoomcen-\za*\m});
    \end{axis}
  \end{tikzpicture}
  \caption{}
  \label{fig:line_two_points_function}
\end{figure}
