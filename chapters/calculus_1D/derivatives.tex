\section{Derivatives}
One of the most important tool in analyzing a function $f:\Rs\to\Rs$ is the ability to quantitatively describe the way it behaves as we change its argument $x$. At any given point a function can either increase in its value, decrease in its value, or stay constant. We would like to develop a method that is able to tell us exactly how such function changes at any given point.

\begin{example}{Quantitative measure of change}{}
  Compare the following three functions on the domain $x\in[a,b]$:

  \centering
  \begin{tikzpicture}
    \begin{axis}[
        graph2d,
        axis line style={-stealth, thick},
        xmin=1, xmax=6,
        ymin=0, ymax=20,
        domain={1:6},
        xticklabels={,,,,,,$b$},
        extra x ticks={1},
        extra x tick labels={$a$},
        extra x tick style={grid=none},
        yticklabels={},
      ]
       \addplot[function, xred] {exp(x)};
       \addplot[function, xgreen] {x^2};
       \addplot[function, xblue] {x^(1.5)};
    \end{axis}
  \end{tikzpicture}

  \flushleft
  While all three functions are increasing on $[a,b]$, it is clear that the rate of increase between the functions is different: the red function increases faster than the green one, which in turn increases faster than the blue one. In fact, even within each function the increase is not uniform: the more $x$ increases so does the rate of increase of the function.
\end{example}

To start developing a method to quantitatively measure the rate of increase of a function, we can first consider one of the simplest functions: $f(x)=x^{2}$. If we "zoom in" on any point of the function we would see that it "straightens out" - the more we zoom in, the more it looks like a straight line (\autoref{eq:zoom_in}).

\newcommand{\xsqrzoomin}[4]{
  \begin{tikzpicture}
    \begin{axis}[
        graph2d,
        axis line style={-stealth, thick},
        xmin=#1, xmax=#2,
        % ymin=0, ymax=20,
        domain={#1:#2},
        xticklabels={},
        yticklabels={},
      ]
       \addplot[function, xred] {x^2};
       \draw[thick] (#3,{(#3)^2}) rectangle (#4,{(#4)^2});
    \end{axis}
  \end{tikzpicture}
}

\begin{figure}
  \centering
  \xsqrzoomin{-10}{10}{2}{3}
  \caption{}
  \label{fig:}
\end{figure}

