\section{Derivatives}
\label{sec:derivatives}

\subsection{Introduction}
\label{sub:introduction}

One of the most important tools in analyzing real functions is the ability to quantitatively describe the way they behaves as we change the argument $x$. At any given point a function can either increase in its value, decrease in its value, stay constant or be undefined. In this section we will explore a method which enables us to quantitatively measure the change in a function's value at (almost) any point in its domain.

\begin{example}{Quantitative measure of change}{}
  Compare the following three functions on the domain $x\in[a,b]$:

  \centering
  \begin{tikzpicture}
    \begin{axis}[
        graph2d,
        axis line style={-stealth, thick},
        xmin=1, xmax=6,
        ymin=0, ymax=20,
        domain={1:6},
        xticklabels={,,,,,,$b$},
        extra x ticks={1},
        extra x tick labels={$a$},
        extra x tick style={grid=none},
        yticklabels={},
      ]
       \addplot[function, xred] {exp(x)};
       \addplot[function, xgreen] {x^2};
       \addplot[function, xblue] {x^(1.5)};
    \end{axis}
  \end{tikzpicture}

  \flushleft
  While all three functions are increasing on $[a,b]$ it is clear that the rate of increase is different in each function: the red function increases faster than the green one, which in turn increases faster than the blue one. In fact, even within each function the increase is not uniform: the more $x$ increases so does the rate of increase of each of the functions.
\end{example}

Any fundamental real function has the following property: if we zoom in enough on some point $\bm{p}=\left(a,f\left(a\right)\right)$ on the function, we would see that it behaves somewhat like a straight line around $\bm{p}$ (\autoref{fig:zoom_in}). In fact, the more we zoom in, the more the function becomes linear around $\bm{p}$. At the limit where the zoom factor is inifinite, the function is exactly linear around $\bm{p}$, and has the same direction (i.e. slope) as the tangent line to the function at $\bm{p}$ (\autoref{fig:derivative_slope}). We call this slope the \emph{derivative} of $f$ at $x=a$, and denote it as $f'(a)$.

\tikzset{
  bslope/.style={very thick, dashed, #1},
}
\begin{figure}
  \centering
  \begin{tikzpicture}
    \pgfmathsetmacro{\zo}{1}
    \pgfmathsetmacro{\za}{0.06}
    \pgfmathsetmacro{\zb}{0.1}
    \pgfmathsetmacro{\zc}{0.005}
    \pgfmathsetmacro{\a}{3.75}
    \pgfmathsetmacro{\b}{3.1}
    \pgfmathsetmacro{\R}{0.03}
    \pgfmathsetmacro{\ma}{DzoomG(\a)}
    \pgfmathsetmacro{\mb}{DzoomG(\b)}
    \begin{axis}[
      zoomin={\a}{\zo},
      width=11cm, height=11cm,
      axis x line=middle,
      axis y line=middle,
      every axis x label/.style={
        at={(ticklabel* cs:1.01)},
        anchor=west,
      },
      every axis y label/.style={
        at={(ticklabel* cs:1.01)},
        anchor=south,
      },
      axis line style={-stealth, thick},
      label style={font=\large},
      xlabel=$x$,
      ylabel=$y$,
      name=ax1,
      ]
      \coordinate (a) at (\a, {zoomF(\a,\a)});
      \coordinate (b) at (\b, {zoomF(\b,\a)});
      \addplot[only marks, mark=*] coordinates {
          (\a,{zoomF(\a,\a)})
          (\b,{zoomF(\b,\a)})
        };
      \addplot[function, xred] {zoomF(x,\a)};
      % Point a
      \coordinate (ca1) at ({\a-\za},{\a-\za});
      \coordinate (ca2) at ({\a-\za},{\a+\za});
      \coordinate (ca3) at ({\a+\za},{\a+\za});
      \coordinate (ca4) at ({\a+\za},{\a-\za});
      \draw (ca1) rectangle (ca3);
      % Point b
      \coordinate (cb1) at ({\b-\zb},{zoomF(\b,\a)-\zb});
      \coordinate (cb2) at ({\b-\zb},{zoomF(\b,\a)+\zb});
      \coordinate (cb3) at ({\b+\zb},{zoomF(\b,\a)+\zb});
      \coordinate (cb4) at ({\b+\zb},{zoomF(\b,\a)-\zb});
      \draw (cb1) rectangle (cb3);
      \node[right of=a, xshift=15pt] (btxt) {$\bm{p}_{a}=\left(a,f(a)\right)$};
      \node[right of=b, xshift=15pt] (btxt) {$\bm{p}_{b}=\left(b,f(b)\right)$};
    \end{axis}
    \begin{axis}[
      zoomin={\a}{\za},
      name=zoom_a,
      at={($(ax1.south east)+(-6cm,7cm)$)},
      ]
      \coordinate (a) at (\a, {zoomF(\a,\a)});
      \coordinate (cc1) at ({\a-\zc},{\a-\zc});
      \coordinate (cc2) at ({\a-\zc},{\a+\zc});
      \coordinate (cc3) at ({\a+\zc},{\a+\zc});
      \coordinate (cc4) at ({\a+\zc},{\a-\zc});
      \draw (cc1) rectangle (cc3);
      \addplot[function, xred] {zoomF(x,\a)};
      \addplot[only marks, mark=*] coordinates {
          (\a,\a)
        };
    \end{axis}
    \begin{axis}[
      zoomin={\b}{\zb},
      name=zoom_b,
      at={($(ax1.south east)+(-7cm,-6cm)$)},
      ]
      \addplot[only marks, mark=*] coordinates {
          (\b,\b)
        };
      \coordinate (b) at (\b, {zoomF(\b,\b)});
      \addplot[function, xred] {zoomF(x,\b)};
      \fill (\b,\b) circle ({\R*\zb});
    \end{axis}
    \begin{axis}[
      zoomin={\a}{\zc},
      width=5cm, height=5cm,
      name=zoom_c,
      at={($(zoom_a.south east)+(2cm,1cm)$)},
      ]
      \coordinate (a) at (\a, {zoomF(\a,\a)});
      \addplot[function, xred] {zoomF(x,\a)};
      \addplot[only marks, mark=*] coordinates {
          (\a,\a)
        };
    \end{axis}
    % To a
    \draw[dashed] (ca2) -- (zoom_a.south west);
    \draw[dashed] (ca3) -- (zoom_a.south east);
    \draw[dashed] (cb1) -- (zoom_b.north west);
    \draw[dashed] (cb4) -- (zoom_b.north east);
    \draw[dashed] (cc3) -- (zoom_c.north west);
    \draw[dashed] (cc4) -- (zoom_c.south west);
  \end{tikzpicture}
  \caption{Zooming in on a real function $f$ at two points: $\bm{p}_{a}=\left(a,f(a)\right)$ (upper right) and $\bm{p}_{b}=\left(b,f(b)\right)$ (bottom right). Note how around each of the points, the function looks somewhat linear: this is more pronounced around $\bm{p}_{b}$ where the function looks linear in the entire zoomed-in area, while near $\bm{p}_{a}$ it looks linear only near the point itself even though the zoom factor is higher.}
  \label{fig:zoom_in}
\end{figure}

\begin{figure}
  \centering
  \begin{tikzpicture}[node distance=13pt]
    \tikzset{
      derivative line/.style={xred, very thick, densely dotted},
      declare function={
        f(\x) = 5*exp(-(\x-6)^2/1.5^2);
        df(\x) = -8/9*(\x-6)*f(\x);
      },
    }
    \pgfmathsetmacro{\a}{5.9}
    \pgfmathsetmacro{\m}{df(\a)}
    \pgfmathsetmacro{\zf}{0.1}
    \begin{axis}[
        name=main,
        graph2d,
        width=11cm, height=11cm, 
        xmin=1, xmax=10,
        ymin=0, ymax=9,
        axis line style={-stealth, thick},
        extra x ticks={\a},
        extra x tick label={$a$},
        extra x tick style={
          major x tick style={draw, xblue},
          tick label style={xblue},
          grid=none,
        },
        extra y ticks={f(\a)},
        extra y tick label={$f(a)$},
        extra y tick style={
          major y tick style={draw, xblue},
          tick label style={xblue},
          grid=none,
        },
        xticklabels={},
        yticklabels={},
        domain={0:10},
        grid=both,
        minor grid style={draw=black!3},
        major grid style={draw=black!3},
      ]
      \draw[derivative line] ({\a-2},{f(\a)-2*\m}) -- ({\a+2},{f(\a)+2*\m}) node[pos=0.7, above, rotate={atan(\m)}] {$m=f'(a)$};
      \addplot[function, xblue] {f(x)};
      \addplot[only marks, mark=*] coordinates {
          (\a,{f(\a)})
        };
      \coordinate (c1) at ({\a-\zf},{f(\a)-\zf});
      \coordinate (c2) at ({\a-\zf},{f(\a)+\zf});
      \coordinate (c3) at ({\a+\zf},{f(\a)+\zf});
      \coordinate (c4) at ({\a+\zf},{f(\a)-\zf});
      \draw[thick] (c1) rectangle (c3);
    \end{axis}
    \begin{axis}[
        at={($(main.south west)+(1cm,7cm)$)},
        name=zoom,
        width=5cm, height=5cm,
        xmin=\a-\zf, xmax=\a+\zf,
        ymin=f(\a)-\zf, ymax=f(\a)+\zf,
        axis line style={-stealth, thick},
        xticklabels={},
        yticklabels={},
        domain={\a-\zf:\a+\zf},
        grid=both,
        minor grid style={draw=black!3},
        major grid style={draw=black!3},
      ]
      \draw[derivative line] ({\a-2},{f(\a)-2*\m}) -- ({\a+2},{f(\a)+2*\m});
      \addplot[function, xblue] {f(x)};
      \addplot[only marks, mark=*] coordinates {
          (\a,{f(\a)})
        };
      \coordinate (A) at (\a,{f(\a)});
      \node[above left of=A] {$\left(a,f(a)\right)$};
    \end{axis}
  \draw[dashed] (c2) -- (zoom.south west);
  \draw[dashed] (c3) -- (zoom.south east);
  \end{tikzpicture}
  \caption{The derivative of the function $f(x)$ at $x=a$ is equal to the slope of the tangent line to $f$ at the point $\left(a,f(a)\right)$.}
  \label{fig:derivative_slope}
\end{figure}

\subsection{Sign and stationary points}
\label{sub:sign and stationary points}
The derivative of a function $f$ at the point $a$ can be one of four possible categories (\autoref{fig:derivative_sign}):
\begin{itemize}
  \item $f(a)>0$, meaning that $f$ \textbf{increases} at $a$.
  \item $f(a)<0$, meaning that $f$ \textbf{decreases} at $a$.
  \item $f(a)=0$, meaning that $a$ is a \emph{stationary point}.
  \item $f(a)$ is undefined, which can mean different things and which we will address later.
\end{itemize}

\begin{figure}
  \centering
  \begin{tikzpicture}[node distance=8pt]
    \pgfmathsetmacro{\a}{2}
    \pgfmathsetmacro{\b}{3.122}
    \pgfmathsetmacro{\c}{6}
    \pgfmathsetmacro{\d}{7.544}
    \pgfmathsetmacro{\e}{9.6}
    \begin{axis}[
      graph2d,
      width=12cm, height=10cm,
      axis line style={-stealth, thick},
      xmin=0, xmax=12,
      ymin=0, ymax=10,
      domain={0:12},
      xticklabels={},
      yticklabels={},
      extra x ticks={\a, \b, \c, \d, \e},
      extra x tick labels={$x_{1}$, $x_{2}$, $x_{3}$, $x_{4}$, $x_{5}$},
      extra x tick style={
        major x tick style={draw, xblue},
        tick label style={xblue},
        grid=none,
      },
      grid=both,
      major grid style={black!3},
      minor grid style={black!3},
      declare function={
        f(\x) = 0.15*(\x-8)^3+1.2*(\x-8)^2+(\x-8)+1;
        Df(\x) = (9*\x^2-96*\x+212)/20;
      }
    ]
    \addplot[function, xgreen] {f(x)};
    \addplot[only marks, mark=*] coordinates {
        (\a,{f(\a)})
        (\b,{f(\b)})
        (\c,{f(\c)})
        (\d,{f(\d)})
        (\e,{f(\e)})
      };
    \foreach \x in {\a, \b, \c, \d, \e}{
      \pgfmathsetmacro{\M}{Df(\x)}
      \pgfmathsetmacro{\N}{sqrt(1/(1+(\M)^2))} % normalizes the length of the line
      \edef\drawSlope{\noexpand\draw[vector, xred] ({\x-\N},{f(\x)-\N*\M}) -- ({\x+\N},{f(\x)+\N*\M});}
      \edef\drawDashes{\noexpand\draw[dashed, xblue!20] (\x,0) -- (\x,{f(\x)});}
      \edef\createCoord{\noexpand\coordinate (p-\x) at (\x, {f(\x)});}
      \drawSlope
      \drawDashes
      \createCoord
    }
    % I would love to do this proceduraly, but it keeps giving me errors so manually it is ¯\_(ツ)_/¯
    \node[left of=p-\a, text=xred, xshift=-20pt] {$f'\left(x_{1}\right)>0$};
    \node[above of=p-\b, text=xred, yshift=5pt] {$f'\left(x_{2}\right)=0$};
    \node[left of=p-\c, text=xred, xshift=-20pt] {$f'\left(x_{3}\right)<0$};
    \node[below of=p-\d, text=xred] {$f'\left(x_{4}\right)=0$};
    \node[left of=p-\e, text=xred, xshift=-20pt] {$f'\left(x_{5}\right)>0$};
    \end{axis}
  \end{tikzpicture}
  \caption{The derivative of a function $f(x)$ at some points $\bm{p}_{i}=\left(x_{i},f\left(x_{i}\right)\right)$ on the function: when $f$ increases the derivative is positive ($\bm{p}_{1},\bm{p}_{5}$), when it decreases the derivative is negative ($\bm{p}_{3}$), and when it is stationary the derivative equals zero ($\bm{p}_{2},\bm{p}_{4}$). The slope of the tangent line to $f$ at each of the points $\bm{p}_{i}$ is drawn as an arrow to make its sign (positive, negative or zero) more clear.}
  \label{fig:derivative_sign}
\end{figure}

\tbw{a graph of a ``complicated'' function together with its derivative}

\tbw{when the derivative isn't defined}

\tbw{differentiability on an interval}

\tbw{a summary of what we learned so far}

\subsection{Calculating the derivative}
\label{sub:calculating the derivative}
How can we quantify the derivative? Let us consider some real function $f$ and a point $\bm{p}_{0} = \left(x_{0},f\left(x_{0}\right)\right)$ on the function. We can then define another point to the right of $x_{0}$: $\bm{p}_{1}=\left(x_{1},f\left(x_{1}\right)\right)$. Since $x_{1}$ is to the right of $x_{0}$ we can write it as $x_{1}=x_{0}+\Delta x$, where $\Delta x>0$. We then connect the two points with a line (\autoref{fig:x0_x1_line}). The slope of this line can then be calculated using \autoref{eq:linear_slope_def}:
\begin{equation}
  m = \frac{\Delta y}{\Delta x} = \frac{f\left(x_{1}\right)-f\left(x_{0}\right)}{x_{1}-x_{0}} = \frac{f\left(x_{0}+\Delta x\right)-f\left(x_{0}\right)}{x_{0}+\Delta x - x_{0}} = \frac{f\left(x_{0}+\Delta x\right)-f\left(x_{0}\right)}{\Delta x}.
  \label{eq:derivative_slope_def}
\end{equation}

\begin{figure}
  \centering
  \begin{tikzpicture}[node distance=13pt]
    \pgfmathsetmacro{\R}{3}
    \pgfmathsetmacro{\c}{1.5}
    \pgfmathsetmacro{\xa}{1.5}
    \pgfmathsetmacro{\xb}{3.5}
    \begin{axis}[
      graph2d,
      width=9cm, height=9cm,
      xmin=0, xmax=4,
      ymin=0, ymax=4,
      xticklabels={,},
      yticklabels={,},
      grid=major,
      major grid style={draw=black!2},
      extra x tick style={
        tick label style={xblue},
        major grid style=black!0,
      },
      extra x ticks={\xa, \xb},
      extra x tick labels={$x_{0}$, $x_{1}$},
      extra y tick style={
        tick label style={xblue},
        major grid style=black!0,
      },
      extra y ticks={{fd(\xa)}, {fd(\xb)}},
      extra y tick labels={$f\left(x_{0}\right)$, $f\left(x_{1}\right)$},
      axis line style={-stealth},
      domain={0:4},
      samples=100,
      declare function={
        fd(\x)=sqrt(\R^2-(\x-\c)^2)-1;
        Dfd(\x)=-(\x-\c)/sqrt(\R^2-(\x-\c)^2);
      },
    ]
    % Coordinate lines lines
    \tikzset{prp/.style={perp, black!20}}
    \draw[prp] (\xa,0) -- (\xa,{fd(\xa});
    \draw[prp] (\xb,0) -- (\xb,{fd(\xb});
    \draw[prp] (0,{fd(\xa)}) -- (\xa,{fd(\xa});
    \draw[prp] (0,{fd(\xb)}) -- (\xb,{fd(\xb});
    % Derivative slope at x=\c
    \pgfmathsetmacro{\m}{(fd(\xb)-fd(\xa))/(\xb-\xa)}
    % Line between points and its slope
    \draw[xred, very thick] (\xa, {fd(\xa)}) -- (\xb,{fd(\xb)});
    % \addplot[function, xred, dashed] {\m*(x-\xa)+fd(\xa)};
    % % Derivative at x0
    % \pgfmathsetmacro{\M}{Dfd(\xa)}
    % \addplot[function, xgreen, dashed] {\M*(x-\xa)+fd(\xa)};
    % Function + points on it
    \addplot[function, xblue] {fd(x)};
    \addplot[only marks, mark=*] coordinates {
        (\xa,{fd(\xa)})
        (\xb,{fd(\xb)})
      };
    % Labels
    \coordinate (p0) at (\xa,{fd(\xa)});
    \coordinate (p1) at (\xb,{fd(\xb)});
    \node[above of=p0] (p0lbl) {$\bm{p}_{0}$};
    \node[above of=p1] (p1lbl) {$\bm{p}_{1}$};
    \end{axis}
  \end{tikzpicture}
  \caption{Text.}
  \label{fig:x0_x1_line}
\end{figure}

We can then take the limit of \autoref{eq:derivative_slope_def} as $\Delta x\to 0$ (\autoref{fig:derivate_limit}):
\begin{equation}
  M = \lim\limits_{\Delta x\to 0} \frac{f\left(x_{0}+\Delta x\right)-f\left(x_{0}\right)}{\Delta x}.
  \label{eq:derivative_def}
\end{equation}
This limit is defined as \emph{the derivative} of $f$ at the point $x=x_{0}$, and it tells us, quantitatively, how $f$ locally behaves at $x_{0}$, i.e. how much does it increase, decrease or stay the same around $x_{0}$.

\newcommand{\derivFig}[1]{
  \begin{tikzpicture}[node distance=10pt]
    \pgfmathsetmacro{\R}{3}
    \pgfmathsetmacro{\c}{1.5}
    \pgfmathsetmacro{\xa}{1.5}
    \pgfmathsetmacro{\xb}{\xa+#1}
    \begin{axis}[
      graph2d,
      width=6cm, height=6cm,
      xmin=0, xmax=4,
      ymin=-0.5, ymax=3.5,
      xticklabels={,},
      yticklabels={,},
      extra x tick style={
        tick label style={xblue},
        major grid style=black!0,
      },
      extra x ticks={\xa, \xb},
      extra x tick labels={},
      grid=major,
      major grid style={draw=black!2},
      axis line style={-stealth},
      domain={0:4},
      samples=40,
      declare function={
        fd(\x)=sqrt(\R^2-(\x-\c)^2)-1;
        Dfd(\x)=-(\x-\c)/sqrt(\R^2-(\x-\c)^2);
      },
    ]
    % Derivative slope at x=\c
    \pgfmathsetmacro{\m}{(fd(\xb)-fd(\xa))/(\xb-\xa)}
    % Brace
    \draw[thick, decorate, decoration={brace, amplitude=3pt, raise=3pt, mirror}]
          (\xa,0) -- (\xb,0) node[midway, below, yshift=-5pt]{$\Delta x$};
    \draw[perp] (\xa,0) -- (\xa,{fd(\xa)});
    \draw[perp] (\xb,0) -- (\xb,{fd(\xb)});
    % Line between points and its slope
    \addplot[function, xred] {\m*(x-\xa)+fd(\xa)};
    % Derivative at x0
    \pgfmathsetmacro{\M}{Dfd(\xa)}
    \addplot[function, xgreen] {\M*(x-\xa)+fd(\xa)};
    % Function + points on it
    \addplot[function, xblue] {fd(x)};
    \addplot[only marks, mark=*] coordinates {
        (\xa,{fd(\xa)})
        (\xb,{fd(\xb)})
      };
    % Labels
    \coordinate (p0) at (\xa,{fd(\xa)});
    \coordinate (p1) at (\xb,{fd(\xb)});
    \node[above of=p0] (p0lbl) {$\bm{p}_{0}$};
    \node[below of=p1] (p1lbl) {$\bm{p}_{1}$};
    \end{axis}
  \end{tikzpicture}
}

\begin{figure}
  \centering
  \begin{subfigure}[b]{0.45\textwidth}
    \centering
    \derivFig{2.4}
  \end{subfigure}
  \hfill
  \begin{subfigure}[b]{0.45\textwidth}
    \centering
    \derivFig{2.0}
  \end{subfigure}
  \hfill
  \begin{subfigure}[b]{0.45\textwidth}
    \centering
    \derivFig{1.5}
  \end{subfigure}
  \hfill
  \begin{subfigure}[b]{0.45\textwidth}
    \centering
    \derivFig{1}
  \end{subfigure}
  \hfill
  \begin{subfigure}[b]{0.45\textwidth}
    \centering
    \derivFig{0.3}
  \end{subfigure}
  \hfill
  \begin{subfigure}[b]{0.45\textwidth}
    \centering
    \derivFig{0.1}
  \end{subfigure}
  \hfill
  \caption{As $\Delta x\to 0$, $\bm{p}_{1}$ approaches $\bm{p}_{0}$ and the slope of the red line connecting the two points approches the slope $M$ of the green line at $\bm{p}_{0}$.}
  \label{fig:derivate_limit}
\end{figure}

\begin{example}{Validation of the derivative using a linear function}{linear_derivative}
  Given a linear function $f(x)=mx+b$, we expect that the derivative of $f$ at any point $x_{0}$ would equal $m$, since the entire function is a line connecting all the points on the function itself. Let us check that:
  \begin{align*}
    M &= \lim\limits_{\Delta x\to 0}\frac{f\left(x_{0}+\Delta x\right)-f\left(x_{0}\right)}{\Delta x}\\
      &= \lim\limits_{\Delta x\to 0}\frac{m\left(x_{0}+\Delta x\right)+\cancel{b}-\left(mx_{0}+\cancel{b}\right)}{\Delta x}\\
      &= \lim\limits_{\Delta x\to 0}\frac{\cancel{mx_{0}}+m\Delta x-\cancel{mx_{0}}}{\Delta x}\\
      &= \lim\limits_{\Delta x\to 0}\frac{m\cancel{\Delta x}}{\cancel{\Delta x}}\\
      &= m.
  \end{align*}
\end{example}

\begin{example}{Derivative of $x^{2}$}{derivative_x^2}
  Unlike for a linear function, we shouldn't expect the derivate of $f(x)=x^{2}$ to be constant at any point. However, we can easily calculate what the derivative would be at some point $x_{0}$:
  \begin{align*}
    M &= \lim\limits_{\Delta x\to 0}\frac{\left(x_{0}+\Delta x\right)^{2}-x_{0}^{2}}{\Delta x}\\
      &= \lim\limits_{\Delta x\to 0}\frac{\cancel{x_{0}^{2}}+2x_{0}\Delta x-\left(\Delta x\right)^{2}-\cancel{x_{0}^{2}}}{\Delta x}\\
      &= \lim\limits_{\Delta x\to 0}\frac{\cancel{\Delta x}\left(2x_{0}-\Delta x\right)}{\cancel{\Delta x}}\\
      &= \lim\limits_{\Delta x\to 0}2x_{0}-\Delta x\\
      &= 2x_{0}.
  \end{align*}
  I.e. we see that any point $x_{0}$ the derivative of $f(x)=x^{2}$ is simply $2x_{0}$.

  For example, at $x_{0}=3$ the derivative is $M=6$, and at $x_{0}=0$ the derivative is $M=0$.
\end{example}

Up until now we have regarded the derivative as a property of some point on a function $f$. However, since we can calculate the derivative at each point of the function\footnote{except for some points which we will discuss later.}, we can collect all these points together to form a new function, which we call the \emph{derivative} of $f$ and denote as $f'$ (read: ``$f$-prime'').

In \autoref{example:linear_derivative} we saw that the derivative of a linear function at any point gives the same value $m$ (namely the slope of the linear function). Therefore, this derivative is itself a \textit{constant} function $f'(x)=m$. When we calculated the derivative of $f(x)=x^{2}$ (\autoref{example:derivative_x^2}), we found that it depends on the point where it was calculated, using the relation $f'(x)=2x$, which is a linear function with slope $2$ that goes through the origin.

Let us now calculate the derivative of some common functions.
\begin{example}{Derivative of $ax^{n}$}{derivative_ax^n}
  The derivative of the function $f(x)=x^{n}$ (where $a\in\Rs$ is a constant) is (recall the binomial expansion, \autoref{eq:binomial_expansion_n}):
  \begin{align*}
    f'(x) &= \lim\limits_{\Delta x\to 0}\frac{(x+\Delta x)^{n}-x^{n}}{\Delta x}\\
          &= \lim\limits_{\Delta x\to 0}\frac{\cancel{x^{n}} + nx^{n-1}\Delta x + \binom{2}{n}x^{n-2}\left(\Delta x\right)^{2} + \dots + nx\left(\Delta x\right)^{n-1} + \left(\Delta x\right)^{n}-\cancel{x^{n}}}{\Delta x}.
  \end{align*}
  We can take $\Delta x$ out of the numerator and cancel it out with the $\Delta x$ in the denominator:
  \begin{align*}
    f'(x) &= \lim\limits_{\Delta x\to 0}\frac{\cancel{\Delta x}\left[ nx^{n-1} + \binom{2}{n}x^{n-2}\Delta x + \dots + nx\left(\Delta x\right)^{n-2} + \left(\Delta x\right)^{n-1} \right]}{\cancel{\Delta x}}\\
          &= \lim\limits_{\Delta x\to 0} nx^{n-1} + \binom{2}{n}x^{n-2}\Delta x + \dots + nx\left(\Delta x\right)^{n-2} + \left(\Delta x\right)^{n-1}.
  \end{align*}
  Since all expressions except $nx^{n-1}$ have some power of $\Delta x$ in them, in the limit $\Delta x\to 0$ they all vanish, leaving us with
  \[
    f'(x) = nx^{n-1}.
  \]

  This derivative is commonly described as the power of $x$ being reduced by $1$ and the expression gaining a factor of $n$ (i.e. the power before reducing it).
\end{example}

\begin{example}{Derivative of $a^{x}$}{}
  Calculating the derivative of $a^{x}$:

  \begin{align*}
    \left(a^{x}\right)' &= \lim\limits_{\Delta x\to 0}\frac{a^{x+\Delta x}-a^{x}}{\Delta x}\\
                        &= \lim\limits_{\Delta x\to 0}\frac{a^{x}(a^{\Delta x}-1)}{\Delta x}\\
                        &= a^{x}\lim\limits_{\Delta x\to 0}\frac{a^{\Delta x}-1}{\Delta x}.
  \end{align*}
  (we take $a^{x}$ out of the limit expression because it isn't affected by a change in $\Delta x$)

  According to \autoref{eq:limit_def_log}, the limit in the last expression is one of the defintions of $\ln(a)$, and therefore
  \[
    \left(a^{x}\right)' = a^{x}\ln(x).
  \]

  In the case where $a=\eu$, we get
  \[
    \left(\eu^{x}\right)' = \eu^{x}\ln(\eu) = \eu^{x},
  \]
  i.e. $\eu^{x}$ is its own derivative.
\end{example}

\begin{example}{Derivative of $\sin(x)$}{}
  Calculating the derivative of $\sin(x)$:
  \begin{align*}
    \sin'(x) &= \lim\limits_{\Delta x\to 0}\frac{\sin\left(x+\Delta x\right)-\sin(x)}{\Delta x}\\
             &= \lim\limits_{\Delta x\to 0}\frac{\sin(x)\cos\left(\Delta x\right)+\cos(x)\sin\left(\Delta x\right)-\sin(x)}{\Delta x}.
  \end{align*}
  We can separate the three terms into three limits:
  \[
    \sin'(x) = \lim\limits_{\Delta x\to 0}\frac{\sin(x)\left[\cos\left(\Delta x\right)-1\right]}{\Delta x} + \lim\limits_{\Delta x\to 0}\frac{\cos(x)\sin\left(\Delta x\right)}{\Delta x}.
  \]
  The second limit equals $\cos(x)$, since $\lim\limits_{c\to 0}\frac{\sin(c)}{c}=1$ (\autoref{eq:XXX}). Since $\sin(x)$ does not change as we decrease $\Delta x$, we can regard it as a constant and take it out of the limit:
  \[
    \sin'(x) = \sin(x)\lim\limits_{\Delta x\to 0}\frac{\cos\left(\Delta x\right)-1}{\Delta x} + \cos(x).
  \]
  Using the double angle identity (\autoref{eq:double_angle}) on $\cos\left(\Delta x\right)$ we get that
  \[
    \cos\left(\Delta x\right) = 1-2\sin^{2}\left(\frac{\Delta x}{2}\right),
  \]
  and by plugging this back into the derivative calculation we get:
  \begin{align*}
    \sin'(x) &= \lim\limits_{\Delta x\to 0}\frac{-2\sin^{2}\left(\frac{\Delta x}{2}\right)}{\Delta x} + \cos(x)\\
             &= \lim\limits_{\Delta x\to 0}\frac{-\cancel{\frac{2}{2}}\sin^{2}\left(\frac{\Delta x}{2}\right)}{\frac{\Delta x}{2}} + \cos(x)\\
             &= \cos(x),
  \end{align*}
  since $\lim\limits_{h\to0}\frac{\sin^{2}(h)}{h}=0$ (\autoref{eq:XXX}).

  \tbw{in the limit section show and prove this limit}
\end{example}

\begin{example}{Derivative of $\sqrt{x}$}{}
  The derivative of the function $f(x)=\sqrt{x}$ is:
  \[
    f'(x) = \lim\limits_{\Delta x\to 0}\frac{\sqrt{x+\Delta x}-\sqrt{x}}{\Delta x}.
  \]
  We can multiply the numerator and denominator each by $\sqrt{x+\Delta x}+\sqrt{x}$. This would allow us to use the relation $(a-b)(a+b)=a^{2}-b^{2}$:
  \begin{align*}
    f'(x) &= \lim\limits_{\Delta x\to 0}\frac{\sqrt{x+\Delta x}-\sqrt{x}}{\Delta x}\\
          &= \lim\limits_{\Delta x\to 0}\frac{\cancelcol[xred]{x}+\cancelcol[xblue]{\Delta x}-\cancelcol[xred]{x}}{\cancelcol[xblue]{\Delta x}\left(\sqrt{x+\Delta x}+\sqrt{x}\right)}\\
          &= \frac{1}{\sqrt{x}+\sqrt{x}}\\
          &= \frac{1}{2\sqrt{x}}.
  \end{align*}
\end{example}

\autoref{tab:common_derivatives} lists some common functions and their derivatives.
\begin{table}[htpb]
	\centering
	\caption{Some common real functions and their derivatives.}
	\label{tab:common_derivatives}
	\begin{NiceTabular}{lll}[
			cell-space-limits=3pt, code-before=\rowcolors{1}{\tabcol!15}{\tabcol!10} \rowcolor{\tabcol!50}{1}
		]
		\toprule
    \RowStyle{\bfseries} $f(x)$ & $f'(x)$ & Remarks\\
		\midrule
    $c$ & $0$ & $c\in\Rs$\\
    $mx+b$ & $m$ & $m,b\in\Rs$\\
    $x^{2}$ & $2x$ & \\
    $x^{n}$ & $nx^{n-1}$ & \\
    $\sqrt{x}$ & $\frac{1}{2\sqrt{x}}$ & $x\geq0$\\
    $\eu^{x}$ & $\eu^{x}$ &\\
    $a^{x}$ & $a^{x}\log(a)$ & $a>0$\\
    $\ln(x)$ & $\frac{1}{x}$ & $x>0$\\
    $\log_{b}(x)$ & $\frac{1}{x\ln(b)}$ & $x,b>0$\\
    $\sin(x)$ & $\cos(x)$ & \\
    $\cos(x)$ & $\sin(x)$ & \\
    $\tan(x)$ & $\frac{1}{\cos^{2}(x)}=1+\tan^{2}(x)$ & $x\notin\left\{\frac{\pi}{2}+k\right\},k\in\mathbb{Z}$\\
    $\arcsin(x)$ & $\frac{1}{\sqrt{1-x^{2}}}$ & $-1<x<1$\\
    $\arccos(x)$ & $-\frac{1}{\sqrt{1-x^{2}}}$ & $-1<x<1$ \\
    $\arctan(x)$ & $\frac{1}{1+x^{2}}$ & \\
    $\sinh(x)$ & $\cosh(x)$ & \\
    $\cosh(x)$ & $\sinh(x)$ & \\
    $\tanh(x)$ & $1-\tanh^{2}(x)$ & \\
		\bottomrule
	\end{NiceTabular}
\end{table}

\section{Linearity and combined functions}
\label{sec:linearity_of_the_derivative}

More often than not we need to derive functions which aren't simple fundamental functions. For example, we might want to derive the polynomial
\[
  P(x)=2x^{4}-3x^{3}-2x+7.
\]
Using the rule we found in \autoref{example:derivative_ax^n} we can derive each term of $P$ separately, yielding the following:
\[
  \left(2x^{4}\right)'=8x^{3},\quad \left(-3x^{3}\right)'=-9x^{2},\quad \left(-2x\right)'=-2,\quad \left(7\right)'=0.
\]

But how do we derive the entire polynomial? Luckily for us there's an exceedingly simple rule for getting the derivative of any addition of two functions: we simply derive each of the functions separately and add the result. For example, the derivative of $P$ is simply the sum of the derivatives of each of the terms of $P$:
\[
  P'(x) = 8x^{3}-9x^{2}-2.
\]

\begin{example}{Derivatives of sums}{}
  The derivative of the function
  \[
    f(x) = 4x^{7} + \sin(x) + 5^{x} + \log_{7}(x)
  \]
  is
  \begin{align*}
    f'(x) &= \left(4x^{7}\right)' + \left(\sin(x)\right)' + \left(5^{x}\right)' + \left(\log_{7}(x)\right)'\\
          &= 28x^{6} + \cos(x) + 5^{x}\ln(5) + \frac{1}{x\ln(7)}.
  \end{align*}
\end{example}

In the most general case, given two real functions $f$ and $g$ which are differentiable over some interval $I$, then for any $x\in I$
\begin{equation}
  \left(\left[f+g\right](x)\right)' = f'(x)+g'(x).
  \label{eq:derivative_additivity}
\end{equation}

A similar idea also applies to functions which are scaled by a real number $\alpha$: if $f$ is differentiable on some interval $I$, then for any $x\in I$
\begin{equation}
  \left(\alpha f(x)\right)' = \alpha f'(x).
  \label{eq:derivative_scalability}
\end{equation}

\begin{example}{Derivative of scaled functions}{}
  The derivative of the function
  \[
    f(x) = \frac{1}{2}\cos(x)
  \]
  is simply
  \[
    f'(x) = -\frac{1}{2}\sin(x).
  \]
\end{example}

\tbw{the derivative as a linear operator}
