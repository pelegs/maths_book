\section{Derivatives}
One of the most important tool in analyzing a function $f:\Rs\to\Rs$ is the ability to quantitatively describe the way it behaves as we change its argument $x$. At any given point a function can either increase in its value, decrease in its value, or stay constant. We would like to develop a method that is able to tell us exactly how such function changes at any given point.

\begin{example}{Quantitative measure of change}{}
  Compare the following three functions on the domain $x\in[a,b]$:

  \centering
  \begin{tikzpicture}
    \begin{axis}[
        graph2d,
        axis line style={-stealth, thick},
        xmin=1, xmax=6,
        ymin=0, ymax=20,
        domain={1:6},
        xticklabels={,,,,,,$b$},
        extra x ticks={1},
        extra x tick labels={$a$},
        extra x tick style={grid=none},
        yticklabels={},
      ]
       \addplot[function, xred] {exp(x)};
       \addplot[function, xgreen] {x^2};
       \addplot[function, xblue] {x^(1.5)};
    \end{axis}
  \end{tikzpicture}

  \flushleft
  While all three functions are increasing on $[a,b]$ it is clear that the rate of increase between the functions is different: the red function increases faster than the green one, which in turn increases faster than the blue one. In fact, even within each function the increase is not uniform: the more $x$ increases so does the rate of increase of the function.
\end{example}

To start developing a method to quantitatively measure the rate of change in a real function, we can first notice a property of some functions: if we zoom in on some point of the function $\bm{p}=\left(a,f\left(a\right)\right)$ we would see that the more we zoom in the more the the function behaves like a straight line around $\bm{p}$ (\autoref{fig:zoom_in}). We can define the limit of this zoom (i.e. when the zoom factor goes to infinity) as the slope of the function at the $\bm{p}$.

\tikzset{
  bslope/.style={very thick, dashed, #1},
}
\begin{figure}
  \centering
  \begin{tikzpicture}
    \pgfmathsetmacro{\zo}{1}
    \pgfmathsetmacro{\za}{0.02}
    \pgfmathsetmacro{\zb}{0.1}
    \pgfmathsetmacro{\a}{3.75}
    \pgfmathsetmacro{\b}{3.1}
    \pgfmathsetmacro{\R}{0.03}
    \pgfmathsetmacro{\ma}{DzoomG(\a)}
    \pgfmathsetmacro{\mb}{DzoomG(\b)}
    \begin{axis}[
      zoomin={\a}{\zo},
      width=11cm, height=11cm,
      axis x line=middle,
      axis y line=middle,
      every axis x label/.style={
        at={(ticklabel* cs:1.01)},
        anchor=west,
      },
      every axis y label/.style={
        at={(ticklabel* cs:1.01)},
        anchor=south,
      },
      axis line style={-stealth, thick},
      label style={font=\large},
      xlabel=$x$,
      ylabel=$y$,
      name=ax1,
      ]
      \addplot[only marks, mark=*] coordinates {
          (\a,{zoomF(\a,\a)})
          (\b,{zoomF(\b,\a)})
        };
      \addplot[function, xred] {zoomF(x,\a)};
      % Point a
      \coordinate (ca1) at ({\a-\za},{\a-\za});
      \coordinate (ca2) at ({\a-\za},{\a+\za});
      \coordinate (ca3) at ({\a+\za},{\a+\za});
      \coordinate (ca4) at ({\a+\za},{\a-\za});
      \draw (ca1) rectangle (ca3);
      % Point b
      \coordinate (cb1) at ({\b-\zb},{zoomF(\b,\a)-\zb});
      \coordinate (cb2) at ({\b-\zb},{zoomF(\b,\a)+\zb});
      \coordinate (cb3) at ({\b+\zb},{zoomF(\b,\a)+\zb});
      \coordinate (cb4) at ({\b+\zb},{zoomF(\b,\a)-\zb});
      \draw (cb1) rectangle (cb3);
    \end{axis}
    \begin{axis}[
      zoomin={\a}{\za},
      name=zoom_a,
      at={($(ax1.south east)+(-6cm,7cm)$)},
      ]
      \coordinate (a) at (\a, {zoomF(\a,\a)});
      % \draw[bslope={xblue}] (\a,\a) -- ({\a+\za},{\a+\za*\ma});
      % \draw[bslope={xblue}] (\a,\a) -- ({\a-\za},{\a-\za*\ma});
      \addplot[function, xred] {zoomF(x,\a)};
      \addplot[only marks, mark=*] coordinates {
          (\a,\a)
        };
      \node[above right of=a, yshift=-3pt] (atxt) {$\bm{p}_{a}=\left(a,f(a)\right)$};
    \end{axis}
    \begin{axis}[
      zoomin={\b}{\zb},
      name=zoom_b,
      at={($(ax1.south east)+(-7cm,-7cm)$)},
      ]
      \addplot[only marks, mark=*] coordinates {
          (\b,\b)
        };
      \coordinate (b) at (\b, {zoomF(\b,\b)});
      \draw[bslope={xdarkgreen}] (\b,\b) -- ({\b+\zb},{\b+\zb*\mb});
      \draw[bslope={xdarkgreen}] (\b,\b) -- ({\b-\zb},{\b-\zb*\mb});
      \addplot[function, xred] {zoomF(x,\b)};
      \fill (\b,\b) circle ({\R*\zb});
      \node[right of=b, anchor=west, xshift=-20pt] (btxt) {$\bm{p}_{b}=\left(b,f(b)\right)$};
    \end{axis}
    % To a
    \draw[dashed] (ca2) -- (zoom_a.south west);
    \draw[dashed] (ca3) -- (zoom_a.south east);
    \draw[dashed] (cb1) -- (zoom_b.north west);
    \draw[dashed] (cb4) -- (zoom_b.north east);
  \end{tikzpicture}
  \caption{Zooming in on a real function $f$ at two points: $\bm{p}_{a}=\left(a,f(a)\right)$ (upper right) and $\bm{p}_{b}=\left(b,f(b)\right)$ (bottom right). Note how around each of the points, the function looks somewhat linear: this is more pronounced around $\bm{p}_{b}$ where the function looks linear in the entire zoomed-in area, while near $\bm{p}_{a}$ it looks linear only near the point itself even though the zoom factor is higher.}
  \label{fig:zoom_in}
\end{figure}

How can we quantify this idea? Let us consider some real function $f$ and a point $\bm{p}_{0} = \left(x_{0},f\left(x_{0}\right)\right)$ on the function. We can then define another point to the right of $x_{0}$: $\bm{p}_{1}=\left(x_{1},f\left(x_{1}\right)\right)$. Since $x_{1}$ is to the right of $x_{0}$ we can write it as $x_{1}=x_{0}+\Delta x$, where $\Delta x>0$. We then connect the two points with a line (\autoref{fig:x0_x1_line}). The slope of this line can then be calculated using \autoref{eq:linear_slope_def}:
\begin{equation}
  m = \frac{\Delta y}{\Delta x} = \frac{f\left(x_{1}\right)-f\left(x_{0}\right)}{x_{1}-x_{0}} = \frac{f\left(x_{0}+\Delta x\right)-f\left(x_{0}\right)}{x_{0}+\Delta x - x_{0}} = \frac{f\left(x_{0}+\Delta x\right)-f\left(x_{0}\right)}{\Delta x}.
  \label{eq:derivative_slope_def}
\end{equation}

\begin{figure}
  \centering
  \begin{tikzpicture}[node distance=13pt]
    \pgfmathsetmacro{\R}{3}
    \pgfmathsetmacro{\c}{1.5}
    \pgfmathsetmacro{\xa}{1.5}
    \pgfmathsetmacro{\xb}{3.5}
    \begin{axis}[
      graph2d,
      width=9cm, height=9cm,
      xmin=0, xmax=4,
      ymin=0, ymax=4,
      xticklabels={,},
      yticklabels={,},
      grid=major,
      major grid style={draw=black!2},
      extra x tick style={
        tick label style={xblue},
        major grid style=black!0,
      },
      extra x ticks={\xa, \xb},
      extra x tick labels={$x_{0}$, $x_{1}$},
      extra y tick style={
        tick label style={xblue},
        major grid style=black!0,
      },
      extra y ticks={{fd(\xa)}, {fd(\xb)}},
      extra y tick labels={$f\left(x_{0}\right)$, $f\left(x_{1}\right)$},
      axis line style={-stealth},
      domain={0:4},
      samples=100,
      declare function={
        fd(\x)=sqrt(\R^2-(\x-\c)^2)-1;
        Dfd(\x)=-(\x-\c)/sqrt(\R^2-(\x-\c)^2);
      },
    ]
    % Coordinate lines lines
    \tikzset{prp/.style={perp, black!20}}
    \draw[prp] (\xa,0) -- (\xa,{fd(\xa});
    \draw[prp] (\xb,0) -- (\xb,{fd(\xb});
    \draw[prp] (0,{fd(\xa)}) -- (\xa,{fd(\xa});
    \draw[prp] (0,{fd(\xb)}) -- (\xb,{fd(\xb});
    % Derivative slope at x=\c
    \pgfmathsetmacro{\m}{(fd(\xb)-fd(\xa))/(\xb-\xa)}
    % Line between points and its slope
    \draw[xred, very thick] (\xa, {fd(\xa)}) -- (\xb,{fd(\xb)});
    % \addplot[function, xred, dashed] {\m*(x-\xa)+fd(\xa)};
    % % Derivatie at x0
    % \pgfmathsetmacro{\M}{Dfd(\xa)}
    % \addplot[function, xgreen, dashed] {\M*(x-\xa)+fd(\xa)};
    % Function + points on it
    \addplot[function, xblue] {fd(x)};
    \addplot[only marks, mark=*] coordinates {
        (\xa,{fd(\xa)})
        (\xb,{fd(\xb)})
      };
    % Labels
    \coordinate (p0) at (\xa,{fd(\xa)});
    \coordinate (p1) at (\xb,{fd(\xb)});
    \node[above of=p0] (p0lbl) {$\bm{p}_{0}$};
    \node[above of=p1] (p1lbl) {$\bm{p}_{1}$};
    \end{axis}
  \end{tikzpicture}
  \caption{Text.}
  \label{fig:x0_x1_line}
\end{figure}

We can then take the limit of \autoref{eq:derivative_slope_def} as $\Delta x\to 0$ (\autoref{fig:derivate_limit}):
\begin{equation}
  M = \lim\limits_{\Delta x\to 0} \frac{f\left(x_{0}+\Delta x\right)-f\left(x_{0}\right)}{\Delta x}.
  \label{eq:derivative_def}
\end{equation}
This limit is defined as \emph{the derivative} of $f$ at the point $x=x_{0}$, and it tells us, quantitatively, how $f$ locally behaves at $x_{0}$, i.e. how much does it increase, decrease or stay the same around $x_{0}$.

\newcommand{\derivFig}[1]{
  \begin{tikzpicture}[node distance=10pt]
    \pgfmathsetmacro{\R}{3}
    \pgfmathsetmacro{\c}{1.5}
    \pgfmathsetmacro{\xa}{1.5}
    \pgfmathsetmacro{\xb}{\xa+#1}
    \begin{axis}[
      graph2d,
      width=6cm, height=6cm,
      xmin=0, xmax=4,
      ymin=-0.5, ymax=3.5,
      xticklabels={,},
      yticklabels={,},
      extra x tick style={
        tick label style={xblue},
        major grid style=black!0,
      },
      extra x ticks={\xa, \xb},
      extra x tick labels={},
      grid=major,
      major grid style={draw=black!2},
      axis line style={-stealth},
      domain={0:4},
      samples=40,
      declare function={
        fd(\x)=sqrt(\R^2-(\x-\c)^2)-1;
        Dfd(\x)=-(\x-\c)/sqrt(\R^2-(\x-\c)^2);
      },
    ]
    % Derivative slope at x=\c
    \pgfmathsetmacro{\m}{(fd(\xb)-fd(\xa))/(\xb-\xa)}
    % Brace
    \draw[thick, decorate, decoration={brace, amplitude=3pt, raise=3pt, mirror}]
          (\xa,0) -- (\xb,0) node[midway, below, yshift=-5pt]{$\Delta x$};
    \draw[perp] (\xa,0) -- (\xa,{fd(\xa)});
    \draw[perp] (\xb,0) -- (\xb,{fd(\xb)});
    % Line between points and its slope
    \addplot[function, xred] {\m*(x-\xa)+fd(\xa)};
    % Derivatie at x0
    \pgfmathsetmacro{\M}{Dfd(\xa)}
    \addplot[function, xgreen] {\M*(x-\xa)+fd(\xa)};
    % Function + points on it
    \addplot[function, xblue] {fd(x)};
    \addplot[only marks, mark=*] coordinates {
        (\xa,{fd(\xa)})
        (\xb,{fd(\xb)})
      };
    % Labels
    \coordinate (p0) at (\xa,{fd(\xa)});
    \coordinate (p1) at (\xb,{fd(\xb)});
    \node[above of=p0] (p0lbl) {$\bm{p}_{0}$};
    \node[below of=p1] (p1lbl) {$\bm{p}_{1}$};
    \end{axis}
  \end{tikzpicture}
}

\begin{figure}
  \centering
  \begin{subfigure}[b]{0.45\textwidth}
    \centering
    \derivFig{2.4}
  \end{subfigure}
  \hfill
  \begin{subfigure}[b]{0.45\textwidth}
    \centering
    \derivFig{2.0}
  \end{subfigure}
  \hfill
  \begin{subfigure}[b]{0.45\textwidth}
    \centering
    \derivFig{1.5}
  \end{subfigure}
  \hfill
  \begin{subfigure}[b]{0.45\textwidth}
    \centering
    \derivFig{1}
  \end{subfigure}
  \hfill
  \begin{subfigure}[b]{0.45\textwidth}
    \centering
    \derivFig{0.3}
  \end{subfigure}
  \hfill
  \begin{subfigure}[b]{0.45\textwidth}
    \centering
    \derivFig{0.1}
  \end{subfigure}
  \hfill
  \caption{As $\Delta x\to 0$, $\bm{p}_{1}$ approaches $\bm{p}_{0}$ and the slope of the red line connecting the two points approches the slope $M$ of the green line at $\bm{p}_{0}$.}
  \label{fig:derivate_limit}
\end{figure}

\begin{example}{Validation of the derivative using a linear function}{linear_derivative}
  Given a linear function $f(x)=mx+b$, we expect that the derivative of $f$ at any point $x_{0}$ would equal $m$, since the entire function is a line connecting all the points on the function itself. Let us check that:
  \begin{align*}
    M &= \lim\limits_{\Delta x\to 0}\frac{f\left(x_{0}+\Delta x\right)-f\left(x_{0}\right)}{\Delta x}\\
      &= \lim\limits_{\Delta x\to 0}\frac{m\left(x_{0}+\Delta x\right)+\cancel{b}-\left(mx_{0}+\cancel{b}\right)}{\Delta x}\\
      &= \lim\limits_{\Delta x\to 0}\frac{\cancel{mx_{0}}+m\Delta x-\cancel{mx_{0}}}{\Delta x}\\
      &= \lim\limits_{\Delta x\to 0}\frac{m\cancel{\Delta x}}{\cancel{\Delta x}}\\
      &= m.
  \end{align*}
\end{example}

\begin{example}{Derivative of $x^{2}$}{derivative_x^2}
  Unlike for a linear function, we shouldn't expect the derivate of $f(x)=x^{2}$ to be constant at any point. However, we can easily calculate what the derivative would be at some point $x_{0}$:
  \begin{align*}
    M &= \lim\limits_{\Delta x\to 0}\frac{\left(x_{0}+\Delta x\right)^{2}-x_{0}^{2}}{\Delta x}\\
      &= \lim\limits_{\Delta x\to 0}\frac{\cancel{x_{0}^{2}}+2x_{0}\Delta x-\left(\Delta x\right)^{2}-\cancel{x_{0}^{2}}}{\Delta x}\\
      &= \lim\limits_{\Delta x\to 0}\frac{\cancel{\Delta x}\left(2x_{0}-\Delta x\right)}{\cancel{\Delta x}}\\
      &= \lim\limits_{\Delta x\to 0}2x_{0}-\Delta x\\
      &= 2x_{0}.
  \end{align*}
  I.e. we see that any point $x_{0}$ the derivative of $f(x)=x^{2}$ is simply $2x_{0}$.

  For example, at $x_{0}=3$ the derivative is $M=6$, and at $x_{0}=0$ the derivative is $M=0$.
\end{example}

Up until now we have regarded the derivative as a property of some point on some function $f$. However, since we can calculate the derivative at each point of the function\footnote{except for some points which we will discuss later.}, we can collect all these points together to form a new function, which we call the \emph{derivative} of $f$ and denote as $f'$ (read: ``$f$-prime'').

In \autoref{example:linear_derivative} we saw that the derivative of a linear function at any point gives the same value $m$ (namely the slope of the linear function). Therefore, this derivative is itself a \textit{constant} function $f'(x)=m$. When we calculated the derivative of $f(x)=x^{2}$ (\autoref{example:derivative_x^2}), we found that it depends on the point where it was calculated, using the relation $f'(x)=2x$, which is a linear function with slope $2$ that goes through the origin.

Let us now calculate the derivative of some common functions.
\begin{example}{Derivative of $ax^{n}$}{derivative_ax^n}
  The derivative of the function $f(x)=ax^{n}$ (where $a\in\Rs$ is a constant) is (recall the binomial expansion, \autoref{eq:binomial_expansion_n}):
  \begin{align*}
    f'(x) &= \lim\limits_{\Delta x\to 0}\frac{a(x+\Delta x)^{n}-ax^{n}}{\Delta x}\\
          &= \lim\limits_{\Delta x\to 0}\frac{a\left[\cancel{x^{n}} + nx^{n-1}\Delta x + \binom{2}{n}x^{n-2}\left(\Delta x\right)^{2} + \dots + nx\left(\Delta x\right)^{n-1} + \left(\Delta x\right)^{n}\right]-\cancel{ax^{n}}}{\Delta x}.
  \end{align*}
  We can take $\Delta x$ out of the numerator and cancel it out with the $\Delta x$ in the denominator:
  \begin{align*}
    f'(x) &= \lim\limits_{\Delta x\to 0}\frac{a\cancel{\Delta x}\left[ nx^{n-1} + \binom{2}{n}x^{n-2}\Delta x + \dots + nx\left(\Delta x\right)^{n-2} + \left(\Delta x\right)^{n-1} \right]}{\cancel{\Delta x}}\\
          &= \lim\limits_{\Delta x\to 0}a\left[ nx^{n-1} + \binom{2}{n}x^{n-2}\Delta x + \dots + nx\left(\Delta x\right)^{n-2} + \left(\Delta x\right)^{n-1} \right].
  \end{align*}
  Since all expressions except $nx^{n-1}$ have some power of $\Delta x$ in them, in the limit $\Delta x\to 0$ they all vanish, leaving us with
  \[
    f'(x) = anx^{n-1}.
  \]

  This derivative is commonly described as the power of $x$ being reduced by $1$ and the expression gaining a factor of $n$ (i.e. the power before reducing it).
\end{example}

\begin{example}{Derivative of $a^{x}$}{}
  \tbw{Note}:

  Use the fact that $\lim\limits_{h\to 0}\frac{a^{h}-1}{h}=\log(a)$. Therefore we also get that $\lim\limits_{h\to 0}\frac{\eu^{h}-1}{h}=\log(\eu)=1$.
\end{example}

\begin{example}{Derivative of $\sin(x)$}{}
  Calculating the derivative of $\sin(x)$:
  \begin{align*}
    \sin'(x) &= \lim\limits_{\Delta x\to 0}\frac{\sin\left(x+\Delta x\right)-\sin(x)}{\Delta x}\\
             &= \lim\limits_{\Delta x\to 0}\frac{\sin(x)\cos\left(\Delta x\right)+\cos(x)\sin\left(\Delta x\right)-\sin(x)}{\Delta x}.
  \end{align*}
  We can separate the three terms into three limits:
  \[
    \sin'(x) = \lim\limits_{\Delta x\to 0}\frac{\sin(x)\left[\cos\left(\Delta x\right)-1\right]}{\Delta x} + \lim\limits_{\Delta x\to 0}\frac{\cos(x)\sin\left(\Delta x\right)}{\Delta x}.
  \]
  The second limit equals $\cos(x)$, since $\lim\limits_{c\to 0}\frac{\sin(c)}{c}=1$ (\autoref{eq:XXX}). Since $\sin(x)$ does not change as we decrease $\Delta x$, we can regard it as a constant and take it out of the limit:
  \[
    \sin'(x) = \sin(x)\lim\limits_{\Delta x\to 0}\frac{\cos\left(\Delta x\right)-1}{\Delta x} + \cos(x).
  \]
  Using the double angle identity (\autoref{eq:double_angle}) on $\cos\left(\Delta x\right)$ we get that
  \[
    \cos\left(\Delta x\right) = 1-2\sin^{2}\left(\frac{\Delta x}{2}\right),
  \]
  and by plugging this back into the derivative calculation we get:
  \begin{align*}
    \sin'(x) &= \lim\limits_{\Delta x\to 0}\frac{-2\sin^{2}\left(\frac{\Delta x}{2}\right)}{\Delta x} + \cos(x)\\
             &= \lim\limits_{\Delta x\to 0}\frac{-\cancel{\frac{2}{2}}\sin^{2}\left(\frac{\Delta x}{2}\right)}{\frac{\Delta x}{2}} + \cos(x)\\
             &= \cos(x),
  \end{align*}
  since $\lim\limits_{h\to0}\frac{\sin^{2}(h)}{h}=0$ (\autoref{eq:XXX}).

  \tbw{in the limit section show and prove this limit}
\end{example}

\begin{example}{Derivatie of $\sqrt{x}$}{}
  The derivative of the function $f(x)=\sqrt{x}$ is:
  \[
    f'(x) = \lim\limits_{\Delta x\to 0}\frac{\sqrt{x+\Delta x}-\sqrt{x}}{\Delta x}.
  \]
  We can multiply the numerator and denominator each by $\sqrt{x+\Delta x}+\sqrt{x}$. This would allow us to use the relation $(a-b)(a+b)=a^{2}-b^{2}$:
  \begin{align*}
    f'(x) &= \lim\limits_{\Delta x\to 0}\frac{\sqrt{x+\Delta x}-\sqrt{x}}{\Delta x}\\
          &= \lim\limits_{\Delta x\to 0}\frac{\cancelcol[xred]{x}+\cancelcol[xblue]{\Delta x}-\cancelcol[xred]{x}}{\cancelcol[xblue]{\Delta x}\left(\sqrt{x+\Delta x}+\sqrt{x}\right)}\\
          &= \frac{1}{\sqrt{x}+\sqrt{x}}\\
          &= \frac{1}{2\sqrt{x}}.
  \end{align*}
\end{example}
