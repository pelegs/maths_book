\chapter{Introduction}
In this chapter we will introduce some key concepts that will be used in later chapters.

\begin{note}{In case you are already familiar with the topics}{}
	It is recommended for readers who are familiar with the topics to at least gloss over this chapter and make sure they know and understand all the concept presented here.
\end{note}

\section{Trigonometry}
Consider a \emph{right triangle} with sides $a,b,c$, an angle $\gamma=\ang{90}$ and two other angles $\alpha,\beta$:

\centering
\begin{tikzpicture}
		\coordinate (A) at (0,0);
		\coordinate (B) at (4,3);
		\coordinate (C) at (4,0);

		\draw[fill=xblue!30] (A) -- node (c) [midway, above, rotate=36.87] {$c$ (Hypotenous)} (B) -- node (a) [midway, right] {$a$ (Opposite)} (C) -- node (b) [midway, below] {$b$ (Adjacent)} cycle;
		\draw[thick] ($(C)+(0,0.3)$) rectangle ($(C)-(0.3,0)$);
		\draw[thick, xpurple!50!black, fill=xpurple!45] (A) -- ($(A)+(0.8,0)$) arc (0:36.87:0.8) node [midway, xshift=-2mm, yshift=-2pt] {$\alpha$} -- cycle;
		\draw[thick, xblue!50!black, fill=xblue!45] (B) -- ($(B)+(0,-0.8)$) arc (270:216.97:0.8) node [midway, above, xshift=5pt] {$\beta$} -- cycle;
		\draw[thick] (A) -- (B) -- (C) -- cycle;
	\end{tikzpicture}
\flushleft

We use the ratios between the three sides of the triangle to define three functions of $\alpha$:
\begin{definition}{The basic triginometric functions}{}
	\vspace{5mm}
	\begin{enumerate}
		\item The \emph{sine} of the angle $\alpha$ is $\sin(\alpha)=\frac{a}{c}$,
		\item the \emph{cosine} of the angle $\alpha$ is $\cos(\alpha)=\frac{b}{c}$, and
		\item the \emph{tangent} of the angle $\alpha$ is $\tan(\alpha)=\frac{a}{b}$, which in turn is equal to $\frac{\sin(\alpha)}{\cos(\alpha)}$.
		\end{enumerate}
	\label{def:basic_trig}
\end{definition}

Due to the \emph{triangle inequality}, $a$ and $b$ must each be less or equal to $c$, which means that $\sin(\alpha)=\frac{a}{c}$ must be less or equal to $1$, and the same is true for $\cos(\alpha)=\frac{b}{c}$.

\begin{note}{}{}
	Since $\gamma=\ang{90}$, the two angles $\alpha$ and $\beta$ each must be in the range of $\ang{0}$ to $\ang{90}$, and $\alpha+\beta=\ang{90}$ (since $\alpha+\beta+\gamma=\ang{180}$). This means that $\sin(\alpha)=\sin(\ang{90}-\beta)$. WRITE ABOUT SIN(A)=COS(B), COS(A)=SIN(B).
\end{note}

\section{Sets}
