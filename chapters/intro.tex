\chapter{Introduction}
In this chapter we will introduce some key concepts that will be used in later chapters.

\begin{note}{In case you are already familiar with the topics}{}
	It is recommended for readers who are familiar with the topics to at least gloss over this chapter and make sure they know and understand all the concept presented here.
\end{note}

\section{Trigonometry}
Consider a \emph{right triangle} with sides $a,b,c$ and angles $\alpha, \beta, \gamma=\ang{90}$ (\figref{righttriangle}). We use the ratios between the three sides of the triangle to define three functions of $\alpha$:
\begin{definition}{The basic triginometric functions}{}

	\vspace{5mm}
	\begin{itemize}
		\item The \emph{sine} of the angle $\alpha$ is $\sin(\alpha)=\frac{a}{c}$.
		\item The \emph{cosine} of the angle $\alpha$ is $\cos(\alpha)=\frac{b}{c}$.
		\item The \emph{tangent} of the angle $\alpha$ is $\tan(\alpha)=\frac{b}{a}$.
		\end{itemize}
	\label{def:basic_trig}

\begin{figure}[H]
	\centering
	\begin{tikzpicture}
		\coordinate (A) at (0,0);
		\coordinate (B) at (4,3);
		\coordinate (C) at (4,0);

		\draw[fill=xblue!30] (A) -- node (c) [midway, above, rotate=36.87] {Hypotenous, $c$} (B) -- node (a) [midway, above, rotate=-90] {Opposite, $a$} (C) -- node (b) [midway, below] {Adjacent, $b$} cycle;
		\draw[thick] ($(C)+(0,0.3)$) rectangle ($(C)-(0.3,0)$);
		\draw[thick, xpurple!50!black, fill=xpurple!45] (A) -- ($(A)+(0.8,0)$) arc (0:36.87:0.8) node [midway, xshift=-2mm, yshift=-2pt] {$\alpha$} -- cycle;
		\draw[thick, xblue!50!black, fill=xblue!45] (B) -- ($(B)+(0,-0.8)$) arc (270:216.97:0.8) node [midway, above, xshift=4pt] {$\beta$} -- cycle;
		\draw[thick] (A) -- (B) -- (C) -- cycle;
	\end{tikzpicture}
	\caption{A right right triangle with sides $a,b,c$ and angles $\alpha,\beta,\gamma=\ang{90}$.}
	\label{fig:righttriangle}
\end{figure}
\end{definition}
\section{Sets}
