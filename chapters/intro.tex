\chapter{Introduction}
In this chapter we will introduce some key concepts that will be used in later chapters.

\begin{note}{In case you are already familiar with the topics}{}
	It is recommended for readers who are familiar with the topics to at least gloss over this chapter and make sure they know and understand all the concept presented here.
\end{note}

\section{Mathematical Symbols and Sets}
\subsection{Logical Statements and their Truth Value}
We start our discussion with the simplest mathematical concept: a \emph{proposition}. A proposition is simply a statement that might be either \true\ or \false.
\begin{example}{Truth of propositions}{}
	\begin{itemize}
		\item $3>1$ (\true)
		\item $-2=5-7$ (\true)
		\item $7<5$ (\false)
		\item The radius of the earth is bigger than that of the moon. (\true)
		\item The word `House' starts with the letter `G'. (\false)
	\end{itemize}
\end{example}

We can group together propositions using \emph{logical operators}. Two of the most common logical operators are \emph{AND} and \emph{OR}.

The \AND{} operator returns a \true\ statement only if \textbf{both} the statements it groups are themselves \true, otherwise it returns \false.

\begin{example}{The AND operator}{}
	\begin{itemize}
		\item $2+4=6$ is \true, $4-2=2$ is \true. $\left(2+4=6\ \text{\AND{}}\ 4-2=2\right)$ is therefore \true.
		\item $2+4=6$ is \true, $2>6$ is \false. $\left(2+4=6\ \text{\AND{}}\ 2>6\right)$ is therefore \false.
		\item $\frac{10}{2}=1$ is \false, $2^{4}=16$ is \true. $\left( \frac{10}{2}=1\ \text{\AND{}}\ 2^{4}=16 \right)$ is therefore \false.
		\item $7<5$ is \false, $10+2=13$ is \false. $\left( 7<5\ \text{\AND{}}\ 10+2=13 \right)$ is therefore \false.
	\end{itemize}
\end{example}

The \OR{} operator returns \true\ if \textbf{at least} one of the statements it groups is true.
\begin{example}{The OR operator}{}
	\begin{itemize}
		\item $2+4=6$ is \true, $4-2=2$ is \true. $\left(2+4=6\ \text{\OR{}}\ 4-2=2\right)$ is therefore \true.
		\item $2+4=6$ is \true, $2>6$ is \false. $\left(2+4=6\ \text{\OR{}}\ 2>6\right)$ is therefore \true.
		\item $\frac{10}{2}=1$ is \false, $2^{4}=16$ is \true. $\left( \frac{10}{2}=1\ \text{\OR{}}\ 2^{4}=16 \right)$ is therefore \true.
		\item $7<5$ is \false, $10+2=13$ is \false. $\left( 7<5\ \text{\OR{}}\ 10+2=13 \right)$ is therefore \false.
	\end{itemize}
\end{example}

The behaviour of both operators can be summarized using a \emph{truth table} (see \tabref{AND_OR_truth_table} below).
\begin{table}[H]
	\centering
	\caption{The truth table for the operators \AND{} and \OR{}.}
	\label{tab:AND_OR_truth_table}
	\begin{tabular}{llll}
		\toprule
		$A$ & $B$ & $A$ AND $B$ & $A$ OR $B$\\
		\midrule
		\true & \true & \true & \true \\
		\true & \false & \false & \true \\
		\false & \true & \false & \true \\
		\false & \false & \false & \false \\
		\midrule
	\end{tabular}
\end{table}

When writing, it is convinient to use \emph{notations} to represent operators: the \AND{} operator is denoted by $\opand$, while the \OR{} operator is denoted by $\opor$.

\begin{example}{Using the notations for \AND{} and \OR{}}{}
	\begin{align*}
		(\falseprop{2+2=5}) &\opand (\trueprop{1-1=0}) \Rightarrow \false\\
		(\falseprop{2+2=5}) &\opor  (\trueprop{1-1=0}) \Rightarrow \true
	\end{align*}
\end{example}

Several more common mathematical notations are given in \tabref{common_math_notations}.
  
\begin{table}[H]
	\centering
	\caption{Common Mathematical NotationsUsed in this Book.}
	\label{tab:common_math_notations}
	\begin{tabular}{ll}
		\toprule
		Symbol & In words\\
		\midrule
		$\neg a$ & \textbf{not} $a$\\
		$a \opand b$ & $a$ \textbf{and} $b$\\
		$a \opor b$ & $a$ \textbf{or} $b$\\
		$a \Rightarrow b$ & $a$ \textbf{implies} $b$\\
		$a \Leftrightarrow b$ & $a$ \textbf{is equivalent to} $b$\\
		$\forall x$ & \textbf{For all} $x$ (...)\\
		$\exists x$ & \textbf{There exists} $x$ \textbf{such that} (...)\\
		$a\defeq b$ & $a$ \textbf{is defined to be} $b$\\
		\midrule
	\end{tabular}
\end{table}

The notation $\Rightarrow$ need a bit of clarification: implication means that we can directly derive a proposition from another proposition. For example, if $x=3$ then $x>2$. The opposite implication can be a \false{} statemt, i.e. for the example above $x>2$ does not imply $x=3$ (denoted as $x>2 \nRightarrow x=3$). Sometimes implication is expressed by using the word \textit{if}: in the above example $x>2$ if $x=3$, but the other way around is not \true{}.

We say that two propositions are \emph{equivalent} when they imply each other. For example: $x=2$ implies that $\frac{x}{2}=1$, while $\frac{x}{2}=1$ implies that $x=2$. We can write this as
\[
	\frac{x}{2}=1 \Leftrightarrow x=2.
\]
Instead of the word \textit{equivalent}, the phrase \textit{if and only if} (sometimes shortened to \emph{iff}) is commonly used, e.g.
\[
	x=2\ \text{iff}\ \frac{x}{2}=1.
\]

\subsection{Sets}
The concept of \emph{sets} is perhaps one of the most basic ideas in modern mathematics. Much of the material convered in this book will be built upon sets and their properties. However, as with the rest of the material presented here - our description of sets will not be thorough nor percise.

For our purposes, a set is a collection of \emph{elements}. These elements can be any concept - be it physical (a chair, a bicycle, a tapir) or abstract (a number, an idea). However, we will consider only sets comprised of numbers. Sets can have finite of infinite number of elements in them.

We denote sets by using curly brackets, and if the number of elements in them is not too big - we display the elements, separated by commas, inside the brackets. In other cases we can express the sets as a sentence or a mathematical proposition.

\begin{example}{Simple sets}{}
	\[
		\left\{ 1,2,3,4 \right\}\qquad\left\{ -4,\frac{3}{7},0,\pi,0.13,-2.5,\frac{e}{3},2^{-\pi} \right\}\qquad\left\{ \text{all even numbers} \right\}
	\]
\end{example}

Sets have two important properties:
\begin{enumerate}
	\item Elements in a set do not repeat.
	\item The order of elements in a set does not matter.
\end{enumerate}

\begin{example}{Important set properties}{}
	Examples demonstrating the two afforementioned important properties of sets:
	\begin{enumerate}
		\item The following is not a proper set:
		\[
			\left\{ 1,1,0,1,0,0,-1,0,0,-1,-1,1 \right\}
	\]

		\item The following sets are all identical:
		\[
			\left\{ 1,2,3,4 \right\}\qquad\left\{ 1,3,2,4  \right\}\qquad\left\{ 3,4,1,2 \right\}\qquad\left\{ 1,3,2,4 \right\}\qquad\left\{ 4,3,2,1 \right\}
		\]
	\end{enumerate}
	
\end{example}

Sets can be denoted using \emph{conditions}, with the symbol $|$ representing the phrase "such that".

\begin{example}{Defining a set using a condition}
	The following set contains all the odd whole numbers between $0$ and $10$, including both:
	\[
		\left\{ 0 < x < 10 \mid x\ \text{is an odd number}\right\}.
	\]
	The definition of this set can be read as

	\vspace{3mm}
	\centering
	\textit{all numbers $x$ that are bigger than $0$ and are smaller than $10$, such that $x$ is odd.}

	\flushleft{}
	(note that the requirement of $x$ to be an odd number means that it is neccessarily a whole number as well)

	\vspace{1em}
	This set can be written explicitly as
	\[
		\left\{ 1,3,5,7,9 \right\}.
	\]
\end{example}

Sets are usually denoted with an uppercase latin letter ($A,B,C,\dots$), while their elements are denoted as lowercase letters ($a,b,\alpha,\phi,\dots$). When we want to denote that an element belongs to a set we use the following symbol: $\in$. Conversly, $\notin$ is used to denote that an element \textit{does not} belong to a set.
	
\begin{example}{Elements in sets}{}
	For the two sets
	\[
		A = \left\{ 1,2,5,7 \right\},\quad B=\left\{ \text{even numbers} \right\},
	\]
	all the following propostions are \true{}:
	\begin{align*}
		&1\in A,\quad 2\in A,\quad 5\in A,\quad 7\in A,\\
		&2\in B,\quad 1\notin B,\quad 5\notin B,\quad 7\notin B.
	\end{align*}
\end{example}

The number of elements in a set, also called its \emph{cardinality} is denoted using two vertical bars (similar to the way absolute values are denoted).

\begin{example}{Cardinality}{}
	For $S=\left\{ -3,0,-2,7,1,\frac{1}{2},5 \right\},\ |S|=7$.
\end{example}

An important special set is the \emph{empty set}, which is the set containing no elements. It is denoted by $\emptyset$, and has the unique property that $|\emptyset|=0$.
(note that there is only one such set)

Two sets are equal if they both contain the exact same elemnts and only these elements, i.e.
\[
	A = B \Longleftrightarrow x\in A \Leftrightarrow x\in B.
\]
This proposition reads `The sets $A$ and $B$ are equal \underline{\textit{if and only if}} any element $x$ in $A$ is also in $B$, and any element $x$ in $B$ is also in $A$'.

When all the elements of a set $B$ are also elements of another set $A$, we say that $B$ is a \emph{subset} of $A$, and we denote that as $B\subset A$. A very useful way of illustrating the relationship beyween two or more sets is by using \emph{Venn diagrams}.

\begin{example}{Subsets and Venn diagrams}{}
	A Venn diagram depicting the set $\textcolor{xblue}{B=\left\{ 0,2 \right\}}$ as a subset of $\textcolor{xred}{A=\left\{ 0,1,2,3,4,\dots,9 \right\}}$:
	\begin{figure}[H]
		\centering
		\begin{tikzpicture}
			\def\firstcircle{(0,0) circle (2)}
			\def\secondcircle{(0.3,0.8) circle (1)}
			\fill[xred!50]\firstcircle;
			\fill[xblue!50]\secondcircle;
			\Large
			\draw \firstcircle node[below left, xshift=-18mm, yshift=11mm] (A) {};
			\draw \secondcircle node[right of=A, xshift=7mm] (B) {};

			% Elements
			\small
			\tikzset{
				node distance={5mm},
			}
			\node at (0.57,1.02) {0};
			\node at (-1.04,-1.25) {1};
			\node at (0.16,0.4) {2};
			\node at (-0.1,-0.82) {3};
			\node at (-0.92,-0.49) {4};
			\node at (-1.36,0.06) {5};
			\node at (1.13,-1.16) {6};
			\node at (1.50,0.62) {7};
			\node at (-1.08,1.3) {8};
			\node at (1.32,-0.79) {9};
		\end{tikzpicture}
	\end{figure}
\end{example}

If for two sets $A,B$ both $A\subset B$ and $B\subset A$, then $A=B$. We can write this fact as a methematical proposition:
\begin{equation}
	(A\subset B) \opand (B\subset A) \Leftrightarrow A=B.
	\label{eq:subset_equal}
\end{equation}

The \emph{intersection} of two sets $A$ and $B$, denoted $A\cup B$, is the set of all elements $x$ such that $x\in A$ \AND{} $x\in B$.

\begin{example}{Intersection of sets}{}
	The intersection of the sets $A=\left\{ 1,2,3,4 \right\}$ and $B=\left\{ 3,4,5,6 \right\}$ is the set $A\cup B=\left\{ 3,4 \right\}$.

	The intersection of the sets $C=\left\{ 0,1,2,6,7 \right\}$ and $D=\left\{ 3,9,-4,5 \right\}$ is the empty set $\emptyset$, since no element is in both sets.
\end{example}
  
The following Venn diagram depicts the intersection of two sets:
\begin{figure}[H]
    \centering
    \begin{tikzpicture}
      \def\firstcircle{(0,0) circle (2)}
      \def\secondcircle{(2.3,0) circle (1.5)}
      \fill[xred!50]\firstcircle;
      \fill[xblue!50]\secondcircle;
      \begin{scope}
        \clip \firstcircle;
        \fill[xgreen!50]\secondcircle;
      \end{scope}
      \draw\firstcircle node[left] {$A$};
      \draw\secondcircle node[right] {$B$};
      \draw (1.4,-0.2) node[above] {$A\cap B$};
    \end{tikzpicture}
  \end{figure}

\Blindtext{}

\section{Relations and Functions}
\section{Trigonometric functions}
\subsection{Basic Definitions}
Consider a \emph{right triangle} $\triangle ABC$ with sides $a,b$, and Hypotenous $c$, where the angle $\angle ACB$ is $\ang{90}$, and the angle $\angle BAC$ is denoted as $\alpha$:

\centering
\begin{tikzpicture}[node distance=3mm]
		\coordinate (A) at (0,0);
		\coordinate (B) at (4,3);
		\coordinate (C) at (4,0);

		\node[left of=A] {$A$};
		\node[above of=B] {$B$};
		\node[right of=C] {$C$};

		\draw[fill=xblue!30] (A) -- node (c) [midway, above, rotate=36.87] {$c$ (Hypotenous)} (B) -- node (a) [midway, right] {$a$ (Opposite)} (C) -- node (b) [midway, below] {$b$ (Adjacent)} cycle;
		\draw[thick] ($(C)+(0,0.3)$) rectangle ($(C)-(0.3,0)$);
		\draw[thick, xpurple!50!black, fill=xpurple!45] (A) -- ($(A)+(1,0)$) arc (0:36.87:1) node [midway, xshift=-3mm, yshift=-2pt] {$\alpha$} -- cycle;
		%\draw[thick, xblue!50!black, fill=xblue!45] (B) -- ($(B)+(0,-0.8)$) arc (270:216.97:0.8) node [midway, above, xshift=5pt] {$\beta$} -- cycle;
		\draw[thick] (A) -- (B) -- (C) -- cycle;
	\end{tikzpicture}
\flushleft

We use the ratios between the three sides of the triangle to define three functions of $\alpha$:
\begin{definition}{The basic triginometric functions}{}
	\vspace{5mm}
	\begin{enumerate}
		\item The \emph{sine} of the angle $\alpha$ is $\sin(\alpha)=\frac{a}{c}$,
		\item the \emph{cosine} of the angle $\alpha$ is $\cos(\alpha)=\frac{b}{c}$, and
		\item the \emph{tangent} of the angle $\alpha$ is $\tan(\alpha)=\frac{a}{b}$, which in turn is equal to $\frac{\sin(\alpha)}{\cos(\alpha)}$.
		\end{enumerate}
	\label{def:basic_trig}
\end{definition}

We can rearrange the above definitions to yield
\begin{align}
	a &= c\sin(\alpha),\nonumber\\
	b &= c\cos(\alpha).
	\label{eq:basic_trig_rearrange}
\end{align}

Normaly, the Hypotenous is the longest side of a right triangle. We will consider here the two edge cases where one of the sides $a,b$ is equal to the Hypotenous (and the other side is thus $0$):
\begin{itemize}
	\item if $a=c$ then $\alpha=\ang{90}$,\
	\item if $b=c$ then $\alpha=0$.
\end{itemize}

The posssible length of $a$ is therefore in the range $0\leq a \leq c$, which means that $0\leq \frac{a}{c} \leq 1$, or since $\sin(\alpha)=\frac{a}{c}$,
\begin{equation}
	0\leq \sin(\alpha) \leq1.
	\label{eq:img_sin}
\end{equation}

The same is of course true for $b$, and thus
\begin{equation}
	0\leq \cos(\alpha) \leq1
	\label{eq:img_cos}
\end{equation}
as well.

As a reminder, the \emph{Pythagorean theorem}\footnote{It's worth mentioning that no three positive integers $a, b$, and $c$ satisfy the equation $a^{n}+b^{n}=c^{n}$ for any integer value of $n>2$. \href{https://en.wikipedia.org/wiki/Fermat\%27s_Last_Theorem}{This can be proven, however the proof is too large to fit in the footnotes}.} states that for a right triangle like the one here,
\begin{equation}
	a^{2} + b^{2} = c^{2}.
	\label{eq:pythagorean_theorem}
\end{equation}
By substituting \xref[eq]{basic_trig_rearrange} into the above we get
\begin{equation*}
	c^{2} = a^{2}+b^{2} = \left( c\sin(\alpha) \right)^{2} + \left( c\cos(\alpha) \right)^{2} = c^{2}\sin^{2}(\alpha) + c^{2}\cos^{2}(\alpha) = c^{2}\left[ \sin^{2}(\alpha) + \cos^{2}(\alpha) \right],
\end{equation*}
and cancelling $c^{2}$ on both sides simply yields
\begin{equation}
	\sin^{2}(\alpha) + \cos^{2}(\alpha) = 1.
	\label{eq:sin2_cos2_1}
\end{equation}

\subsection{The Unit Circle}
The range of the trigonometric functions can be extended by using the \emph{unit circle}: a circle of radius $R=1$ is placed such that its center lies at the origin of a 2-dimensional axis system, i.e. at the point $\bm{O}=(0,0)$. A radius to a point $\bm{P}=(x,y)$ on the circle's circumference is the drawn. This radius has an angle $\theta$ to the $x$-axis. A line from $\bm{P}$ perpendicular to the $x$-axis intersecting at the point $\bm{D}$ is drawn (see \xref{unit_circle}).

The triangle $\triangle OPD$ is a right triangle. Therefore, we can use the trigonometic functions to calculate the coordinates of the point $\bm{P}=(x,y)$:
\begin{align}
	x &= R\cos(\theta) = \cos(\theta),\nonumber\\
	y &= R\sin(\theta) = \sin(\theta).
	\label{eq:xy_P}
\end{align}

We can now define $\cos(\theta)$ and $\sin(\theta)$ as the values of $x$ and $y$, respectively, as a function of $\theta$.

We will switch to measuring angles in \emph{Radians} instead of degrees: $\theta$ radians are equal to the length of an arc on a unit circle, which corresponds to the angle $\theta$ (\xref{radians}). This allows us to use the same units as $x$ and $y$: for example, when length is measured in [\si{meter}], an angle in radians is measured in [\si{meter}] as well. The full circumference of a circle equals $2\pi$ radians, and therefore a single radian is equivalent to $\frac{180}{\pi} \approx \ang{57.3}$. \xref[tab]{rad_degs} shows some common angles in radians.

\begin{table}
	\caption{Common angles in radians.}
	\label{tab:rad_degs}
	\centering
	\begin{tabular}{ll}
		\toprule
		Degrees & Radians \\
		\midrule
		\ang{0} & 0 \\
		\ang{45} & $\frac{\pi}{4}$ \\
		\ang{90} & $\frac{\pi}{2}$ \\
		\ang{180} & $\pi$ \\
		\ang{270} & $\frac{3\pi}{2}$ \\
		\ang{360} & $2\pi$ \\
		\bottomrule
	\end{tabular}
\end{table}

Another advantage which we gain by defining the trigonometric functions using the unit circle is the extension of their domain to all of $\mathbb{R}$: an angle of size $2\frac{1}{2}\pi$ (equivalent to $\ang{450}$), for example is the same as an angle of size $\frac{1}{2}\pi$ ($\ang{90}$), and an angle of $-\frac{1}{6}\pi$ ($-\ang{30}$) is the same as $\frac{5}{6}\pi$ ($\ang{330}$).

\begin{figure}
	\centering
	\begin{tikzpicture}
		\pgfmathsetmacro{\ax}{4.5}
		\pgfmathsetmacro{\un}{3.5}
		\pgfmathsetmacro{\th}{35}
		\coordinate (D) at ({\un*cos(\th)},0);

		\draw[thick, fill=black!5] (A) circle (\un);
		\draw[vector, <->] (-\ax,0) -- (\ax,0) node [right] {\Large$x$};
		\draw[vector, <->] (0,-\ax) -- (0,\ax) node [above] {\Large$y$};
		\draw[ultra thick, xblue, rotate=\th] (A) -- node [midway, above, rotate=\th] {$R=1$} (\un,0) node (B) {};
		\draw[ultra thick, densely dotted, xpurple] (B.center) -- ({\un*cos(\th)},0);
		\draw[thick] ($(A)+(0.8,0)$) arc (0:\th:0.8) node [midway, xshift=-2mm, yshift=-2pt] {$\theta$};
		\draw[thick] ($(D)+(0,0.3)$) -- ++(-0.3,0) -- ++(0,-0.3);

		\filldraw (A) circle (2pt) node[below left] {$(0,0)=\bm{O}$};
		\filldraw (\un,0) circle (2pt) node[below right] {$(1,0)$};
		\filldraw (0,\un) circle (2pt) node[above right] {$(0,1)$};
		\filldraw (-\un,0) circle (2pt) node[below left] {$(-1,0)$};
		\filldraw (0,-\un) circle (2pt) node[below right] {$(0,-1)$};
		\filldraw (D) circle (2pt) node[below, anchor=north east, xshift=2mm] {$(x,0)=\bm{D}$};
		\filldraw (B) circle (2pt) node[right, xshift=1mm, yshift=1mm] {$\bm{P}=(x,y)$};
	\end{tikzpicture}
	\caption{A unit circle with (...)}
	\label{fig:unit_circle}
\end{figure}

\begin{figure}
	\centering
	\begin{tikzpicture}
		\pgfmathsetmacro{\ax}{4.5}
		\pgfmathsetmacro{\un}{3.5}
		\pgfmathsetmacro{\th}{35}
		\coordinate (D) at ({\un*cos(\th)},0);

		\draw[thick, fill=black!5] (A) circle (\un);
		\draw[vector, <->] (-\ax,0) -- (\ax,0) node [right] {\Large$x$};
		\draw[vector, <->] (0,-\ax) -- (0,\ax) node [above] {\Large$y$};
		%\draw[ultra thick, xblue, rotate=\th] (A) -- node [midway, above, rotate=\th] {$R=1$} (\un,0) node (B) {};
		\draw[xred, fill=xred!15] (A) -- (\un,0) arc (0:\th:\un) -- cycle;
		\draw[xred, ultra thick]  (\un,0) arc (0:\th:\un) node [midway, right] {\Large$\theta$ radians};
		\draw[thick] ($(A)+(0.8,0)$) arc (0:\th:0.8) node [midway, xshift=-2mm, yshift=-2pt] {$\theta$};

		\filldraw (\un,0) circle (2pt) node[below right] {$(1,0)$};
		\filldraw (0,\un) circle (2pt) node[above right] {$(0,1)$};
		\filldraw (-\un,0) circle (2pt) node[below left] {$(-1,0)$};
		\filldraw (0,-\un) circle (2pt) node[below right] {$(0,-1)$};
	\end{tikzpicture}
	\caption{Radians}
	\label{fig:radians}
\end{figure}

\begin{figure}
	\centering
	\begin{tikzpicture}
		\pgfmathsetmacro{\ax}{4.5}
		\pgfmathsetmacro{\un}{3.5}
		\pgfmathsetmacro{\th}{35}
		\coordinate (D) at ({\un*cos(\th)},0);

		\filldraw[xred!35] (A) -- (\un,0) arc (0:90:\un);
		\filldraw[xblue!35] (A) -- (0,\un) arc (90:180:\un);
		\filldraw[xgreen!35] (A) -- (-\un,0) arc (180:270:\un);
		\filldraw[xorange!35] (A) -- (0,-\un) arc (270:360:\un);
		\node at ({ \un/2.5},{ \un/2.5}) {\Huge$1$};
		\node at ({-\un/2.5},{ \un/2.5}) {\Huge$2$};
		\node at ({-\un/2.5},{-\un/2.5}) {\Huge$3$};
		\node at ({ \un/2.5},{-\un/2.5}) {\Huge$4$};
		
		\draw[thick] (A) circle (\un);
		\draw[vector, <->] (-\ax,0) -- (\ax,0) node [right] {\Large$x$};
		\draw[vector, <->] (0,-\ax) -- (0,\ax) node [above] {\Large$y$};
		\filldraw (A) circle (2pt) node[below right] {$(0,0)$};
		\filldraw (\un,0) circle (2pt) node[below right] {$(1,0)$};
		\filldraw (0,\un) circle (2pt) node[above right] {$(0,1)$};
		\filldraw (-\un,0) circle (2pt) node[below left] {$(-1,0)$};
		\filldraw (0,-\un) circle (2pt) node[below right] {$(0,-1)$};
	\end{tikzpicture}
	\caption{The different quadrants of the unit circle.}
	\label{fig:unit_circle_quadrants}
\end{figure}

\begin{table}
	\caption{Text}
	\label{tab:quadrants_trig_vals}
	\centering
	\begin{tabular}{lll}
		\toprule
		Quadrant & $\cos(\theta)=x$ & $\sin(\theta)=y$\\
		\midrule
		\rowcolor{xred!35}1 & $\left[ 0,1 \right]$ & $\left[ 0,1 \right]$\\
		\rowcolor{xblue!35}2 & $\left[-1,0 \right]$ & $\left[ 0,1 \right]$\\
		\rowcolor{xgreen!35}3 & $\left[-1,0 \right]$ & $\left[-1,0 \right]$\\
		\rowcolor{xorange!35}4 & $\left[ 0,1 \right]$ & $\left[-1,0 \right]$\\
		\bottomrule
	\end{tabular}
\end{table}

\section{Exponential and Logarithmic Functions}
