\chapter{Introduction}
In this chapter we will introduce some key concepts that will be used in later chapters.

\begin{note}{In case you are already familiar with the topics}{}
	It is recommended for readers who are familiar with the topics to at least gloss over this chapter and make sure they know and understand all the concept presented here.
\end{note}

\section{Trigonometry}
\subsection{Basic Definitions}
Consider a \emph{right triangle} $ABC$ with sides $a,b$, and Hypotenous $c$, where the angle $\angle ACB$ is $\ang{90}$, and the angle $\angle BAC$ is denoted as $\alpha$:

\centering
\begin{tikzpicture}[node distance=3mm]
		\coordinate (A) at (0,0);
		\coordinate (B) at (4,3);
		\coordinate (C) at (4,0);

		\node[left of=A] {$A$};
		\node[above of=B] {$B$};
		\node[right of=C] {$C$};

		\draw[fill=xblue!30] (A) -- node (c) [midway, above, rotate=36.87] {$c$ (Hypotenous)} (B) -- node (a) [midway, right] {$a$ (Opposite)} (C) -- node (b) [midway, below] {$b$ (Adjacent)} cycle;
		\draw[thick] ($(C)+(0,0.3)$) rectangle ($(C)-(0.3,0)$);
		\draw[thick, xpurple!50!black, fill=xpurple!45] (A) -- ($(A)+(1,0)$) arc (0:36.87:1) node [midway, xshift=-3mm, yshift=-2pt] {$\alpha$} -- cycle;
		%\draw[thick, xblue!50!black, fill=xblue!45] (B) -- ($(B)+(0,-0.8)$) arc (270:216.97:0.8) node [midway, above, xshift=5pt] {$\beta$} -- cycle;
		\draw[thick] (A) -- (B) -- (C) -- cycle;
	\end{tikzpicture}
\flushleft

We use the ratios between the three sides of the triangle to define three functions of $\alpha$:
\begin{definition}{The basic triginometric functions}{}
	\vspace{5mm}
	\begin{enumerate}
		\item The \emph{sine} of the angle $\alpha$ is $\sin(\alpha)=\frac{a}{c}$,
		\item the \emph{cosine} of the angle $\alpha$ is $\cos(\alpha)=\frac{b}{c}$, and
		\item the \emph{tangent} of the angle $\alpha$ is $\tan(\alpha)=\frac{a}{b}$, which in turn is equal to $\frac{\sin(\alpha)}{\cos(\alpha)}$.
		\end{enumerate}
	\label{def:basic_trig}
\end{definition}

We can rearrange the above definitions to yield
\begin{align}
	a &= c\sin(\alpha),\\
	b &= c\cos(\alpha).
	\label{eq:basic_trig_rearrange}
\end{align}

Normaly, the Hypotenous is the longest side of a right triangle. We will consider here the two edge cases where one of the sides $a,b$ is equal to the Hypotenous (and the other side is thus $0$):
\begin{itemize}
	\item if $a=c$ then $\alpha=\ang{90}$,\
	\item if $b=c$ then $\alpha=0$.
\end{itemize}

The posssible length of $a$ is therefore in the range $0\leq a \leq c$, which means that $0\leq \frac{a}{c} \leq 1$, or since $\sin(\alpha)=\frac{a}{c}$,
\begin{equation}
	0\leq \sin(\alpha) \leq1.
	\label{eq:img_sin}
\end{equation}

The same is of course true for $b$, and thus
\begin{equation}
	0\leq \cos(\alpha) \leq1
	\label{eq:img_sin}
\end{equation}
as well.

\subsection{The Unit Circle}
\begin{tikzpicture}
	\pgfmathsetmacro{\ax}{5}
	\pgfmathsetmacro{\un}{4}
	\pgfmathsetmacro{\th}{35}
	\coordinate (C) at ({\un*cos(\th)},0);
	\draw[thick, fill=xgreen!10] (A) circle (\un);
	\draw[vector, <->] (-\ax,0) -- (\ax,0) node [right] {$x$};
	\draw[vector, <->] (0,-\ax) -- (0,\ax) node [above] {$y$};
	\draw[ultra thick, xblue, rotate=\th] (A) -- node [midway, above, rotate=\th] {$R=1$} (\un,0) node (B) {};
	\draw[ultra thick, densely dotted, xpurple] (B.center) -- ({\un*cos(\th)},0);
	\draw[ultra thick, densely dotted, xred] (A) -- ({\un*cos(\th)},0);
	\draw[very thick, xred, decorate, decoration={brace, amplitude=3pt, raise=4pt, mirror}] (A) -- ({\un*cos(\th)},0) node[midway, below, yshift=-5pt] {$\cos(\alpha)$};
	\draw[very thick, xpurple, decorate, decoration={brace, amplitude=3pt, raise=4pt}] (B.center) -- ({\un*cos(\th)},0) node [midway, right, xshift=14mm] (Atxt) {$\sin(\alpha)$};
	\draw[very thick, xpurple] (Atxt.west) -- ++(-33pt,0);
	\draw[thick] ($(A)+(0.8,0)$) arc (0:\th:0.8) node [midway, xshift=-2mm, yshift=-2pt] {$\alpha$};
	\draw[thick] ($(C)+(0,0.3)$) -- ++(-0.3,0) -- ++(0,-0.3);
\end{tikzpicture}

\section{Sets}
