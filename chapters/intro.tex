\chapter{Introduction}
In this chapter we will introduce some key concepts that will be used in later chapters.

\begin{note}{In case you are already familiar with the topics}{}
	It is recommended for readers who are familiar with the topics to at least gloss over this chapter and make sure they know and understand all the concept presented here.
\end{note}

\section{Mathematical Sybols and Sets}
\subsection{Logical Statements and their Truth Value}
\Blindtext[1]

\section{Relations and Functions}
\section{Trigonometric functions}
\subsection{Basic Definitions}
Consider a \emph{right triangle} $\triangle ABC$ with sides $a,b$, and Hypotenous $c$, where the angle $\angle ACB$ is $\ang{90}$, and the angle $\angle BAC$ is denoted as $\alpha$:

\centering
\begin{tikzpicture}[node distance=3mm]
		\coordinate (A) at (0,0);
		\coordinate (B) at (4,3);
		\coordinate (C) at (4,0);

		\node[left of=A] {$A$};
		\node[above of=B] {$B$};
		\node[right of=C] {$C$};

		\draw[fill=xblue!30] (A) -- node (c) [midway, above, rotate=36.87] {$c$ (Hypotenous)} (B) -- node (a) [midway, right] {$a$ (Opposite)} (C) -- node (b) [midway, below] {$b$ (Adjacent)} cycle;
		\draw[thick] ($(C)+(0,0.3)$) rectangle ($(C)-(0.3,0)$);
		\draw[thick, xpurple!50!black, fill=xpurple!45] (A) -- ($(A)+(1,0)$) arc (0:36.87:1) node [midway, xshift=-3mm, yshift=-2pt] {$\alpha$} -- cycle;
		%\draw[thick, xblue!50!black, fill=xblue!45] (B) -- ($(B)+(0,-0.8)$) arc (270:216.97:0.8) node [midway, above, xshift=5pt] {$\beta$} -- cycle;
		\draw[thick] (A) -- (B) -- (C) -- cycle;
	\end{tikzpicture}
\flushleft

We use the ratios between the three sides of the triangle to define three functions of $\alpha$:
\begin{definition}{The basic triginometric functions}{}
	\vspace{5mm}
	\begin{enumerate}
		\item The \emph{sine} of the angle $\alpha$ is $\sin(\alpha)=\frac{a}{c}$,
		\item the \emph{cosine} of the angle $\alpha$ is $\cos(\alpha)=\frac{b}{c}$, and
		\item the \emph{tangent} of the angle $\alpha$ is $\tan(\alpha)=\frac{a}{b}$, which in turn is equal to $\frac{\sin(\alpha)}{\cos(\alpha)}$.
		\end{enumerate}
	\label{def:basic_trig}
\end{definition}

We can rearrange the above definitions to yield
\begin{align}
	a &= c\sin(\alpha),\nonumber\\
	b &= c\cos(\alpha).
	\label{eq:basic_trig_rearrange}
\end{align}

Normaly, the Hypotenous is the longest side of a right triangle. We will consider here the two edge cases where one of the sides $a,b$ is equal to the Hypotenous (and the other side is thus $0$):
\begin{itemize}
	\item if $a=c$ then $\alpha=\ang{90}$,\
	\item if $b=c$ then $\alpha=0$.
\end{itemize}

The posssible length of $a$ is therefore in the range $0\leq a \leq c$, which means that $0\leq \frac{a}{c} \leq 1$, or since $\sin(\alpha)=\frac{a}{c}$,
\begin{equation}
	0\leq \sin(\alpha) \leq1.
	\label{eq:img_sin}
\end{equation}

The same is of course true for $b$, and thus
\begin{equation}
	0\leq \cos(\alpha) \leq1
	\label{eq:img_cos}
\end{equation}
as well.

As a reminder, the \emph{Pythagorean theorem}\footnote{It's worth mentioning that no three positive integers $a, b$, and $c$ satisfy the equation $a^{n}+b^{n}=c^{n}$ for any integer value of $n>2$. \href{https://en.wikipedia.org/wiki/Fermat\%27s_Last_Theorem}{This can be proven, however the proof is too large to fit in the footnotes}.} states that for a right triangle like the one here,
\begin{equation}
	a^{2} + b^{2} = c^{2}.
	\label{eq:pythagorean_theorem}
\end{equation}
By substituting \xref[eq]{basic_trig_rearrange} into the above we get
\begin{equation*}
	c^{2} = a^{2}+b^{2} = \left( c\sin(\alpha) \right)^{2} + \left( c\cos(\alpha) \right)^{2} = c^{2}\sin^{2}(\alpha) + c^{2}\cos^{2}(\alpha) = c^{2}\left[ \sin^{2}(\alpha) + \cos^{2}(\alpha) \right],
\end{equation*}
and cancelling $c^{2}$ on both sides simply yields
\begin{equation}
	\sin^{2}(\alpha) + \cos^{2}(\alpha) = 1.
	\label{eq:sin2_cos2_1}
\end{equation}

\subsection{The Unit Circle}
The range of the trigonometric functions can be extended by using the \emph{unit circle}: a circle of radius $R=1$ is placed such that its center lies at the origin of a 2-dimensional axis system, i.e. at the point $\bm{O}=(0,0)$. A radius to a point $\bm{P}=(x,y)$ on the circle's circumference is the drawn. This radius has an angle $\theta$ to the $x$-axis. A line from $\bm{P}$ perpendicular to the $x$-axis intersecting at the point $\bm{D}$ is drawn (see \xref{unit_circle}).

The triangle $\triangle OPD$ is a right triangle. Therefore, we can use the trigonometic functions to calculate the coordinates of the point $\bm{P}=(x,y)$:
\begin{align}
	x &= R\cos(\theta) = \cos(\theta),\nonumber\\
	y &= R\sin(\theta) = \sin(\theta).
	\label{eq:xy_P}
\end{align}

We can now define $\cos(\theta)$ and $\sin(\theta)$ as the values of $x$ and $y$, respectively, as a function of $\theta$.

We will switch to measuring angles in \emph{Radians} instead of degrees: $\theta$ radians are equal to the length of an arc on a unit circle, which corresponds to the angle $\theta$ (\xref{radians}). This allows us to use the same units as $x$ and $y$: for example, when length is measured in [\si{meter}], an angle in radians is measured in [\si{meter}] as well. The full circumference of a circle equals $2\pi$ radians, and therefore a single radian is equivalent to $\frac{180}{\pi} \approx \ang{57.3}$. \xref[tab]{rad_degs} shows some common angles in radians.

\begin{table}
	\caption{Common angles in radians.}
	\label{tab:rad_degs}
	\centering
	\begin{tabular}{ll}
		\toprule
		Degrees & Radians \\
		\midrule
		\ang{0} & 0 \\
		\ang{45} & $\frac{\pi}{4}$ \\
		\ang{90} & $\frac{\pi}{2}$ \\
		\ang{180} & $\pi$ \\
		\ang{270} & $\frac{3\pi}{2}$ \\
		\ang{360} & $2\pi$ \\
		\bottomrule
	\end{tabular}
\end{table}

Another advantage which we gain by defining the trigonometric functions using the unit circle is the extension of their domain to all of $\mathbb{R}$: an angle of size $2\frac{1}{2}\pi$ (equivalent to $\ang{450}$), for example is the same as an angle of size $\frac{1}{2}\pi$ ($\ang{90}$), and an angle of $-\frac{1}{6}\pi$ ($-\ang{30}$) is the same as $\frac{5}{6}\pi$ ($\ang{330}$).

\begin{figure}
	\centering
	\begin{tikzpicture}
		\pgfmathsetmacro{\ax}{4.5}
		\pgfmathsetmacro{\un}{3.5}
		\pgfmathsetmacro{\th}{35}
		\coordinate (D) at ({\un*cos(\th)},0);

		\draw[thick, fill=black!5] (A) circle (\un);
		\draw[vector, <->] (-\ax,0) -- (\ax,0) node [right] {\Large$x$};
		\draw[vector, <->] (0,-\ax) -- (0,\ax) node [above] {\Large$y$};
		\draw[ultra thick, xblue, rotate=\th] (A) -- node [midway, above, rotate=\th] {$R=1$} (\un,0) node (B) {};
		\draw[ultra thick, densely dotted, xpurple] (B.center) -- ({\un*cos(\th)},0);
		\draw[thick] ($(A)+(0.8,0)$) arc (0:\th:0.8) node [midway, xshift=-2mm, yshift=-2pt] {$\theta$};
		\draw[thick] ($(D)+(0,0.3)$) -- ++(-0.3,0) -- ++(0,-0.3);

		\filldraw (A) circle (2pt) node[below left] {$(0,0)=\bm{O}$};
		\filldraw (\un,0) circle (2pt) node[below right] {$(1,0)$};
		\filldraw (0,\un) circle (2pt) node[above right] {$(0,1)$};
		\filldraw (-\un,0) circle (2pt) node[below left] {$(-1,0)$};
		\filldraw (0,-\un) circle (2pt) node[below right] {$(0,-1)$};
		\filldraw (D) circle (2pt) node[below, anchor=north east, xshift=2mm] {$(x,0)=\bm{D}$};
		\filldraw (B) circle (2pt) node[right, xshift=1mm, yshift=1mm] {$\bm{P}=(x,y)$};
	\end{tikzpicture}
	\caption{A unit circle with (...)}
	\label{fig:unit_circle}
\end{figure}

\begin{figure}
	\centering
	\begin{tikzpicture}
		\pgfmathsetmacro{\ax}{4.5}
		\pgfmathsetmacro{\un}{3.5}
		\pgfmathsetmacro{\th}{35}
		\coordinate (D) at ({\un*cos(\th)},0);

		\draw[thick, fill=black!5] (A) circle (\un);
		\draw[vector, <->] (-\ax,0) -- (\ax,0) node [right] {\Large$x$};
		\draw[vector, <->] (0,-\ax) -- (0,\ax) node [above] {\Large$y$};
		%\draw[ultra thick, xblue, rotate=\th] (A) -- node [midway, above, rotate=\th] {$R=1$} (\un,0) node (B) {};
		\draw[xred, fill=xred!15] (A) -- (\un,0) arc (0:\th:\un) -- cycle;
		\draw[xred, ultra thick]  (\un,0) arc (0:\th:\un) node [midway, right] {\Large$\theta$ radians};
		\draw[thick] ($(A)+(0.8,0)$) arc (0:\th:0.8) node [midway, xshift=-2mm, yshift=-2pt] {$\theta$};

		\filldraw (\un,0) circle (2pt) node[below right] {$(1,0)$};
		\filldraw (0,\un) circle (2pt) node[above right] {$(0,1)$};
		\filldraw (-\un,0) circle (2pt) node[below left] {$(-1,0)$};
		\filldraw (0,-\un) circle (2pt) node[below right] {$(0,-1)$};
	\end{tikzpicture}
	\caption{Radians}
	\label{fig:radians}
\end{figure}

\begin{figure}
	\centering
	\begin{tikzpicture}
		\pgfmathsetmacro{\ax}{4.5}
		\pgfmathsetmacro{\un}{3.5}
		\pgfmathsetmacro{\th}{35}
		\coordinate (D) at ({\un*cos(\th)},0);

		\filldraw[xred!35] (A) -- (\un,0) arc (0:90:\un);
		\filldraw[xblue!35] (A) -- (0,\un) arc (90:180:\un);
		\filldraw[xgreen!35] (A) -- (-\un,0) arc (180:270:\un);
		\filldraw[xorange!35] (A) -- (0,-\un) arc (270:360:\un);
		\node at ({ \un/2.5},{ \un/2.5}) {\Huge$1$};
		\node at ({-\un/2.5},{ \un/2.5}) {\Huge$2$};
		\node at ({-\un/2.5},{-\un/2.5}) {\Huge$3$};
		\node at ({ \un/2.5},{-\un/2.5}) {\Huge$4$};
		
		\draw[thick] (A) circle (\un);
		\draw[vector, <->] (-\ax,0) -- (\ax,0) node [right] {\Large$x$};
		\draw[vector, <->] (0,-\ax) -- (0,\ax) node [above] {\Large$y$};
		\filldraw (A) circle (2pt) node[below right] {$(0,0)$};
		\filldraw (\un,0) circle (2pt) node[below right] {$(1,0)$};
		\filldraw (0,\un) circle (2pt) node[above right] {$(0,1)$};
		\filldraw (-\un,0) circle (2pt) node[below left] {$(-1,0)$};
		\filldraw (0,-\un) circle (2pt) node[below right] {$(0,-1)$};
	\end{tikzpicture}
	\caption{The different quadrants of the unit circle.}
	\label{fig:unit_circle_quadrants}
\end{figure}

\begin{table}
	\caption{Text}
	\label{tab:quadrants_trig_vals}
	\centering
	\begin{tabular}{lll}
		\toprule
		Quadrant & $\cos(\theta)=x$ & $\sin(\theta)=y$\\
		\midrule
		\rowcolor{xred!35}1 & $\left[ 0,1 \right]$ & $\left[ 0,1 \right]$\\
		\rowcolor{xblue!35}2 & $\left[-1,0 \right]$ & $\left[ 0,1 \right]$\\
		\rowcolor{xgreen!35}3 & $\left[-1,0 \right]$ & $\left[-1,0 \right]$\\
		\rowcolor{xorange!35}4 & $\left[ 0,1 \right]$ & $\left[-1,0 \right]$\\
		\bottomrule
	\end{tabular}
\end{table}

%\section{Sets}
