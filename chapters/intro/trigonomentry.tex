\pgfplotsset{
  layers/axis lines on top/.define layer set={
    axis background,
    axis grid,
    axis ticks,
    axis tick labels,
    pre main,
    main,
    axis lines,
    axis descriptions,
    axis foreground,
  }{/pgfplots/layers/standard},
}

\pgfkeys{
	/pgfplots/unit circle/.style={
		set layers=axis lines on top,
		width=9cm, height=9cm,
		axis x line=middle,
		axis y line=middle,
		xlabel=$x$,
		ylabel=$y$,
		every axis x label/.style={
			at={(ticklabel* cs:1.01)},
			anchor=west,
		},
		every axis y label/.style={
			at={(ticklabel* cs:1.01)},
			anchor=south,
		},
		axis line style={stealth-stealth, thick},
		label style={font=\Large},
		tick label style={font=\Large},
		samples=100,
		xmin=-1.25, xmax=1.25,
		ymin=-1.25, ymax=1.25,
		xtick={-1,0,1},
		ytick={-1,0,1},
		xticklabels={},
		yticklabels={},
		domain=-1.5:1.5,
		grid=both,
		major grid style={black!5},
	},
}


\section{Trigonometric functions}
\subsection{Basic Definitions}
Consider a \emph{right triangle} $\triangle ABC$ with sides $a,b$, and Hypotenous $c$, where the angle $\angle ACB$ is $\ang{90}$, and the angle $\angle BAC$ is denoted as $\alpha$:

\centering
\begin{tikzpicture}[node distance=3mm]
		\coordinate (A) at (0,0);
		\coordinate (B) at (4,3);
		\coordinate (C) at (4,0);

		\node[left of=A] {$A$};
		\node[above of=B] {$B$};
		\node[right of=C] {$C$};

		\draw[fill=xblue!30] (A) -- node (c) [midway, above, rotate=36.87] {$c$ (Hypotenous)} (B) -- node (a) [midway, right] {$a$ (Opposite)} (C) -- node (b) [midway, below] {$b$ (Adjacent)} cycle;
		\draw[thick] ($(C)+(0,0.3)$) rectangle ($(C)-(0.3,0)$);
		\draw[thick, xpurple!50!black, fill=xpurple!45] (A) -- ($(A)+(1,0)$) arc (0:36.87:1) node [midway, xshift=-3mm, yshift=-2pt] {$\alpha$} -- cycle;
		%\draw[thick, xblue!50!black, fill=xblue!45] (B) -- ($(B)+(0,-0.8)$) arc (270:216.97:0.8) node [midway, above, xshift=5pt] {$\beta$} -- cycle;
		\draw[thick] (A) -- (B) -- (C) -- cycle;
	\end{tikzpicture}
\flushleft

We use the ratios between the three sides of the triangle to define three functions of $\alpha$:
\vspace{5mm}
\begin{itemize}
	\item The \emph{sine} of the angle $\alpha$ is $\sin(\alpha)=\frac{a}{c}$,
	\item the \emph{cosine} of the angle $\alpha$ is $\cos(\alpha)=\frac{b}{c}$, and
	\item the \emph{tangent} of the angle $\alpha$ is $\tan(\alpha)=\frac{a}{b}$, which in turn is equal to $\frac{\sin(\alpha)}{\cos(\alpha)}$.
\end{itemize}

We can rearrange the above definitions:
\begin{align}
	a &= c\sin(\alpha),\nonumber\\
	b &= c\cos(\alpha).
	\label{eq:basic_trig_rearrange}
\end{align}

Normaly, the Hypotenous is the longest side of a right triangle. We will consider here the two edge cases where one of the sides $a$ or $b$ is equal to the Hypotenous (and the other side is thus $0$):
\begin{itemize}
	\item if $a=c$ then $\alpha=\ang{90}$,\
	\item if $b=c$ then $\alpha=\ang{0}$.
\end{itemize}

The posssible length of $a$ is therefore in the range $0\leq a \leq c$, which means that $0\leq \frac{a}{c} \leq 1$. Since $\sin(\alpha)=\frac{a}{c}$ this means that the image of $\sin(\alpha)$ is $[0,1]$. The same idea is also true for $b$, and therefore $[0,1]$ is the image of $\cos(\alpha)$ as well.

As a reminder, the \emph{Pythagorean theorem}\footnote{It's worth mentioning that no three positive integers $a, b$, and $c$ satisfy the equation $a^{n}+b^{n}=c^{n}$ for any integer value of $n>2$. \href{https://en.wikipedia.org/wiki/Fermat\%27s_Last_Theorem}{This can be proven, however the proof is too large to fit in the footnotes}.} states that for a right triangle with sides $a, b$ and $c$,
\begin{equation}
	a^{2} + b^{2} = c^{2}.
	\label{eq:pythagorean_theorem}
\end{equation}
By substituting \xref[eq]{basic_trig_rearrange} into the Pythagorean theorem we get
\begin{align*}
	c^{2} &= a^{2}+b^{2}\\
	&= \left[ c\sin(\alpha) \right]^{2} + \left[ c\cos(\alpha) \right]^{2}\\
	&= c^{2}\sin^{2}(\alpha) + c^{2}\cos^{2}(\alpha)\\
	&= c^{2}\left[ \sin^{2}(\alpha) + \cos^{2}(\alpha) \right],
\end{align*}
and therefore
\begin{equation}
	\sin^{2}(x) + \cos^{2}(x) = 1.
	\label{eq:sin sqr plus cos sqr equals 1}
\end{equation}

\subsection{The Unit Circle}
We defined $\sin(\alpha)$ and $\cos(\alpha)$ so far in way such that their domains are both $[\ang{0},\ang{90}]$, and their images are both $[0,1]$. However, there is a simple way to extend these functions such that both their domains are $\mathbb{R}$, and both their images are $[-1,1]$: by using a \emph{unit circle}.

\figref{unit_circle} depicts a unit circle: it is simply a circle of radius $R=1$, which is placed such that its center lies at the origin of a 2-dimensional axis system (i.e. at the point $\bm{O}=(0,0)$. We then draw a line from $\bm{O}$ to a point $\bm{P}=(x,y)$ on the circumference of the unit circle. We call the angle between the line $\bm{OP}$ and the $x$-axis $\theta$. We then draw another line, this time from the point $\bm{P}$ to a point $\bm{D}$ on the $x$-axis, such that $\bm{PD}$ is perpendicular to the $x$-axis.

The triangle $\triangle OPD$ is a right triangle. Therefore, we can use the trigonometic functions to calculate the coordinates of the point $\bm{P}=(x,y)$:
\begin{align}
	x &= R\cos(\theta) = \cos(\theta),\nonumber\\
	y &= R\sin(\theta) = \sin(\theta).
	\label{eq:xy_P}
\end{align}

We then define $\cos(\theta)$ and $\sin(\theta)$ as a function of $\theta$:
\begin{align}
	\sin(\theta) &= y,\nonumber\\
	\cos(\theta) &= x.
	\label{eq:unit circle definition of sin and cos}
\end{align}

Using this definition, the angle $\theta$ can take any value between $\ang{0}$ and $\ang{360}$. In fact, the values of $\theta$ can be extended to any real number in degrees: if it is more than $\ang{360}$\ldots (MUST FINISH THIS PART + QUADRENTS)

\subsection{Radians}
Using degrees to measure angles in a sphere creates an inconvinience: the domain and image of the trigonometric functions have different units. In order to measure both these magnitudes using the same unit we switch to measuring angles on a circle using \emph{radians} instead of degrees. One radian equals the length of a single radius $R$ of the circle (in the case of the unit circle this is always $R=1$). We define an inner angle $\theta$ to equal one radian if the arc length it represents is equal to $R$ (see \figref{radians}).

How much is a radian in degrees? The full circumference of any circle with radius $R$ equals $2\pi R$, which means that a single radian $R$ is equivalent to $\frac{\ang{180}}{\pi} \approx \ang{57.3}$. \xref[tab]{rad_degs} shows some common angles and their equivalent value in radians.

\begin{table}
	\caption{Common angles in radians.}
	\label{tab:rad_degs}
	\centering
	\begin{tabular}{ll}
		\toprule
		Degrees & Radians \\
		\midrule
		\ang{0} & 0 \\
		\ang{45} & $\frac{\pi}{4}$ \\
		\ang{90} & $\frac{\pi}{2}$ \\
		\ang{180} & $\pi$ \\
		\ang{270} & $\frac{3\pi}{2}$ \\
		\ang{360} & $2\pi$ \\
		\bottomrule
	\end{tabular}
\end{table}

\begin{figure}
	\centering
	\begin{tikzpicture}[scale=0.9]
		\pgfmathsetmacro{\ax}{4.5}
		\pgfmathsetmacro{\un}{3.5}
		\pgfmathsetmacro{\th}{35}
		\coordinate (D) at ({\un*cos(\th)},0);

		\def\xcol{xred}
		\draw[thick, fill=\xcol!5] (A) circle (\un);
		\draw[vector, <->] (-\ax,0) -- (\ax,0) node [right] {\Large$x$};
		\draw[vector, <->] (0,-\ax) -- (0,\ax) node [above] {\Large$y$};
		\draw[ultra thick, \xcol, rotate=\th] (A) -- node [midway, above, rotate=\th] {$R=1$} (\un,0) node (B) {};
		\draw[very thick, densely dotted, black!50] (B.center) -- ({\un*cos(\th)},0);
		\draw[thick] ($(A)+(1.1,0)$) arc (0:\th:1.1) node [midway, xshift=-2mm, yshift=-2pt] {$\theta$};
		\draw[thick] ($(D)+(0,0.3)$) -- ++(-0.3,0) -- ++(0,-0.3);

		\filldraw (A) circle (2pt) node[below left] {$(0,0)=\bm{O}$};
		\filldraw (\un,0) circle (2pt) node[below right] {$(1,0)$};
		\filldraw (0,\un) circle (2pt) node[above right] {$(0,1)$};
		\filldraw (-\un,0) circle (2pt) node[below left] {$(-1,0)$};
		\filldraw (0,-\un) circle (2pt) node[below right] {$(0,-1)$};
		\filldraw (D) circle (2pt) node[below, anchor=north east, xshift=2mm] {$(x,0)=\bm{D}$};
		\filldraw (B) circle (2pt) node[right, xshift=1mm, yshift=1mm] {$\bm{P}=(x,y)$};
	\end{tikzpicture}
	\caption{A unit circle with a point $\bm{P}=(x,y)$ on its circumference. The triangle $\triangle \bm{OPD}$ is a right triangle with sides $\bm{OD}=x$, $\bm{OP}=y$ and an angle $\theta$ opposing the side $\bm{DP}$.}
	\label{fig:unit_circle}
\end{figure}

\begin{figure}
	\centering
	\begin{tikzpicture}[node distance=5mm, every node/.style={font=\large}]
		\begin{axis}[
			unit circle,
		]
			\def\xcol{xpurple}
			\def\sr{0.3}
			\def\rad{57.3}
			\coordinate (O) at (0,0);
			\coordinate (A) at (1,0);
			\coordinate (B) at ({cos(\rad)},{sin(\rad)});
			\coordinate (C) at ({\sr*cos(\rad)},{\sr*sin(\rad)});
			\coordinate (T) at ({\sr/1.5*cos(\rad/2)},{\sr/1.5*sin(\rad/2)});
			\coordinate (L1) at ({0.8*cos(\rad/2)},{0.8*sin(\rad/2)});
			\coordinate (L2) at ({0.35*cos(\rad/2)},{0.35*sin(\rad/2)});
			
			\draw[very thick, fill=\xcol!5] (A) arc (0:360:1);
			%\draw[line width=5mm, xpurple!75] (A) arc (0:\rad:1);
			\def\outershift#1{\raisebox{1ex}}
			\draw[very thick, fill=\xcol!50, postaction={decorate, decoration={text along path, text align=center, text={|\large\outershift|arc length=$R$}}}] (O) -- node[midway, above, rotate=\rad] {$R$} (B) arc (\rad:0:1) -- node[midway, below] {$R$} cycle;
			\def\innershift#1{\raisebox{-2.5ex}}
			\draw[postaction={decorate, decoration={text along path, text align=center, text={|\large\innershift|$1$ radians}}}] (B) arc (\rad:0:1);
			\draw[very thick, fill=\xcol!20] (O) -- (C) arc (\rad:0:\sr) -- cycle;
			
			\addplot+[only marks, black] coordinates {(0,0) (1,0) ({cos(\rad)},{sin(\rad)})};
			\node[below left of=O] {$\bm{O}$};
			\node[below right of=A] {$\bm{A}$};
			\node[above of=B] {$\bm{B}$};
			\node at (T) {$\theta$};
			\draw[vector, thin] (L1) -- (L2);
		\end{axis}
	\end{tikzpicture}
	\caption{In this figure the arc $\bm{AB}$ has the same length of the radii $\bm{OA}$ and $\bm{OB}$ (all are equal to $R$), and therefore $\theta=1$ radians.}
	\label{fig:radians}
\end{figure}

MORE WILL BE WRITTEN HERE!

\subsection{Graphs}
As seen previosuly, the functions $\sin(x)$ and $\cos(x)$ are periodic, having both the period $T=2\pi$. Their graphs are depicted in \figref{sin and cos graphs}. The value of $\sin(x)$ is always equal to that of $\cos\left(x-\frac{\pi}{2}\right)$: we say that the two functions have a \emph{phase difference} of $\pi/2$.

\begin{figure}
	\centering
	\begin{tikzpicture}
		\begin{axis}[
				graph2d,
				width=15cm, height=4cm,
				xmin=-10, xmax=10,
				ymin=-1.2, ymax=1.2,
				xtick={-9.425,-7.854,...,9.425,10.996},
				xticklabels={$-3\pi$, $-\frac{5}{2}\pi$, $-2\pi$, $-\frac{3}{2}\pi$, $-\pi$, $-\frac{1}{2}\pi$, , $\frac{1}{2}\pi$, $\pi$, $\frac{3}{2}\pi$, $2\pi$, $\frac{5}{2}\pi$, $3\pi$},
				domain=-10:10,
			]
			\addplot[function, xred] {sin(\x*180/pi)};
			\addplot[function, xgreen] {cos(\x*180/pi)};
		\end{axis}
	\end{tikzpicture}
	\caption{The graphs of \textcolor{xred}{\bm{$\sin(x)$}} and \textcolor{xgreen}{\bm{$\cos(x)$}} for $x\in[-10,10]$. Note how the graph of $\cos(x)$ is ``lagging'' behind the graph of $\sin(x)$ by $\pi/2$.}
	\label{fig:sin and cos graphs}
\end{figure}

\subsection{Identities}
\Blindtext

\subsection{Usefull theorems}
The area $S$ of a triangle $\triangle ABC$ can be calculated using the length $L$ any side of the triangle (in this context called a \emph{base}) and the height $h$ to its opposing vertex (see \figref{area of a triangle}):
\begin{equation}
	S = \frac{1}{2}Lh.
	\label{eq:area of a triangle}
\end{equation}

\begin{figure}
	\centering
	\begin{tikzpicture}
		\coordinate (A) at (-2,1);
		\coordinate (B) at (1,3.5);
		\coordinate (C) at (2,1);
		\coordinate (D) at (B|-C);
		\draw[very thick, fill=xblue!20] (A) node[left] {$A$} -- (B) node[above] {$B$} node[midway, above left] {$c$} -- node[midway, right] {$a$} (C) node[right] {$C$} -- cycle node[midway, below] {$b$};
		\draw[thick, dashed] (B) -- (D) node[midway, left] {$h$};
		\draw[thick] (D) -- ++(0.2,0) -- ++(0,0.2) -- ++(-0.2,0);
		\pic[draw, thick, angle radius=11mm, angle eccentricity=0.7, "$\alpha$"] {angle=C--A--B};
	\end{tikzpicture}
	\caption{The area of a triangle using the side $b$ as a base, and its corresponding height to the point $B$. The angle opposing the side $A$ is marked as $\alpha$.}
	\label{fig:area of a triangle}
\end{figure}

The triangle with sides $cbh$ is a right triangle, $c$ being its hypotenous. We can therefore infer the size of $h$ using $\alpha$:
\begin{equation}
	h = c\sin(\alpha).
	\label{eq:first equation in law of sines}
\end{equation}
Subtituting this back to \eqref{area of a triangle} yields that the area of the triangle is
\begin{equation}
	S = \frac{1}{2}bc\sin(\alpha).
	\label{eq:area using sin theta}
\end{equation}
There is nothing special about choosing the side $b$ as a base: we can also use $a$ or $c$ for the calculation. This will yield, repspectively,
\begin{align}
	S &= \frac{1}{2}ac\sin(\gamma),\\
	S &= \frac{1}{2}ac\sin(\beta),
	\label{eq:}
\end{align}
where $\beta$ is the angle opposing $b$ and $\gamma$ is the angle opposing $c$. Since $S$ is the same in all cases, we simply multiply each of the area equations by $2$ and divide by $abc$, which yields
\begin{equation}
	\frac{\sin(\alpha)}{a} = \frac{\sin(\beta)}{b} = \frac{\sin(\gamma)}{c},
	\label{eq:law of sines}
\end{equation}
i.e. in a triangle, the ratio between any side and the sine of its opposing angle is always the same no matter which side we choose. This theorem is called the \emph{law of sines}.

\begin{example}{Law of sines}{}
	Here will be an example.
\end{example}
%\begin{figure}
%	\centering
%	\begin{tikzpicture}
%		\pgfmathsetmacro{\ax}{4.5}
%		\pgfmathsetmacro{\un}{3.5}
%		\pgfmathsetmacro{\th}{35}
%		\coordinate (D) at ({\un*cos(\th)},0);
%
%		\filldraw[xred!35] (A) -- (\un,0) arc (0:90:\un);
%		\filldraw[xblue!35] (A) -- (0,\un) arc (90:180:\un);
%		\filldraw[xgreen!35] (A) -- (-\un,0) arc (180:270:\un);
%		\filldraw[xorange!35] (A) -- (0,-\un) arc (270:360:\un);
%		\node at ({ \un/2.5},{ \un/2.5}) {\Huge$1$};
%		\node at ({-\un/2.5},{ \un/2.5}) {\Huge$2$};
%		\node at ({-\un/2.5},{-\un/2.5}) {\Huge$3$};
%		\node at ({ \un/2.5},{-\un/2.5}) {\Huge$4$};
%		
%		\draw[thick] (A) circle (\un);
%		\draw[vector, <->] (-\ax,0) -- (\ax,0) node [right] {\Large$x$};
%		\draw[vector, <->] (0,-\ax) -- (0,\ax) node [above] {\Large$y$};
%		\filldraw (A) circle (2pt) node[below right] {$(0,0)$};
%		\filldraw (\un,0) circle (2pt) node[below right] {$(1,0)$};
%		\filldraw (0,\un) circle (2pt) node[above right] {$(0,1)$};
%		\filldraw (-\un,0) circle (2pt) node[below left] {$(-1,0)$};
%		\filldraw (0,-\un) circle (2pt) node[below right] {$(0,-1)$};
%	\end{tikzpicture}
%	\caption{The different quadrants of the unit circle.}
%	\label{fig:unit_circle_quadrants}
%\end{figure}

%\begin{table}
%	\caption{Text}
%	\label{tab:quadrants_trig_vals}
%	\centering
%	\begin{tabular}{lll}
%		\toprule
%		Quadrant & $\cos(\theta)=x$ & $\sin(\theta)=y$\\
%		\midrule
%		\rowcolor{xred!35}1 & $\left[ 0,1 \right]$ & $\left[ 0,1 \right]$\\
%		\rowcolor{xblue!35}2 & $\left[-1,0 \right]$ & $\left[ 0,1 \right]$\\
%		\rowcolor{xgreen!35}3 & $\left[-1,0 \right]$ & $\left[-1,0 \right]$\\
%		\rowcolor{xorange!35}4 & $\left[ 0,1 \right]$ & $\left[-1,0 \right]$\\
%		\bottomrule
%	\end{tabular}
%\end{table}

