\pgfkeys{
	/pgfplots/complex plane/.style={
	axis x line=middle,
	axis y line=middle,
	xlabel=$\Re$,
	ylabel=$\Im$,
	every axis x label/.style={
		at={(ticklabel* cs:1.02)},
		anchor=west,
	},
	every axis y label/.style={
		at={(ticklabel* cs:1.02)},
		anchor=south,
	},
	axis line style={stealth-stealth, thick},
	label style={font=\large},
	tick label style={font=\large},
	samples=100,
	xmin=-3, xmax=3,
	ymin=-3, ymax=3,
	domain=-3:3,
	grid=both,
	major grid style={black!5},
}}

\section{Complex Numbers}
\subsection{Algebraic approach}
Real numbers, while being extremely useful, are not complete - they can't solve all equations involving numbers. For example, the equation
\begin{equation}
	x^{2} + 1 = 0
	\label{eq:no_real_solutions}
\end{equation}
has no real solutions, since there can be no real number $x$ such that $x^{2}=-1$. However, we can choose to define a new number, $i=\sqrt{-1}$ and using it to build a new number system. This system is of course the set of complex numbers, $\mathbb{C}$. It is defined as the set of all $z$ such that
\begin{equation}
	z = a+ib,
	\label{eq:complex_number}
\end{equation}
where $a,b\in\mathbb{R}$ and $i=\sqrt{-1}$. We call $a$ the \emph{real component} of $z$ or $\Re(z)$, and $b$ its \emph{imaginary component} or $\Im(z)$\footnote{There is nothing more ``real'' about real numbers than imaginary numbers, but unfortunately that's the terminology we're stuck with \shrug}. These numbers appear a lot all throughout the exact sciences (but especially in physics and engineering), so we must at the very least learn their basic properties.

It is not so obvious that we can add two different kinds of numbers together, but it works (the linear algebra chapter sheds more light on this idea). What is important is that we always keep these two parts separated. We see this when we add together two complex numbers $z_{1},z_{2}$:
\begin{equation}
	z = z_{1}+z_{2} = \left( a_{1}+b_{1}i \right) + \left( a_{2}+b_{2}i \right) = \left( a_{1}+a_{2} \right) + \left( b_{1}+b_{2} \right)i.
	\label{eq:complex_addition}
\end{equation}
The real part of $z$ is therefore $a_{1}+b_{1}$, and its imaginary part is $b_{1}+b_{2}$.

What happens when we multiply two complex numbers? Let's check:
\begin{align}
	z = z_{1}z_{2} &= \left( a_{1}+b_{1}i \right)\left( a_{2}+b_{2}i \right)\nonumber\\
	&= a_{1}a_{2} + ia_{1}b_{2} + ia_{2}b_{1} + i^{2}b_{1}b_{2}\nonumber\\
	&= a_{1}a_{2} + ia_{1}b_{2} + ia_{2}b_{1} - b_{1}b_{2}\nonumber\\
	&= \left( a_{1}a_{2} - b_{1}b_{2} \right) + i\left( a_{1}b_{2} + a_{2}b_{1} \right).
	\label{eq:complex_product}
\end{align}
We see that we can still separate the real part and imaginary part of the result. What happens in the case of two real numbers? For real numbers $b=0$, and thus \eqref{complex_product} devolves to $z=a_{1}a_{2}\in\mathbb{R}$, which is exactly what we expect: multiplying two real numbers yields their product, which is a real number. Notice that this doesn't happen with purely imaginary numbers: multiplying together two imaginary numbers (i.e. numbers for which $a=0$) results in a real number. Will get to understand why this happens very soon.

When discussing real numbers sometimes we like to refer to their \textit{magnitude}, i.e. their absolute value. With complex numbers this is defined as
\begin{equation}
	|z| = \sqrt{a^{2}+b^{2}},
	\label{eq:complex_magnitude}
\end{equation}
i.e. in a sense, to get the magnitude of a complex number we imagine its two components as being perpendicular and calculate the length of the resulting hypotenous (cf.\ the Pythagorean theorem). In fact, this is one very useful interpertation of complex numbers, which we will explore in depth in the next subsection.

A very important operation that can be applied to complex numbers is \emph{conjugation}. The conjugate of a complex number $z=a+bi$ is defined as
\begin{equation}
	\conj{z} = a-bi,
	\label{eq:complex_conjugation}
\end{equation}
i.e. conjugating a number is simply negating its imaginary part. When we multiply a complex number by its own complex conjugate we get
\begin{equation}
	z\conj{z} = (a+bi)(a-bi) = a^{2} + abi - abi - b^{2}i^{2} = a^{2}+b^{2},
	\label{eq:conjugate_product}
\end{equation}
i.e. $z\conj{z} = |z|^{2}$. The inverse of a complex number can be expressed as
\begin{equation}
	z^{-1} = \frac{\conj{z}}{|z|^{2}}.
	\label{eq:complex_inverse}
\end{equation}

\subsection{Geometric approach}
As alluded to in the previous subsection, we can interpret a complex number $z=a+bi$ as two components in a 2-dimensional space (called the \emph{complex plane}), in which the horizontal axis represents real components, and the vertical access represents imaginary components:
\begin{figure}[H]
	\centering
	\begin{tikzpicture}
		\tikzstyle{every node}=[font=\large]
		\pgfmathsetmacro{\a}{2}
		\pgfmathsetmacro{\b}{2.5}
		\pgfmathsetmacro{\t}{atan2(\b,\a)}
		\begin{axis}[
			complex plane,
			width=10cm, height=10cm,
			xtick={-3,-2.5,...,3},
			xticklabels={},
			extra x ticks={\a},
			extra x tick labels={$a$},
			ytick={-3,-2.5,...,3},
			yticklabels={},
			extra y ticks={\b},
			extra y tick labels={$b$},
		]
		\draw[very thick, xred] (0,0) -- node[midway, above, rotate=\t] {$|z|=\sqrt{a^{2}+b^{2}}$} (\a,\b) node[complex](zdot){};
		\node[xred, above of=zdot, yshift=-6mm] {$z=a+bi$};
		\end{axis}
	\end{tikzpicture}
	\caption{Text.}
	\label{fig:complex number}
\end{figure}

Drawing a line from $z$ to $a$ (on the real axis) creates a right triangle. We can then define $\theta$ to be the angle near the origin and $r$ the length of the hypotenous:
\begin{figure}[H]
	\centering
	\begin{tikzpicture}
		\tikzstyle{every node}=[font=\large]
		\pgfmathsetmacro{\a}{2}
		\pgfmathsetmacro{\b}{2.5}
		\pgfmathsetmacro{\t}{atan2(\b,\a)}
		\begin{axis}[
			complex plane,
			width=10cm, height=10cm,
			xtick={-3,-2.5,...,3},
			xticklabels={},
			extra x ticks={\a},
			extra x tick labels={$a$},
			ytick={-3,-2.5,...,3},
			yticklabels={},
			extra y ticks={\b},
			extra y tick labels={$b$},
		]
		\draw[thick, dashed] (\a,\b) -- (\a,0);
		\draw[thick] ({\a-0.25},0) -- ({\a-0.25},0.25) -- (\a,0.25);
		\draw[very thick, xpurple] (0.75,0) arc (0:\t:0.75);
		\node[xpurple] at ({cos(\t/2)},{sin(\t/2)}) {$\theta$};
		\draw[very thick, xred] (0,0) -- node[midway, above, rotate=\t] {$r$} (\a,\b) node[complex](zdot){};
		\node[xred, above of=zdot, yshift=-6mm] {$z$};
		\end{axis}
	\end{tikzpicture}
	\caption{Text.}
	\label{fig:complex number 2}
\end{figure}
We call $r$ the \emph{magnitude} of $z$, and $\theta$ its \emph{argument}.

Using \eqref{xy_P} the real and imaginary components of $z$ are
\begin{align}
	a &= r\cos(\theta),\nonumber\\
	b &= r\sin(\theta),
	\label{eq:complex_components}
\end{align}
and $z$ can be re-written as
\begin{equation}
	z = r\left( \cos(\theta) + i\sin(\theta) \right).
	\label{eq:complex_geometric_form}
\end{equation}
Which we call the \emph{polar form} of $z$ (contrasted with $z=a+ib$ being the \emph{Cartesian form} of $z$).

Inverting the relations in \eqref{complex_components} yields the relations
\begin{align}
	r &= a^{2}+b^{2},\nonumber\\
	\theta &= \arctan\left(\frac{b}{a}\right).
	\label{eq:complex_components_geometric}
\end{align}

Let's examine the same properties of complex numbers shown in Equations \ref{eq:complex_addition}, \ref{eq:complex_product} and \ref{eq:complex_conjugation}, and verify that they work in the polar form of complex numbers. We start with addition (\eqref{complex_addition}):
\begin{align}
	z_{1}+z_{2} &= r_{1}\left[ \cos\left( \theta_{1} \right) + i\sin\left( \theta_{1} \right) \right] + r_{2}\left[ \cos\left( \theta_{2} \right) + i\sin\left( \theta_{2} \right) \right]\nonumber\\
	&= \underbrace{r_{1}\cos\left( \theta_{1} \right)}_{a_{1}} + \underbrace{r_{2}\cos\left( \theta_{2} \right)}_{a_{2}} + i\underbrace{r_{1}\sin\left( \theta_{1} \right)}_{b_{1}} + i\underbrace{r_{2}\sin\left( \theta_{2} \right)}_{b_{2}}\nonumber\\
&= \left( a_{1}+a_{2} \right) + i\left( b_{1}+b_{2} \right).
	\label{eq:complex_addition_geometric}
\end{align}
We see that indeed, the polar form of complex numbers adheres to the addition rule in \eqref{complex_addition}. Next is the product rule:
\begin{align}
	z_{1}z_{2} &= r_{1}\left[ \cos\left(\theta_{1}\right) + i\sin\left(\theta_{1}\right) \right] \cdot r_{2}\left[ \cos\left(\theta_{2}\right) + i\sin\left(\theta_{2}\right) \right]\nonumber\\
	&= r_{1}r_{2}\left[ \cos\left( \theta_{1} \right)\cos\left( \theta_{2} \right) + i\cos\left( \theta_{1} \right)\sin\left( \theta_{2} \right) + i\sin\left( \theta_{1} \right)\cos\left( \theta_{2} \right) -\sin\left( \theta_{1} \right)\sin\left( \theta_{2} \right)  \right]\nonumber\\
	&= r_{1}\cos\left( \theta_{1} \right)r_{2}\cos\left( \theta_{2} \right) - r_{1}\sin\left( \theta_{1} \right)r_{2}\sin\left( \theta_{2} \right) + i\left[ r_{1}\cos\left( \theta_{1} \right)r_{2}\sin\left( \theta_{2} \right) + r_{1}\sin\left( \theta_{1} \right)r_{2}\cos\left( \theta_{2} \right) \right]\nonumber\\
	&= \left( a_{1}a_{2}-b_{1}b_{2} \right) + i\left( a_{1}b_{2} + a_{2}b_{1} \right),
	\label{eq:complex_product_geometric}
\end{align}
which is indeed the result seen in \eqref{complex_product}. We can also develope further the second row of \eqref{complex_product_geometric} using some trigonometry (specifically the trigonometric identities xx and yy):
\begin{align}
	z_{1}z_{2} &= r_{1}r_{2}\left[ \cos\left( \theta_{1} \right)\cos\left( \theta_{2} \right) + i\cos\left( \theta_{1} \right)\sin\left( \theta_{2} \right) + i\sin\left( \theta_{1} \right)\cos\left( \theta_{2} \right) -\sin\left( \theta_{1} \right)\sin\left( \theta_{2} \right) \right]\nonumber\\
	&= r_{1}r_{2}\left[ \cos\left( \theta_{1} \right)\cos\left( \theta_{2} \right)-\sin\left( \theta_{1} \right)\sin\left( \theta_{2} \right) + i\left[ \cos\left( \theta_{1} \right)\sin\left( \theta_{2} \right) + \sin\left( \theta_{1} \right)\cos\left( \theta_{2} \right) \right] \right]\nonumber\\
	&= r_{1}r_{2}\left[ \cos\left( \theta_{1}+\theta_{2} \right) + i\sin\left( \theta_{1}+\theta_{2} \right)  \right].
	\label{eq:complex_product_geometric_insight}
\end{align}
This is a very important result: it shows that multiplying a complex number $z_{1}$ by another complex number $z_{2}$ gives a complex number with magnitude $r_{1}r_{2}$, i.e. the product of the magnitudes of the two complex numbers, and argument $\theta_{1}+\theta_{2}$, i.e. the argument of $z_{1}$ rotated by the argument of $z_{2}$ (or vice-versa). We will consider this result in more detail soon.


In the polar form the complex conjugate of a number $z=r\left[ \cos\left( \theta \right) +i\sin\left( \theta \right) \right]$ can be brought about by substituting $-\theta$ into the arguments of the trigonometric functions:
\begin{align}
	\conj{z} &= r\left[ \cos\left( -\theta \right) + i\sin\left( -\theta \right) \right]\nonumber\\
	&= r\left[ \cos\left( \theta \right) - i\sin\left( \theta \right) \right]\nonumber\\
	&= r\cos\left( \theta \right) - ir\sin\left( \theta \right)\nonumber\\
	&= a-ib.
	\label{eq:complex_conjugation_geometric}
\end{align}

Lastly, let's show that \eqref{conjugate_product} can be derived in the polar form:
\begin{align}
	z\conj{z} &= r\left[ \cos\left( \theta \right) + i\sin\left( \theta \right) \right] \cdot r\left[ \cos\left( \theta \right) - i\sin\left( \theta \right) \right]\nonumber\\
	&= r^{2}\left[ \cos^{2}\left( \theta \right) -\cancel{i\cos\left( \theta \right)\sin\left( \theta \right)} +i\cancel{\sin\left( \theta \right)\cos(\theta)} + \sin^{2}\left( \theta \right) \right]\nonumber\\
	&= r^{2}\left[ \sin^{2}\left( \theta \right) + \cos^{2}\left( \theta \right) \right]\nonumber\\
	&= r^{2} = a^{2}+b^{2}.
	\label{eq:conjugate_product_geometric}
\end{align}

In 1748 Leonhard Euler published his famous work \textit{Introductio in analysin infinitorum}\footnote{Latin for \textbf{Introduction to the Analysis of the Infinite}.}. In it he introduced the following relation, called \emph{Euler's formula}:
\begin{equation}
	e^{ix} = \sin(x) + i\cos(x).
	\label{eq:Euler's_formula}
\end{equation}
Using Euler's formula a complex number $z$ can be written as
\begin{equation}
	z = re^{i\theta}.
	\label{eq:complex number using Eurler's formula}
\end{equation}

In \tabref{complex_exponentials} we can see some usefull complex exponentials $e^{xi}$. Specifically, setting $x=\pi$ yields the famous \emph{Eurler's identity}, considered by many to be one of the most beautiful equations in mathematics, as it binds together five important numbers, namely $0,1,\pi,e$ and $i$:
\begin{equation}
	e^{\pi i} + 1 = 0.
	\label{eq:Euler's identity}
\end{equation}

\begin{table}[H]
	\centering
	\caption{Values of $e^{ix}$ for some useful values of $x$ (cf. xxxx for the values of $\sin(\theta)$ and $\cos(\theta)$).}
	\label{tab:complex_exponentials}
	\begin{tabular}{lccl}
		\toprule
		$x$ & $\cos(x)$ & $\sin(x)$ & $z=e^{ix}$\\
		\midrule
		$\frac{\pi}{2}$ & $0$ & $1$ & $i$\\
		$\pi$ & $-1$ & $0$ & $-1$\\
		$\frac{3\pi}{2}$ & $0$ & $-1$ & $-i$\\
		$\frac{\pi}{3}$ & $\frac{1}{2}$ & $\frac{\sqrt{3}}{2}$ & $\frac{1}{2}\left( 1+i\sqrt{3} \right)$\\
		$\frac{\pi}{4}$ & $\frac{\sqrt{2}}{2}$ & $\frac{\sqrt{2}}{2}$ & $\frac{\sqrt{2}}{2}\left( 1+i \right)$\\
		$\frac{\pi}{6}$ & $\frac{\sqrt{3}}{2}$ & $\frac{1}{2}$ & $\frac{1}{2}\left( \sqrt{3}+i \right)$\\
		\bottomrule
	\end{tabular}
\end{table}

\tabref{complex_exponentials} also shows us the integer behaviours of $i$:
\begin{equation}
	i^{2}=-1,\ i^{3}=-i, i^{4}=1,\ i^{5}=i,\ i^{6}=-1,\ i^{7}=-i,\ \dots
	\label{eq:powers of i}
\end{equation}

\subsection{Complex roots of polynomial functions}
We began this section discussing an equation that can be solved over the real numbers (\eqref{no_real_solutions}). Let us now use similar equations to find roots of complex numbers, focusing on equtions of the type
\begin{equation}
	z^{n}=1,
	\label{eq:power equation complex}
\end{equation}
where $n\in\mathbb{N}$. Over the real numbers, depending on the parity of $n$ (i.e. whether it is even or odd) there are either two solutions (namely $z_{1,2}=\pm1$) or a single solution (namely $z=1$). However over the complex numbers there are always exactly $n$ solutions to the equation. Since we already know the solutions to $z^{2}=1$, we move on to $z^{3}=1$. One solution to the equation is pretty obvious: $z_{1}=1$. To find the other two solutions we switch to the polar form, in which $1$ has magnitude $r=1$ and argument $\theta=0$, i.e.
\begin{equation}
	1 = \cos(0) + i\sin(0).
	\label{eq:bla}
\end{equation}

We saw earlier that multiplying two complex numbers $z_{1}z_{2}$ together results in a number which has magnitude that is the product of the magnitudes of $z_{1}$ and $z_{2}$, and an argument which is the sum of the arguments of $z_{1}$ and $z_{2}$ (\eqref{complex_product_geometric_insight}). This means that the two remaining solutions to $z^{3}=1$ must have magnitude $r=1$, and arguments $\theta$ such that $3\theta=0$. Recall however that we are dealing with a circle of radius $1$, meaning that there are infinitely many angles which corresponds to a full circle, namely $\theta=2\pi k$ where $k\in\mathbb{Z}$. There are exactly two angles in $[0,2\pi)$ which when multipled by $3$ result in a complete circle: $\theta_{2}=\frac{2\pi}{3}$ and $\theta_{3}=\frac{4\pi}{3}$.

Therefore, the three solutions to $z^{3}=1$ are (see \figref{third-roots of 1}):
\begin{align}
	z_{1} &= \cos(0) + i\sin(0),\nonumber\\
	z_{2} &= \cos\left( \frac{2\pi}{3} \right) + i\sin\left( \frac{2\pi}{3} \right),\nonumber\\
	z_{3} &= \cos\left( \frac{4\pi}{3} \right) + i\sin\left( \frac{4\pi}{3} \right),
	\label{eq:third-roots of 1, polar form}
\end{align}
which correspond to the following Cartesian forms:
\begin{align}
	z_{1} &= 1,\nonumber\\
	z_{2} &= -\frac{1}{2}+\frac{\sqrt{3}}{2},\nonumber\\
	z_{3} &= -\frac{1}{2}-\frac{\sqrt{3}}{2}.
	\label{eq:third-roots of 1, Cartesian form}
\end{align}

\begin{figure}[H]
	\centering
	\begin{tikzpicture}
		\tikzstyle{every node}=[font=\large]
		\begin{axis}[
			complex plane,
			width=10cm, height=10cm,
			xmin=-1.25, xmax=1.25,
			ymin=-1.25, ymax=1.25,
			xtick={-1.5,-1,...,1.5},
			ytick={-1.5,-1,...,1.5},
		]
		\coordinate (O) at (0,0);
		\coordinate (z1) at (1,0);
		\coordinate (z2) at (-0.5,{sqrt(3)/2});
		\coordinate (z3) at (-0.5,{-sqrt(3)/2});
		\draw[black!20] (z1) arc (0:360:1);
		\draw[xblue] (0.25,0) arc (0:120:0.25);
		\draw[xgreen] (0.2,0) arc (0:240:0.2);
		\draw[very thick, xred] (O) -- (z1) node[complex, label=above:$z_{1}$]{};
		\draw[very thick, xblue] (O) -- (z2) node[complex, label=above left:$z_{2}$]{};
		\draw[very thick, xgreen] (O) -- (z3) node[complex, label=below left:$z_{3}$]{};
		\node[xred] at (0.5,0.1) {$\theta_{1}=0$};
		\node[xblue, rotate=50] at ({cos(60)/2.5+0.1},{sin(60)/2.5+0.1}) {$\theta_{2}=\frac{2\pi}{3}$};
		\node[xgreen] at (-0.5,0.1) {$\theta_{3}=\frac{4\pi}{3}$};
		\end{axis}
	\end{tikzpicture}
	\caption{The three complex solutions to $z^{3}=1$.}
	\label{fig:third-roots of 1}
\end{figure}

We can verify that $z_{2}$ indeed solves the equation\footnote{Using $(a+b)^{3}=a^{3}+3a^{2}b+3ab^{2}+b^{3}$.}:
\begin{align}
	\left( -\frac{1}{2}+\frac{\sqrt{3}}{2} \right)^{3} &= -\frac{1}{8} + 3\frac{\sqrt{3}}{8}i - 3\frac{3}{8}i^{2} + \frac{3\sqrt{3}}{8}i^{3}\nonumber\\
	&= -\frac{1}{8} + \cancel{3i\frac{\sqrt{3}}{8}} + 3\frac{3}{8} - \cancel{3i\frac{\sqrt{3}}{8}}\nonumber\\
	&= -\frac{1}{8} + \frac{9}{8}\nonumber\\
	&= 1.
	\label{eq:bla}
\end{align}
Verfying that $z_{3}$ solves the equation as well is left as an excercise to the reader.
