\section{Linear functions} % (fold)
\label{sec:Linear functions}
% section Linear functions (end)

One simple family of functions are the \emph{linear functions}. These are functions of the form
\begin{equation}
  f(x) = mx+b,
  \label{eq:linear_funcs_def}
\end{equation}
where $m$ and $b$ are both real numbers. We call $m$ the \emph{slope} of $f$, while $b$ is the \emph{free variable} of $f$. The slope of a linear function is simply the change in its value as we increase $x$, and is constant everywhere: for example, take the function $f(x)=2x-1$. Its slope is $m=2$ and the free variable is $b=-1$ (see \autoref{fig:2x-1}). At $x=2$, the function has the value $f(2)=4-1=3$. Increasing $x$ by $\Delta x=3$ to $x=5$ gives $f(5)=10-1=9$, an increase of $\Delta y=9-3=6$ i.e. $2\Delta x$. At $x=8$, another $\Delta x=3$, the function has the value $f(8)=16-1=15$ - again, an increase of $\Delta y=2\Delta x=6$.

\begin{figure}
  \centering
  \begin{tikzpicture}
    \begin{axis}[
      graph2d,
      width=12cm, height=12cm,
      xmin=-4, xmax=4,
      ymin=-4, ymax=4,
      domain={-4:4},
    ]
    \addplot[function, xred] {2*x-1} node[pos=0.4, above, rotate=63] {$f(x)=2x-1$};
    \draw[very thick, decorate, decoration={brace, amplitude=3pt, mirror, raise=5pt}]
         (1,1) -- (2,1) node[midway, below, yshift=-5pt] {$\Delta x=1$};
    \draw[very thick, decorate, decoration={brace, amplitude=3pt, raise=5pt}]
         (1,1) -- (1,3) node[midway, above, rotate=90, yshift=5pt] {$\Delta y=2$};
     \draw[thick, dashed] (1,3) -- (2,3);
     \draw[thick, dashed] (2,1) -- (2,3);
    \end{axis}
  \end{tikzpicture}
  \caption{The function $f(x)=2x-1$.}
  \label{fig:2x-1}
\end{figure}

We can therefore write
\begin{equation}
  m = \frac{\Delta y}{\Delta x}.
  \label{eq:linear_slope_def}
\end{equation}

On the other hand, the free variable $b$ is the value of the function at $x=0$, since then $mx=0$:
\begin{equation}
  b = f(0).
  \label{eq:linear_free_var}
\end{equation}

Two linear functions with the same slope are always parallel, while two linear functions with the same free variable meet at $x=0$.

From a geometric point of view, the slope of a linear function is
\begin{equation}
  m = \tan\left(\theta\right),
  \label{eq:slope_angle}
\end{equation}
where $\theta$ is the angle between the $x$-axis and the line represented by the function. We will discuss trigonometric functions in \autoref{sec:trigonometry}.
