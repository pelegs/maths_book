\section{Exponential and Logarithmic Functions}
In the previous section we dealt with functions composed of integer powers of $x$. We will now shortly focus on functions where $x$ is in the power itself and their inverse functions.

An \emph{exponential function}, or simply an \emph{exponential}, is a real function of the type
\begin{equation}
	f(x) = b^{x},
	\label{eq:exponent}
\end{equation}
where $b>0$ is called the \emph{base} of the exponentiation, and $x$ the exponential. All exponents, regardless of base, are always positive. In addition, all exponents pass through the point $(0,1)$ since $b^{0}=1$ for any real positive number, and through the point $(1,b)$ since $b^{1}=b$. When $b>1$ the function is increasing on $\mathbb{R}$, while for $b<1$ the function is descending on $\mathbb{R}$.

\begin{example}{Exponential functions}{exponents}
	The following are graphs of the exponential functions \textcolor{xred}{$\bm{1.5^{x}}$}, \textcolor{xblue}{$\bm{2^{x}}$} and \textcolor{xgreen}{$\bm{3.5^{x}}$}:
	\begin{figure}[H]
		\centering
		\begin{tikzpicture}
			\begin{axis}[
					graph2d,
					y axis line style={-stealth, thick},
					width=10cm, height=10cm,
					xmin=-5, xmax=3,
					ymin=0, ymax=4,
					domain=-5:3,
					restrict y to domain=0:5,
				]
				\addplot[function, xred] {(1.5)^\x};
				\addplot[function, xblue] {2^\x};
				\addplot[function, xgreen] {3.5^\x};
			\end{axis}
		\end{tikzpicture}
	\end{figure}

	And the following are graphs of the exponential functions \textcolor{xpurple}{$\bm{0.7^{x}}$}, \textcolor{xorange}{$\bm{0.5^{x}}$} and $\bm{0.2^{x}}$:
	\begin{figure}[H]
		\centering
		\begin{tikzpicture}
			\begin{axis}[
					graph2d,
					y axis line style={-stealth, thick},
					width=10cm, height=10cm,
					xmin=-5, xmax=3,
					ymin=0, ymax=4,
					domain=-5:3,
					restrict y to domain=0:5,
				]
				\addplot[function, xpurple] {0.7^\x};
				\addplot[function, xorange] {0.5^\x};
				\addplot[function, black] {0.2^\x};
			\end{axis}
		\end{tikzpicture}
	\end{figure}
\end{example}

As a reminder, the following are two well known properties of exponents: given a base $b>0$,
\begin{align}
	b^{-x} &= \frac{1}{b^{x}},\\
	b^{x}b^{y} &= b^{x+y}.
	\label{eq:exponents_properties}
\end{align}

A special base for exponential functions is the real, non-aglebric number $e$. This number has many names, among them is \emph{Euler's number}, but in the constant of exponentials it is known as the \emph{natural base}. Its exact value is not entirely important for the moment: it is about $2.718$, and in any case it is not possible to write it as there it has infinitely many digits after the period. It is very common across different fields of mathematics and science to write $\exp(x)$ instead of $e^{x}$.

The inverse function to exponentials are the \emph{logatithmic functions} (or simply \emph{logarithms}), i.e.\ for any real $b>0,\ b\neq1$,
\begin{equation}
	\log_{b}\left( b^{x} \right) = b^{\log_{b}(x)} = x.
	\label{eq:logarithms}
\end{equation}
In essence, the logarithm in base $b$ of a number $x$ answers the question \textit{``what is the number $a$ for which $b^{a}=x$?''}. Being the inverses of exponential functions, all logarithms go through the point $(1,0)$, and each also passes through its own point $(b,1)$.

\begin{example}{Logarighmic functions}{logarithms}
	The following are graphs of the logatithmic functions \textcolor{xred}{${\log_{1.5}(x)}$}, \textcolor{xblue}{${\log_{2}(x)}$} and \textcolor{xgreen}{${\log_{3.5}(x)}$}:
	\begin{figure}[H]
		\centering
		\begin{tikzpicture}
			\begin{axis}[
					graph2d,
					x axis line style={-stealth, thick},
					width=10cm, height=10cm,
					xmin=0, xmax=4,
					ymin=-5, ymax=3,
					domain=0:4,
					restrict y to domain=-10:4,
				]
				\addplot[function, xred] {ln(\x)/ln(1.5)};
				\addplot[function, xblue] {ln(\x)/ln(2)};
				\addplot[function, xgreen] {ln(\x)/ln(3.5)};
			\end{axis}
		\end{tikzpicture}
	\end{figure}

	...and the following are graphs of the exponential functions \textcolor{xpurple}{$\log_{0.75}(x)$}, \textcolor{xorange}{$\log_{0.5}(x)$} and $\log_{0.2}(x)$:
	\begin{figure}[H]
		\centering
		\begin{tikzpicture}
			\begin{axis}[
					graph2d,
					x axis line style={-stealth, thick},
					width=10cm, height=10cm,
					xmin=0, xmax=4,
					ymin=-3, ymax=5,
					domain=0:4,
					restrict y to domain=-4:10,
				]
				\addplot[function, xpurple] {ln(\x)/ln(0.75)};
				\addplot[function, xorange] {ln(\x)/ln(0.5)};
				\addplot[function, black] {ln(\x)/ln(0.2)};
			\end{axis}
		\end{tikzpicture}
	\end{figure}
\end{example}

A useful property of logarithms is that they can help reduce ranges spanning several orders of magnitude to numebrs humans can deal with. The easiest way to see this is using $b=10$: $10^{1}=10$, and so $\log_{10}(10)=1$. $10^{2}=100$, and so $\log_{10}(100)=2$. $10^{3}=1000$, and so $\log_{10}(1000)=3$, etc. The value of the logarithm goes by $1$ for each raise in order of magnitude of its argument.

Therefore, if we have some measurement $x$ which can hold values spanning severl orders of magnitude (say $x\in[3,1500000000]$), then it can sometimes be useful to use instead the logarithmic value of $x$ (which in our case would span the range $\log_{10}(x)\in[0.477,9.176]$). This is done in many fields of science, for example some definitions of entropy\footnote{$S=k_{\text{B}}\log\left( \Omega \right)$}, acid disociation constants\footnote{$\text{p}K_{a}=-\log\left( K_{\text{diss}} \right)$}, pH\footnote{$\text{pH}=-\log\left(\ce{[H+]}\right)$} and more.

\begin{example}{Logarithms as eavluating orders of magnitude}{}
	In the following graph of $\log_{2}(x)$, each increase by power of two in $x$ (i.e. $x=1,2,4,8,16,\dots$) yields only a single increase in $y$ (i.e. $y=0,1,2,3,4,\dots$). This shows how logarithms shift our perspective from absolute values to orders of magnitude.
	\begin{figure}[H]
		\centering
		\begin{tikzpicture}
			\begin{axis}[
					graph2d,
					x axis line style={-stealth, thick},
					width=10cm, height=8cm,
					xmin=0, xmax=16,
					ymin=-4, ymax=8,
					domain=0:16,
					restrict y to domain=-7:10,
					grid=major,
					xtick={0,1,2,4,8,16},
					ytick={-4,-3,...,8},
				]
				\addplot[function, xpurple] {ln(\x)/ln(2)};
			\end{axis}
		\end{tikzpicture}
	\end{figure}
\end{example}

Using the definition of the logarithmic function $\log_{b}(x)$ (\eqref{logarithms}) and the product rule for exponentials (\eqref{exponents_properties}), a similar rule can be derived for logarithms. Let $x,y>0$ and $b>0,\ b\neq1$ all be real numbers. We define
\begin{equation}
	\log_{b}(x)=M,\ \log_{b}(y)=N,
	\label{eq:log_addition_rule_step1}
\end{equation}
which means
\begin{equation}
	b^{M}=x,\ b^{N}=y.
	\label{eq:log_addition_rule_step2}
\end{equation}
From \eqref{exponents_properties} we know that
\begin{equation}
	xy = b^{M}b^{N} = b^{M+N},
	\label{eq:log_addition_rule_step3}
\end{equation}
and by re-applying the definition of logarithmic functions we get that
\begin{equation}
	\log_{b}(xy) = M+N = \log_{b}(x) + \log_{b}(y).
	\label{eq:log_addition_rule_step4}
\end{equation}

Simlilarily to \eqref{log_addition_rule_step4}, division yields subtraction:
\begin{equation}
	\log_{b}\left(\frac{x}{y}\right) = \log_{b}(x)-\log_{b}(y).
	\label{eq:log_subtraction_rule}
\end{equation}

Equations \ref{eq:log_addition_rule_step4} and \ref{eq:log_subtraction_rule} reveal another valueable property of logarithms: they reduce multiplication to addition (and subsequently division to subtraction). While today this property doesn't seem very impressive, in pre-computers days it helped carrying on complicated calculations, using tables of pre-calculated logarithms (called simply \emph{logarithm tables}) - a sight rarely seen today.

Taking one step forward in regards to reduction of operations, logarithms reduce powers to multiplication:
\begin{equation}
	\log_{b}\left( x^{k} \right) = k\log_{b}(x).
	\label{eq:log_product_rule}
\end{equation}
for any $k\in\mathbb{R}$.

(TBW:\@ proving this will be in the chapter questions to the reader)

Any logarithm $\log_{b}(x)$ can be expressed using another base, i.e. $\log_{a}(x)$ (where $a>0,\ a\neq1$) using the following formula:
\begin{equation}
	\log_{a}(x) = \log_{b}(x)\cdot\log_{a}(b).
	\label{eq:log_base_change}
\end{equation}
(TBW:\@ proving this too will be a question to the reader)

\begin{example}{Changing logarithm base}{}
	Expressing $\log_{4}(x)$ in terms of $\log_{2}(x)$:
	\[
		\log_{4}(x) = \log_{2}(x)\cdot\underbrace{\log_{4}(2)}_{=\frac{1}{2}} = \frac{1}{2}\log_{2}(x).
	\]
\end{example}

Much like with exponentials, the number $e$ plays an important role when it comes to logarithms, for reasons that are discussed in the calculus chapter (ref). For now, we will just mention that $\log_{e}(x)$ gets a special notation: $\ln(x)$, which stands for \emph{natural logarithm}. This notation is mainly used in applied mathematics and science, while in pure mathematics the notation is simply $\log(x)$, i.e. without mentioning the base \footnote{Depending on convention and context, this notation can refer to logarithm in any other base, most commonly $\log_{10}(x)$ and $\log_{2}(x)$.}.

For reason we will see in the calculus chapter, it is relatively simple to calculate both the exponential and logarithm in base $e$. Therefore, many operations in modern computations are actually done using these functions, for example calculating logarithms in other bases:
\begin{equation}
	\log_{b}(x) = \frac{\ln(x)}{\ln(b)}.
	\label{eq:ln_base_change}
\end{equation}
Another operation commonly using both $e^{x}$ and $\ln(x)$ is raising a real number $a$ to a real power $b$: using the properties of both exponential and logarithmic functions, any such power can be expressed as
\begin{equation}
	a^{b} = e^{b\ln(a)}.
	\label{eq:powers_using_e}
\end{equation}
