\section{Conic sections}
\label{sec:Conic sections}
An important family of shapes in mathematics are the \emph{conic sections}, which arise as the intersection of a plane and an infinite ``right circular'' cone (let's call it \textit{RC cone} for simplicity). An RC cone is a 3D shape which is created when a right triangle is rotating by $\ang{360}$ around the triangle's heiht (see \autoref{fig:RC_cone}).

\begin{figure}
  \centering
  \begin{tikzpicture}
    % From here: https://tex.stackexchange.com/a/171172
    \fill[top color=xpurple!30, bottom color=xpurple!15, shading=axis, opacity=0.25]
      (0,0) circle (2cm and 0.5cm);
    \fill[left color=xpurple!30!black, right color=xpurple!30!black, middle color=xpurple!30, shading=axis, opacity=0.1]
      (2,0) -- (0,6) -- (-2,0) arc (180:360:2cm and 0.5cm);
    \draw[arcnode={5mm}{$\theta$}] (0,4) arc (-90:-70.1:2);
    \draw (-2,0) arc (180:360:2cm and 0.5cm) -- (0,6) -- cycle;
    \draw[dashed] (-2,0) arc (180:0:2cm and 0.5cm);
    \draw[dashed, fill=xblue, fill opacity=0.25, text opacity=1] (2,0) -- node[below] {$r$} (0,0) -- node[left] {h} (0,6) ;
    \draw (0,8pt) -- ++(8pt,0) -- (8pt,0);
    \draw[thick, xblue] (2.25,0) arc (0:-75:2cm and 0.75cm);
  \end{tikzpicture}
  \caption{Rotating a right triangle by $\ang{360}$ around it height $h$, yielding a right icrcular cone with radius $r$ and angle $\theta$.}
  \label{fig:RC_cone}
\end{figure}
