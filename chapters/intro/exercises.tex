\section{Exercises}
\subsection{Problems}
\begin{enumerate}
  \item Write the following sets explicitly:
    \begin{enumerate}[label=(\roman*)]
      \item $\left\{ x\in \mathbb{N}\mid1<x\leq7\right\}$
      \item $\left\{ x\in \mathbb{Z}\mid x<5\right\}$
      \item $\left\{ x\in \mathbb{R}\mid x^{2}=-1 \right\}$
      \item $\left\{ x\in \mathbb{N}\wedge x\in \mathbb{Q} \right\}$
      \item $\left\{ x\in \mathbb{R}\mid x^{2}-3x-4=0 \right\}$
      \item $\left\{ x\in\mathbb{R}\mid x<5\wedge x \geq 2\right\}$
    \end{enumerate}
  
	\item Determine the relation between the sets:
    \begin{enumerate}[label=(\roman*)]
      \item $A=\left\{ 1,\ 2,\ 3\right\},\ B=\left\{ 1,\ 2 \right\}$
      \item $A=\varnothing,\ B=\left\{ 2,\ -5,\ \pi \right\}$
      \item $A=\mathbb{Z},\ B=\left\{ \pm x \mid x\in\mathbb{N} \cup \right\{0\left\} \right\}$
			\item $A=\left\{\pi, \eu, \sqrt{2}\right\},\ B=\mathbb{Q}$
    \end{enumerate}

	\item Write all elements in $S^{2}\times W$, where $S=\{\alpha,\beta,\gamma\}$ and $W=\{x,y,z\}$. Find a condition that guarantees $S^{2}\times W = W\times S^{2}$.

	\item How many different injective functions $f:\{1,2\}\to\{1,2\}$ exist? How many injective functions $f:\{1,2,3\}\to\{1,2,3\}$ exist? How many inject functions $f:\{1,2,\dots,n\}\to\{1,2,\dots,n\}$ exist for a given $n\in\mathbb{N}$?

	\item For each of the real functions below, find a set on which it is surjective (use a graphing calculator if you are not familiar with the shape of a function):
		\[
			x^{2},\ x^{3}-5,\ \eu^{-x^{2}/2},\ \sin(x),\ \sin(x)+\cos(x),\ x\eu^{x}.
		\]

	\item Given two sets $A,B$ such that $|B|=|A|-1$, can a bijective function $f:A\to B$ exist? Explain your answer.
	\item MORE EXERCISES TO BE WRITTEN\ldots
\end{enumerate}

\subsection{Solutions}
\begin{enumerate}
	\item For each of the sets we first write how to read the notation in words, followed by its explicit form:
		\begin{enumerate}[label={(\roman*)}]
		\item Any \textbf{natural number} such that it is bigger than $1$ and smaller or equal to $7$. These are of course the numbers
			\[
				\left\{2,3,4,5,6,7\right\}.
			\]
		
		\item Any \textbf{integer} such that it is smaller than $5$. These are the numbers
			\[
				\left\{4,3,2,1,0,-1,-2,-3,\dots\right\}.
			\]
		
		\item Any \textbf{real number} $x$ such that $x^{2}=-1$. Since for any $x\in\mathbb{R},\ x^{2}\geq0$ - there is no such real number $x$ whose square equals $-1$. Therefore this definition describes the empty set, i.e. $\emptyset$.
		
		\item Any \textbf{natural number} that is also a \text{rational number}. Since any natural number is also a rational number (e.g. $4=\frac{4}{1}=\frac{8}{2}$, etc.) the definition actually simply describes the set of natural numbers, $\mathbb{N}$. This fact can also be written as
			\[
				\mathbb{N} \cap \mathbb{Q} = \mathbb{N}.
			\]
		
		\item Any \textbf{real number} such that it solves the equation $x^{2}-3x-4=0$. The solutions can be found using the quadratic formula:
			\[
				\frac{3\pm\sqrt{3^{2}+4\cdot4}}{2} = \frac{3\pm\sqrt{9+16}}{2} = \frac{3\pm5}{2} = 4,-1.
			\]
			Therefore the set described by the definition is simply
			\[
				\left\{4,-1\right\}.
			\]
		
		\item Any \textbf{real number} that is smaller than $5$ \textbf{and} is bigger than or equal to $2$. This definition describes the half-open interval
			\[
				[2,5).
			\]
	\end{enumerate}
\item Bla
\end{enumerate}
