\section{Exercises}
\subsection{Problems}
\begin{enumerate}
  \item Write the following sets explicitly:
    \begin{enumerate}[label=(\roman*)]
      \item $\left\{ x\in \mathbb{N}\mid1<x\leq7\right\}$
      \item $\left\{ x\in \mathbb{Z}\mid x<5\right\}$
      \item $\left\{ x\in \mathbb{R}\mid x^{2}=-1 \right\}$
      \item $\left\{ x\in \mathbb{N}\wedge x\in \mathbb{Q} \right\}$
      \item $\left\{ x\in \mathbb{R}\mid x^{2}-3x-4=0 \right\}$
      \item $\left\{ x\in\mathbb{R}\mid x<5\wedge x \geq 2\right\}$
    \end{enumerate}
  
	\item Determine the relation between the sets:
    \begin{enumerate}[label=(\roman*)]
      \item $A=\left\{ 1,\ 2,\ 3\right\},\ B=\left\{ 1,\ 2 \right\}$
      \item $A=\varnothing,\ B=\left\{ 2,\ -5,\ \pi \right\}$
      \item $A=\mathbb{Z},\ B=\left\{ \pm x \mid x\in\mathbb{N} \cup \right\{0\left\} \right\}$
			\item $A=\left\{\pi, \eu, \sqrt{2}\right\},\ B=\mathbb{Q}$
    \end{enumerate}

	\item Write all elements in $S^{2}\times W$, where $S=\{\alpha,\beta,\gamma\}$ and $W=\{x,y,z\}$. Find a condition that guarantees $S^{2}\times W = W\times S^{2}$.

	\item How many different \textbf{bijective} functions exist between a set with $2$ elements and another set with $2$ elements (e.g. $f:\{1,2\}\to\{\alpha,\beta\}$)? How many exist between two sets, each with $3$ elements? Between two sets each with $n$ elements?

	\item For each of the real functions below, find a set on which it is surjective (use a graphing calculator if you are not familiar with the shape of a function):
		\[
			x^{2},\ x^{3}-5,\ \eu^{-x^{2}/2},\ \sin(x),\ \sin(x)+\cos(x),\ x\eu^{x}.
		\]

	\item Given two sets $A,B$ such that $|A|\neq|B|$, can a bijective function $f:A\to B$ exist? Explain your answer.

	\item Find all real roots of the following polynomial function:
		\[
			f(x) = x^{3} + x^{2} - 6x.
		\]

	\item Given a real $b>0$ and $k$, prove that for any real $x>0$
		\[
			\log_{b} \left( x^{k} \right) = k\log_{b} \left( x \right).
		\]

	\item Show that for any positive real $x,b$
		\[
			\log_{b} \left( \frac{1}{x} \right) = -\log_{b}(x).
		\]

	\item Solve the following equation for any real $x>0$: (CHECK SOLUTION! MIGHT BE WRONG)
		\[
			\log_{2}(x) + \log_{4}(x-1) = \log_{16}\left(x^{3}\right).
		\]

	\item During the second age of Middle-earth, $20$ rings of power were forged. The following poem describes their distribution amongs the different peoples of the land:

% Three Rings for the Elven-kings under the sky,
% Seven for the Dwarf-lords in their halls of stone,
% Nine for Mortal Men doomed to die,
% One for the Dark Lord on his dark throne
% In the Land of Mordor where the Shadows lie.

	\item The horizon on a spherical planet such as the earth\footnote{yes.} is defined as the distance from an observer to the point where the ground disappears behind the planet's curve. Following is a 2-dimensional depiction, where $\textcolor{xblue}{\bm{O}}$ is the observer, $\bm{h}$ its height above the planet surface, $\textcolor{xblue}{\bm{X}}$ the horizon point and $\textcolor{xred}{\bm{d_{a}}}$ the air-distance from the observer to the horizon and $\textcolor{xgreen}{\bm{d_{g}}}$ the ground-distance from the observer to the horizon:

		\begin{center}
			\begin{tikzpicture}
				\pgfmathsetmacro{\R}{2}
				\pgfmathsetmacro{\h}{0.6}
				\pgfmathsetmacro{\th}{asin(sqrt(2*\R*\h+\h^2)/(\R+\h))}
				\pgfmathsetmacro{\ph}{90-\th}
				\coordinate (O) at (0,0);
				\coordinate (obs) at (0,{\R+\h});
				\coordinate (top) at (0,\R);
				\coordinate (horiz) at ({\R*cos(\ph)},{\R*sin(\ph)});
				\draw[very thick, fill=xblue!20] (O) circle (\R);
				\draw[ultra thick] (top) -- (obs) node [midway, left] {$h$};
				\draw[ultra thick, xred] (obs) -- (horiz) node [midway, above right] {$d_{a}$};
				\draw[ultra thick, xgreen] (horiz) arc (\ph:90:\R) node [midway, below, xshift=-1mm] {$d_{g}$};
				\fill[xblue] (obs) circle (0.07) node [above] {$O$};
				\fill[xblue] (horiz) circle (0.07) node [above right] {$X$};
			\end{tikzpicture}
		\end{center}

		\begin{enumerate}[label=(\roman*)]
			\item Find an expression for the air-distance $\textcolor{xred}{\bm{d_{a}}}$ and ground-distance $\textcolor{xgreen}{\bm{d_{g}}}$ to the horizon as a function of the radius $\bm{R}$ and height $\bm{h}$. (\textbf{hint}: find a relevant right triangle containing $\textcolor{xred}{\bm{d_{a}}}$ and the radius of the planet)
			\item Given that the Earth's radius is about $6371\si{km}$ ($6.371\times10^{6}\si{m}$) and an average person is $1.75\si{m}$ tall - what is the distance to the horizon for a person standing at sea-level (both air- and ground-distances)? What would these distances be at the following heights: $165\si{m}$ (Eiffel tower's observation deck), $9.1\si{km}$ (average cruising altitude of a passenger jet) and $408\si{km}$ (average altitude of the International Space Station)?
			\item How many degrees does the horizon drops from eye-level as function of $\bm{h}$? (\textit{eye-level} in this context means the direction tangent to the planet's surface)
		\end{enumerate}

	\item Calculate the following complex product - first using the algebraic form and then the polar form, showing that the result is the same in both cases:
		\[
			z = z_{1}^{2}z_{2} = \left( \sqrt{3}+\iu \right)^{2} \left( -2+\sqrt{12}\iu \right).
		\]

	\item Prove that the \textbf{sum} of all the roots of the complex equation $z^{n}=1$ is always zero when $n\geq2$, i.e. if $w_{0},w_{1},\dots,w_{n-1}$ are the roots of the equation, then
		\[
			\sum\limits_{k=0}^{n-1}w_{k} = 0.
		\]
		\textbf{Hint}: for $|r|\neq1$,
		\[
			\sum\limits_{k=0}^{n-1}r^{k} = 1 + r + r^{2} + r^{3} + \dots + r^{n-1} = \frac{1-r^{n}}{1-r}.
		\]

	\item MORE EXERCISES TO BE WRITTEN\ldots
\end{enumerate}

\subsection{Solutions}
\begin{enumerate}
	\item For each of the sets we first write how to read the notation in words, followed by its explicit form:
		\begin{enumerate}[label={(\roman*)}]
			\item Any \textbf{natural number} such that it is bigger than $1$ and smaller or equal to $7$. These are of course the numbers
				\[
					\left\{2,3,4,5,6,7\right\}.
				\]

			\item Any \textbf{integer} such that it is smaller than $5$. These are the numbers
				\[
					\left\{4,3,2,1,0,-1,-2,-3,\dots\right\}.
				\]

			\item Any \textbf{real number} $x$ such that $x^{2}=-1$. Since for any $x\in\mathbb{R},\ x^{2}\geq0$ - there is no such real number $x$ whose square equals $-1$. Therefore this definition describes the empty set, i.e. $\emptyset$.

			\item Any \textbf{natural number} that is also a \text{rational number}. Since any natural number is also a rational number (e.g. $4=\frac{4}{1}=\frac{8}{2}$, etc.) the definition actually simply describes the set of natural numbers, $\mathbb{N}$. This fact can also be written as
				\[
					\mathbb{N} \cap \mathbb{Q} = \mathbb{N}.
				\]

			\item Any \textbf{real number} such that it solves the equation $x^{2}-3x-4=0$. The solutions can be found using the quadratic formula:
				\[
					\frac{3\pm\sqrt{3^{2}+4\cdot4}}{2} = \frac{3\pm\sqrt{9+16}}{2} = \frac{3\pm5}{2} = 4,-1.
				\]
				Therefore the set described by the definition is simply
				\[
					\left\{4,-1\right\}.
				\]

			\item Any \textbf{real number} that is smaller than $5$ \textbf{and} is bigger than or equal to $2$. This definition describes the half-open interval
				\[
					[2,5).
				\]
		\end{enumerate}

	\item Relations between sets:
		\begin{enumerate}[label={(\roman*)}]
			\item All the elements in the set $B$ are also in the set $A$ ($1,2$), but there's an element in $A$ which is not in $B$ (namely $3$). Therefore, $B$ is a subset of $A$:
				\[
					B\subset A.
				\]

			\item The empty set is a subset of any set (and a proper subset of any set except itself), therefore
				\[
					A\subset B.
				\]

			\item The set $B$ is defined as all the natural numbers, their negatives and zero. This is exactly the definition of the integers $\mathbb{Z}$, which set $A$ in this case. Therefore
				\[
					A=B.
				\]

			\item All of the elements in $A$ are irrational numbers. The set $B$ is the set of \textbf{rational numbers}, and therefore the sets are disjoined:
				\[
					A\cap B = \emptyset.
				\]
		\end{enumerate}

	\item $S^{2}$ is a Cartesian product of $S$ with itself:
		\[
			S^{2} = \left\{ (\alpha,\alpha), (\alpha,\beta), (\alpha,\gamma), (\beta,\alpha), (\beta,\beta), (\beta,\gamma), (\gamma,\alpha), (\gamma,\beta), (\gamma,\gamma) \right\}.
		\]
		Therefore, to form the Cartesian product $S^{2}\times W$ we simply take each of the elements is $S^{2}$ and add to it an element from $W$:
		\begin{align*}
			S^{2}\times W =
			\left\{(\alpha,\alpha,x), (\alpha,\beta,x), (\alpha,\gamma,x), (\beta,\alpha,x), (\beta,\beta,x), (\beta,\gamma,x), (\gamma,\alpha,x), (\gamma,\beta,x), (\gamma,\gamma,x)\right.\\
			\left.(\alpha,\alpha,y), (\alpha,\beta,y), (\alpha,\gamma,y), (\beta,\alpha,y), (\beta,\beta,y), (\beta,\gamma,y), (\gamma,\alpha,y), (\gamma,\beta,y), (\gamma,\gamma,y)\right.\\
			\left.(\alpha,\alpha,z), (\alpha,\beta,z), (\alpha,\gamma,z), (\beta,\alpha,z), (\beta,\beta,z), (\beta,\gamma,z), (\gamma,\alpha,z), (\gamma,\beta,z), (\gamma,\gamma,z)\right\}.
		\end{align*}

		Note that the number of elements in $S$ is $3$, and so the number of elements in $S^{2}$ is $3\times3=9$. The number of elements in $W$ is also $3$, and so the number of elements in $S^{2}\times W$ is $9\times3=27$.

		The Cartesian product $W\times S^{2}$ has the same structure as $S^{2}\times W$, except that the elements from $W$ are now on the left (remember that the order of elements is important in tuples, unlike with sets):
		\begin{align*}
			S^{2}\times W =
			\left\{(x,\alpha,\alpha), (x,\alpha,\beta), (x,\alpha,\gamma), (x,\beta,\alpha), (x,\beta,\beta), (x,\beta,\gamma), (x,\gamma,\alpha), (x,\gamma,\beta), (x,\gamma,\gamma)\right.\\
			\left.(y,\alpha,\alpha), (y,\alpha,\beta), (y,\alpha,\gamma), (y,\beta,\alpha), (y,\beta,\beta), (y,\beta,\gamma), (y,\gamma,\alpha), (y,\gamma,\beta), (y,\gamma,\gamma)\right.\\
			\left.(z,\alpha,\alpha), (z,\alpha,\beta), (z,\alpha,\gamma), (z,\beta,\alpha), (z,\beta,\beta), (z,\beta,\gamma), (z,\gamma,\alpha), (z,\gamma,\beta), (z,\gamma,\gamma)\right\}.
		\end{align*}

		One way of ensuring that $S^{2}\times W = W\times S^{2}$ is by making all tuples equal, i.e. if
		\[
			\alpha=\beta=\gamma=x=y=z,
		\]
		then
		\[
			S^{2}\times W = \left\{ (\alpha,\alpha,\alpha) \right\} = W\times S^{2}.
		\]

	\item We start by counting the number of possible bijective functions $f_{2}:\{1,2\}\to\{\alpha,\beta\}$. For each element in the domain of $f_{2}$ there are two options to connect to an element in the function's image: either $1$ or $2$. So we can have
		\begin{align*}
			1&\mapsto\alpha,\ \text{or}\\
			1&\mapsto\beta.
		\end{align*}
		(recall that the symbol $x\mapsto y$ means that the element $x$ is mapped by the function to the element $y$)

		For each of the above options, there is only a single option left for the element $2$:
		\begin{align*}
			2&\mapsto\beta  \ \text{if}\ 1\mapsto\alpha,\\
			2&\mapsto\alpha \ \text{if}\ 2\mapsto\beta.
		\end{align*}

		Therefore, we have two possible choices for mapping $1$, each of which dictates the choice for mapping $2$. Altogether there are two such bijective functions:

		\vspace{2em}
		\begin{center}
		\begin{tikzpicture}
			\Large
			
			\draw[thick, xred, fill=xred!20] (0,0) ellipse (0.25cm and 1cm);
			\node (1) at (0,0.5cm) {$1$};
			\node (2) at (0,-0.5cm) {$2$};
			
			\draw[thick, xgreen, fill=xgreen!20] (1.5cm,0) ellipse (0.25cm and 1cm);
			\node (alpha) at (1.5cm,0.5cm) {$\alpha$};
			\node (beta) at (1.5cm,-0.5cm) {$\beta$};

			\draw[arrow] (1) -- (alpha);
			\draw[arrow] (2) -- (beta);
		\end{tikzpicture}
		\hspace{2cm}
		\begin{tikzpicture}
			\Large
			
			\draw[thick, xred, fill=xred!20] (0,0) ellipse (0.25cm and 1cm);
			\node (1) at (0,0.5cm) {$1$};
			\node (2) at (0,-0.5cm) {$2$};
			
			\draw[thick, xgreen, fill=xgreen!20] (1.5cm,0) ellipse (0.25cm and 1cm);
			\node (alpha) at (1.5cm,0.5cm) {$\alpha$};
			\node (beta) at (1.5cm,-0.5cm) {$\beta$};

			\draw[arrow] (1) -- (beta);
			\draw[arrow] (2) -- (alpha);
		\end{tikzpicture}
		\end{center}

	We can use the same logic for the case of $3$ elements in each set. Let's define the function as
	\[
		f_{3}:\{1,2,3\}\to\{\alpha,\beta,\gamma\}.
	\]

	We start with all the possibilities for connecting $1$:

	\begin{align*}
		1&\mapsto\alpha,\ \text{or}\\
		1&\mapsto\beta,\ \text{or}\\
		1&\mapsto\gamma.
	\end{align*}

	For each of the above choices, we have two remaining choices for connecting $2$ (since one element in the image of $f_{3}$ is already decided by our choice for $1$). Then, for the last element $3$ we remain with a single option only, since two elements in the image of $f_{3}$ are already connected to by $1$ and $2$. Altogether we therefore have
	\[
		3\times2\times1 = 6
	\]
	total bijective functions $f_{3}$.

	You probably already noticed the pattern: for a function
	\[
		f_{n}:\{n\ \text{elements}\} \to \{n\ \text{elements}\},
	\]
	we have $n$ choices for connecting the first element, then $n-1$ options for connecting the second element, then $n-2$ options for connecting the the third element\ldots and by the last element we have only a single option. Therefore, the number of bijective functions $f_{n}$ is
	\[
		n\times(n-1)\times(n-2)\times\dots\times3\times2\times1 = n!
	\]

	Note that indeed $2!=2$ and $3!=6$, which agrees with the results we got for $f_{2}$ and $f_{3}$, respectively.

	\item solution\ldots

	\item A function is bijective if and only if it is both a injective and surjective. There are two cases for $|A|\neq|B|$:
		\begin{enumerate}
			\item $|A|>|B|$, in which case there is at least one element in $A$ which is not connected to any element in $B$: otherwise, there are at least two elements in $A$ that connect to the same element in $B$. In the first case the relation is not a function, and in the second it is not injective and therefore not bijective.
			\item $|A|<|B|$, in which case there must be at least one element in $B$ that is not connected to by any element from $A$ (by the defintion of a function, there cannot be any element in $A$ that is connected to more than a single element in $B$). Therefore such a function is not surjective and thus not bijective.
		\end{enumerate}

	\item The polynomial $f(x)$ can be re-written as
		\[
			f(x) = x \left( x^{2}+x-6 \right).
		\]
		Therefore one of its roots are when $x=0$, and the other when $x^{2}+x-6=0$. Using the quadratic formula we get that $x^{2}+x-6=0$ when
		\[
			x_{1,2} = \frac{-1\pm\sqrt{1-4(-6)}}{2} = \frac{-1\pm\sqrt{25}}{2} = \frac{-1\pm5}{2} = 2,-3.
		\]
		Thus, altogether the roots of $f$ are $\{-3,0,2\}$.

	\item As with most logarithm-related proofs, we transform the problem into the realm of exponents: let us set $m=\log_{b}(x)$. We then get that $x=b^{m}$. If we raise both by to the $k$-th power, we get
		\begin{align*}
			x^{k} &= \left( b^{m} \right)^{k}\\
				  &= b^{mk}.
		\end{align*}
		Taking the logarithm in base $b$ of both sides of the above relation gives
		\begin{align*}
			\log_{b} \left( x^{k} \right) &= \log_{b} \left( b^{mk} \right)\\
										  &= mk\\
										  &= k\log_{b}(x).
		\end{align*}
		The last step results from our original defintion that $m=\log_{b}(x)$.

	\item Using the relation proved in the previous question and setting $k=-1$ we get
		\[
			\log_{b} \left( \frac{1}{x} \right) = \log_{b} \left( x^{-1} \right) = -1\cdot\log_{b}(x) = -\log_{b}(x).
		\]

	\item Using the logarithm base-change rule (\autoref{eq:log_base_change}), we set all logarithms to same base ($b=16$):
		\begin{align*}
			\log_{2}(x) &= \log_{16}(x)\cdot\log_{2}(16) = 4\log_{16}(x).\\
			\log_{4}(x-1) &= \log_{16}(x-1)\cdot\log_{4}(16) = 2\log_{16}(x-1).
		\end{align*}
		Therefore, the expression is equivalent to
		\[
			4\log_{16}(x) = 2\log_{16}(x-1) + \log_{16}(x^{3}).
		\]
		We can now bring the coefficients $4$ and $2$ inside the logarithm expression:
		\[
			\log_{16}(x^{4}) = \log_{16}(\left[x-1\right]^{2}) + \log_{16}\left(x^{3}\right).
		\]
		Using the logarithmic addition law, we get:
		\[
			\log_{16}(x^{4}) = \log_{16}(x^{3}\left[x-1\right]^{3}).
		\]
		We can now discard $\log_{16}$ on both sides, and we're left with a simple equation to solve:
		\[
			x^{4} = x^{3} \left( x-1 \right)^{2},
		\]
		the solutions of which are $x_{1}=0$ and $x_{2,3}=\frac{3\pm\sqrt{5}}{2}$, of which only $x_{2} = \frac{3+\sqrt{5}}{2}$ is valid: $x_{1}$ isn't valid since $x>0$, and $x_{3}$ isn't valid since $x_{3}-1<0$, and thus $\log_{b}\left(x_{3}-1\right)$ isn't defined over the real numbers.

	\item 
		\begin{enumerate}[label=(\roman*)]
			\item
		We start with drawing two radial lines from the center of the planet $\bm{C}$: one to the point on the surface where the observer meets the ground, and one to the horizon. The triangle $\triangle COX$ is a right triangle: the angle $\angle CXO=\ang{90}$:
		
		\begin{center}
			\begin{tikzpicture}
				\pgfmathsetmacro{\R}{2}
				\pgfmathsetmacro{\h}{0.6}
				\pgfmathsetmacro{\th}{asin(sqrt(2*\R*\h+\h^2)/(\R+\h))}
				\pgfmathsetmacro{\ph}{90-\th}
				\coordinate (O) at (0,0);
				\coordinate (obs) at (0,{\R+\h});
				\coordinate (top) at (0,\R);
				\coordinate (horiz) at ({\R*cos(\ph)},{\R*sin(\ph)});
				\draw[very thick, fill=xblue!20] (O) circle (\R);
				\fill[xpurple!40] (O) -- ({\R*0.4*cos(\ph)},{\R*0.4*sin(\ph)}) arc (\ph:90:{0.4*\R}) -- cycle;
				\draw[opacity=0] ({\R*0.25*cos(\ph)},{\R*0.25*sin(\ph)}) arc (\ph:90:{0.25*\R}) node [midway, opacity=1, xpurple] {$\theta$};
				\draw[ultra thick] (top) -- (obs) node [midway, left] {$h$};
				\draw[ultra thick, xred] (obs) -- (horiz) node [midway, above right] {$d_{a}$};
				\draw[thick, dashed] (O) -- (0,\R) node [midway, left] {$R$};
				\draw[thick, dashed] (O) -- (horiz) node [midway, below right] {$R$};
				\fill[xblue] (obs) circle (0.07) node [above] {$O$};
				\fill[xblue] (horiz) circle (0.07) node [above right] {$X$};
				\fill[xblue] (O) circle (0.07) node [below left] {$C$};
			\end{tikzpicture}
		\end{center}

	Using the Pythagorean theorem (with $R+h$ as the hypotenuse) we can calculate $d_{a}$:
	\[
		d_{a}^{2} + R^{2} = \left( R+h \right)^{2}.
	\]
	By expanding the right-hand side, cancelling $R^{2}$ and rearranging we get
	\[
		d_{a} = \sqrt{2Rh+h^{2}}.
	\]

	To get $d_{g}$ we need to find the angle $\theta$ between the lines $CX$ and $CO$. For that purpose we can use the law of sines (\autoref{eq:law of sines}):
	\[
		\frac{d_{a}}{\sin(\theta)} = \frac{R+h}{\sin(\ang{90})} = R+h.
	\]
	(since $\sin(\ang{90})=1$)

	Isolating $\sin(\theta)$ and subtituting the value of $d_{a}$ as function of $R$ and $h$ yields:
	\[
		\sin(\theta) = \frac{d_{a}}{R+h} = \frac{\sqrt{2Rh+h^{2}}}{R+h}.
	\]
	Therefore
	\[
		\theta = \arcsin \left( \frac{\sqrt{2Rh+h^{2}}}{R+h} \right).
	\]

	When $\theta$ is given in radians, the length $d_{g}$ then simply becomes
	\[
		d_{g} = R\theta = R\cdot\arcsin \left( \frac{\sqrt{2Rh+h^{2}}}{R+h} \right).
	\]

\item
	For an average person on Earth ($h=1.75\si{m},\ R=6.371\times10^{6}\si{m}$), standing at sea level, the air-distance to the horizon is therefore
	\[
		d_{a} = \sqrt{2Rh+h^{2}} = \sqrt{2\cdot6.371\times10^{6}\cdot1.75+1.75^{2}} \approx 4722\si{m} = 4.722\si{km}.
	\]
	The ground-distance, on the other hand, is
	\begin{align*}
		d_{g} &= R\cdot\arcsin \left( \frac{\sqrt{2Rh+h^{2}}}{R+h} \right)\\
			  &= 6.371\times10^{6}\si{m} \cdot \arcsin \left( \frac{4722\si{m}}{6.371\times10^{6}\si{m}+1.75\si{m}} \right)\\
			  &\approx 4722\si{m}.
	\end{align*}

\item Let us call the angle representing the drop of the horizon from eye-level $\textcolor{xgreen!75!black}{\bm{\varphi}}$:
		\begin{center}
			\begin{tikzpicture}
				\pgfmathsetmacro{\R}{2}
				\pgfmathsetmacro{\h}{0.6}
				\pgfmathsetmacro{\th}{asin(sqrt(2*\R*\h+\h^2)/(\R+\h))}
				\pgfmathsetmacro{\ph}{90-\th}
				\coordinate (O) at (0,0);
				\coordinate (obs) at (0,{\R+\h});
				\coordinate (top) at (0,\R);
				\coordinate (horiz) at ({\R*cos(\ph)},{\R*sin(\ph)});
				\draw[very thick, fill=xblue!20] (O) circle (\R);
				\fill[xpurple!40] (O) -- ({\R*0.4*cos(\ph)},{\R*0.4*sin(\ph)}) arc (\ph:90:{0.4*\R}) -- cycle;
				\draw[opacity=0] ({\R*0.25*cos(\ph)},{\R*0.25*sin(\ph)}) arc (\ph:90:{0.25*\R}) node [midway, opacity=1, xpurple] {$\theta$};
				\fill[xgreen!40] (obs) -- ($(obs)+(0.75,0)$) arc (0:-(\ph-8):0.75) -- cycle;
				\draw[opacity=0] ($(obs)+(0.6,0)$) arc (0:-(\ph-8):0.6) node [midway, opacity=1, xgreen!75!black] {$\varphi$};
				\draw[ultra thick] (top) -- (obs);
				\draw[ultra thick, xred] (obs) -- (horiz);
				\draw[thick, dashed] (O) -- (0,\R);
				\draw[thick, dashed] (O) -- (horiz);
				\draw[thick, dashed, xblue] ($(obs)-(1cm,0)$) -- ++(2cm,0);
				\fill[xblue] (obs) circle (0.07) node [above] {$O$};
				\fill[xblue] (horiz) circle (0.07) node [above right] {$X$};
				\fill[xblue] (O) circle (0.07) node [below left] {$C$};
			\end{tikzpicture}
		\end{center}

		Since $\triangle COX$ is a right triangle ($\angle CXO$ being the right angle), we know $\theta$ from previously and all angles in a triangle sum up to $\deg{180}$, the angle $\angle COX$ is equal to $\ang{90}-\theta$. This in turn means that $\varphi$ is equal to
		\[
			\ang{90} - \left( \ang{90}-\theta \right) = \theta = \arcsin \left( \frac{\sqrt{2Rh+h^{2}}}{R+h} \right).
		\]

	\vspace{1em}
	The following table sums up all the (approximate) air- and ground-distances to the horizon and the drop of the horizon from eye-level for each of the heights mentioned in the exercise:
	\begin{center}
		\begin{tabular}{lllll}
			\toprule
			Position & Height [\si{m}] & $d_{a} [\si{km}]$ & $d_{g} [\si{km}]$ & $\theta [\si{\degree}]$\\
			\midrule
			Person at sea level & 1.75 & 4.7 & 4.7 & 0.04 \\
			Eiffel Tower observation & 165 & 45.8 & 45.8 & 0.41 \\
			Average cruising altitude & 9100 & 340.6 & 340.3 & 3.06 \\
			Internation Space Station & 408000 & 2316 & 2221 & 19.98 \\
			\bottomrule
		\end{tabular}
	\end{center}

	\vspace{1em}
	Note that as the height $h$ grows, the difference between $d_{a}$ and $d_{g}$ grows too. We can see this clearly when plotting $d_{a}(h)$ and $d_{g}(h)$ in the same graph (disregarding the units and values for now, since we are only interested in the qualitative behaviour of both distances):

	\begin{center}
		\begin{tikzpicture}
			\pgfmathsetmacro{\Re}{6.371}
			\begin{axis}[
				graph2d,
				width=10cm, height=10cm,
				xmin=0, xmax=4,
				domain=0:4,
				restrict y to domain=0:10,
				axis line style={-stealth, thick},
				xlabel={$h$},
				ylabel={Distance},
				xticklabels={,},
				yticklabels={,},
				grid style={line width=.1pt, draw=gray!10},
				major grid style={line width=.2pt,draw=gray!20},
				declare function={da(\h)=sqrt(2*\Re*\h+\h^2);},
				declare function={dg(\h)=\Re*rad(asin(sqrt(2*\Re*\h+\h^2)/(\Re+\h)));},
			]
			\addplot[ultra thick, xblue] {da(x)} node[pos=0.5, above, font=\Large, yshift=2mm] {$d_{a}$};
			\addplot[ultra thick, xred] {dg(x)} node[pos=0.5, below, font=\Large, xshift=1mm] {$d_{g}$};
			\end{axis}
		\end{tikzpicture}
	\end{center}

	For small values of $h$ the two functions are very close to eachother, and as $h$ grows they grow apart, with $d_{a}>d_{g}$.
\end{enumerate}

	\item
		\begin{descitemize}
			\item [Algebraic form] we simply expand all parantheses and multiply everything:
				\begin{align*}
					\left(\sqrt{3}+\iu\right)^{2}\left(-2+\sqrt{12}\iu\right) &= \left(\sqrt{3}+\iu\right)\left(\sqrt{3}+\iu\right)\left(-2+\sqrt{12}\iu\right)\\
																			  &= \left( 3+\sqrt{3}\iu+\sqrt{3}\iu-1 \right) \left( -2+\sqrt{12}\iu \right)\\
																			  &= \left( 2+2\sqrt{3}\iu \right) \left( -2+\sqrt{12}\iu\right)\\
																			  &= 2 \left( 1+\sqrt{3}\iu \right) \left( -2+\sqrt{4\cdot3}\iu \right)\\
																			  &= 2 \left( 1+\sqrt{3}\iu \right) \left( -2+2\sqrt{3}\iu \right)\\
																			  &= 4 \left( 1+\sqrt{3}\iu \right) \left( -1+\sqrt{3}\iu \right)\\
																			  &= 4 \left( -1+\sqrt{3}\iu-\sqrt{3}\iu-3 \right)\\
																			  &= 4 \left( -4 \right)\\
																			  &= -16.
				\end{align*}

			\item [Polar form] first we use \autoref{eq:complex_components_geometric} to find the polar form of the two complex numbers:
				\begin{align*}
					z_{1} = \sqrt{3}+\iu &\Rightarrow 
					\begin{cases}
						&r_{1}=\sqrt{3+1}=2,\\
						&\theta_{1}=\arctan \left( \frac{1}{\sqrt{3}} \right) = \frac{\pi}{6}.
					\end{cases}\\[2mm]
					z_{2} = -2+\sqrt{12} &\Rightarrow
					\begin{cases}
						&r_{2}=\sqrt{4+12}=4,\\
						&\theta_{2}=\arctan \left( -\frac{\sqrt{12}}{2} \right) = \frac{2\pi}{3}.
					\end{cases}\\
				\end{align*}
				Therefore, $z_{1}^{2}z_{2}$ in polar form is
				\begin{align*}
					z_{1}^{2}z_{2} &= \left(r_{1}e^{\theta_{1}\iu}\right)^{2}r_{2}e^{\theta_{2}\iu}\\
								   &= r_{1}^{2}r_{2}e^{\left(2\theta_{1}+\theta_{2}\right)\iu}\\
								   &= 2^{2}\cdot4e^{\left(\frac{2\pi}{6}+\frac{2\pi}{3}\right)\iu}\\
								   &= 16e^{\pi\iu}.
				\end{align*}
				Since $e^{\pi\iu} = -1$ (\autoref{eq:Euler's identity}), we get that indeed
				\[
					z_{1}^{2}z_{2} = -16,
				\]
				just as we got in the algebraic form.
		\end{descitemize}

	\item It is easy to see that for even values of $n$ the statement holds: for each $w_{k}$ theres an oposing $w_{m}$ ($m\neq k$) such that $w_{k}+w_{m}=0$. See for example $n=4$ and $n=6$ in \autoref{fig:nth roots of 1}.

		For a more general proof which includes the odd values of $n$ we must work a bit harder. Recall that the $k$-th root of the equation $z^{n}=1$ has the following form (\autoref{eq:nth roots of 1}):
		\[
			w_{k} = \eu^{\frac{2\pi}{n}\iu k}.
		\]

		We can re-write the sum of the roots as
		\[
			\sum\limits_{k=0}^{n-1} \left(\eu^{\frac{2\pi\iu}{n}}\right)^{k},
		\]
		(since $x^{ab} = (x^{a})^{b}$)

		Using the hint we note that in this sum $r=\eu^{\frac{2\pi\iu}{n}}$, and thus
		\def\lsk{1.25\normalbaselineskip}
		\begin{align*}
			\sum\limits_{k=0}^{n-1} \left( \eu^{\frac{2\pi\iu}{n}} \right)^{k} &= \frac{1-\left(\eu^{\frac{2\pi\iu}{n}}\right)^{n}}{1-\eu^{\frac{2\pi\iu}{n}}} = \frac{1-\eu^{\frac{2\pi\iu}{n}n}}{1-\eu^{\frac{2\pi\iu}{n}}} = \frac{1-\eu^{2\pi\iu}}{1-\eu^{\frac{2\pi\iu}{n}}} = \frac{1-1}{1-\eu^{\frac{2\pi\iu}{n}}} = 0.
		\end{align*}
\end{enumerate}
